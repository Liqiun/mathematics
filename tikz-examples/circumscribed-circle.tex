\documentclass[tikz, border = 10pt]{standalone}

%%%<
\usepackage{verbatim}
\usetikzlibrary{intersections, through}
%%%>

\begin{comment}
:Title: Construction of the circumscribed circle of an arbitrary triangle
:Author: Moti Ben-Ari
:Tags: Geometry;Intersections;Through;Straightedge and compass
:Slug: Circumscribed circle of a triangle

TikZ features: name path, name intersections, circle through

  Construction
    Given an arbitrary triangle, its circumscribed circle is
    constructed by straightedge and compass.
    The three perpendicular bisectors intersect in a single point
    which is the center of the circumscribed circle.

  Method:
    1. Construct perpendicular bisectors of two sides. A bisector
       is constructed by drawing two circles whose centers are
       the vertices of the side and whose radius
       is the length of the side.

    2. "circle through" is used to draw the circles through the
       two vertices of each side and "intersections" is used
       to identify the intersections of the circles.
       
    3. The line joining the intersections of the circles is the
       perpendicular bisector.

    4. The intersection of two perpendicular bisectors is the
       center of the circumscribed circle.
       The circle is constructed using "circle through"
       one of the vertices.
       
    5. To verify that all three bisectors meet at a single point,
       the third perpendicular bisector is constructed.
       A circumscribed circle is constructed using the intersection
       of the third bisector and one of the previous ones.
       If the intersection is the same the picture will not change.

  Notes:
    1. An example by Sam Britt first draws the circle and then
       locates the vertices of the triangle on the circumference.
       This example follows the method for constructing the
       circumscribed circle starting from an arbitrary triangle.

    2. Instead of using "circle through", it would be possible to
       use "veclen" to obtain the raidus.
       However, Euclid's compass did not use lengths and, in fact,
       it was defined as a "collapsing" compass that lost its length
       if lifted from the paper. See: 
         Godfried Toussaint.
         A New Look at Euclid's Second Proposition,
         _The Mathematical Intelligencer_ 15(3), 1993, 12--23.
       and
         "Surprising constructions with straightedge and compass"
         https://www.weizmann.ac.il/sci-tea/benari/mathematics.

    3. The circles used to construct the bisectors are large.
       A "\clip" command is used to obtain a smaller picture.

    4. The final picture is cluttered and is simplified by not
       drawing the circles used to obtain the perpendicular
       bisectors. You can obtain the full picture by removing
       the comment on the option "draw" as indicated below.
\end{comment}

\begin{document}

\begin{tikzpicture}
  
  % Comment out the clip command to obtain the full picture.
  \clip (-4, -9) rectangle (16, 11);
  
  % Define the coordinates of an arbitrary triangle,
  %   label the vertices and draw dots at each vertex.

  \coordinate[label = left:$A$]  (A) at ( 0, 0);
  \coordinate[label = right:$B$] (B) at (10, 0);
  \coordinate[label = above:$C$] (C) at ( 7 ,8);
  \fill (A) circle (2pt);
  \fill (B) circle (2pt);
  \fill (C) circle (2pt);
  
  % Draw the triangle.

  \draw[very thick] (A) -- (B) -- (C) -- cycle;
  
  % Construct the perpendicular bisector of side AB.
  
  % Assign path names to the circles.

  \node[%draw,   % Uncomment to display the circle
    red, thick, name path = ABcircle] at (A)
    [circle through = (B)] {};
  \node[%draw,   % Uncomment to display the circle
    red, thick, name path = BAcircle] at (B)
    [circle through = (A)] {};
  
  % Obtain the two intersections of the circles and assign names.

  \path [name intersections = {% % The comment is necessary!
    of = ABcircle and BAcircle,
    by = {ab1, ab2}
    }];
  
  % Draw the perpendicular bisector as a line whose end points are
  %   the two intersections and draw dots are each point.

  \draw[red, thick, name path = ABperp] (ab1) -- (ab2);
  \fill[red] (ab1) circle (2pt);
  \fill[red] (ab2) circle (2pt);

  % Construct the perpendicular bisector of side AC.
  % The method is the same as the method for side AB.

  \node[%draw,   % Uncomment to display the circles
    teal, thick, name path = ACcircle] at (A)
    [circle through = (C)] {};
  \node[%draw,   % Uncomment to display the circles
    teal, thick, name path = CAcircle] at (C)
    [circle through = (A)] {};
  
  \path [name intersections = {% % The comment is necessary!
    of = ACcircle and CAcircle,
    by = {ac1, ac2}
    }];
  
  \draw[teal, thick, name path = ACperp] (ac1) -- (ac2);
  \fill[teal] (ac1) circle (2pt);
  \fill[teal] (ac2) circle (2pt);
  
  % Find the intersection of two perpendicular bisectors AB and AC.
  %   The intersection is the center of the circumscribed circle.

  \path [name intersections = {% % The comment is necessary!
     of = ABperp and ACperp,
     by = {center1}
     }];
  \fill (center1) circle(4pt);
  
  % Draw the circumscribed circle.

  \node[draw, very thick] at (center1)
    [circle through = (A)] {};
  
  % Construct the third perpendicular bisector to see if it
  %   intersects the other bisectors at the same point.
  
  % Construct the perpendicular bisector of side BC.
  % The method is the same as the method for side AB.
  
  \node[%draw,   % Uncomment to display the circles
    blue, thick, name path = BCcircle]
    at (B) [circle through = (C)] {};
  \node[%draw,   % Uncomment to display the circles
    blue, thick, name path = CBcircle]
    at (C) [circle through = (B)] {};
  
  \path [name intersections = {% % The comment is necessary!
    of = BCcircle and CBcircle,
    by = {bc1, bc2}
    }];
  
  \draw[blue, thick, name path = BCperp] (bc1) -- (bc2);
  \fill[blue] (bc1) circle (2pt);
  \fill[blue] (bc2) circle (2pt);
  
  % Locate the intersection of the perpendicular bisectors
  %   AC and BC. 
  %   This is also the center of the circumscribed circle.

  \path [name intersections = {% % The comment is necessary!
    of = BCperp and ACperp,
    by = {center2}
    }];
  \fill (center2) circle(4pt);
  
  % Draw another circumscribed circle.

  \node[draw, very thick] at (center2)
    [circle through = (B)] {};
  
  % If you see two distinct circles with distinct centers 
  %   something is wrong!!!
  
\end{tikzpicture}
\end{document}
