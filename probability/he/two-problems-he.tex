% !TeX root = two-problems-he.tex

%%%%%%%%%%%%%%%%%%%%%%%%%%%%%%%%%%%%%%%%%%%%%%%%%%%%%%%%%%%%%%%%

\documentclass[11pt,a4paper]{article}

\usepackage[utf8x]{inputenc}
\usepackage[english,hebrew]{babel}

\usepackage{verbatim}
\usepackage{url}

\usepackage{tikz}
\usetikzlibrary{intersections,calc,through,arrows.meta}
\tikzset {>=Stealth}

\textwidth=155mm
\textheight=230mm
\topmargin=0pt
\headheight=0pt
\oddsidemargin=0mm
\evensidemargin=0mm
\headsep=0pt
\parindent=0pt
\renewcommand{\baselinestretch}{1.15}
\setlength{\parskip}{0.3\baselineskip plus 1pt minus 1pt}

\newcommand*{\disfrac}[2]{\displaystyle\frac{#1}{#2}}

\begin{document}

\thispagestyle{empty}

\begin{center}
\textbf{\LARGE שתי בעיות בהסתברות}

\bigskip

\textbf{\large מוטי בן-ארי}

\bigskip

\selectlanguage{english}
\url{http://www.weizmann.ac.il/sci-tea/benari/}

\smallskip

\selectlanguage{hebrew}
\begin{footnotesize}
מוטי בן-ארי
\L{\copyright{}\ 2020 }
\end{footnotesize}
\end{center}

\selectlanguage{english}
\begin{footnotesize}
This work is licensed under the Creative Commons Attribution-ShareAlike 3.0 Unported License. To view a copy of this license, visit \url{http://creativecommons.org/licenses/by-sa/3.0/} or send a letter to Creative Commons, 444 Castro Street, Suite 900, Mountain View, California, 94041, USA.
\end{footnotesize}

\selectlanguage{hebrew}

\bigskip
\bigskip

מסמך זה מבוסס על שתי הבעיות הראשונות ב:

\selectlanguage{english}
$\quad$ Frederick Mosteller. \textit{Fifty Challenging Problems in Probability with Solutions}, Dover, 1965.
\selectlanguage{hebrew}

הבעיות מתאימות לתלמידים בבתי ספר תיכוניים. הפתרון שלי לבעיה הראשונה שונה בתכלית מפתרונו של
\L{Mosteller},
ומראה שלעתים יש פתרונות שונים לאותה בעיה. הבעיה השנייה מעניינת כי הפתרון נוגד את האינטואיציה.
\L{Mosteller}
מראה שהפתרון שנוגד את האיטואיציה ברור מאליו כאשר מתנתחים את הבעיה.


\section{שליפת גרביים ממגירה}

\begin{quote}
במגירה נמצאים גרביים אדומים וגרביים שחורים. אם נשלוף שני גרביים )ללא החזרה( בצורה אקראית, ההסתברות ששני הגרביים אדומים היא 
$\frac{1}{2}$. 
\begin{enumerate}
\item 
מה המספר הקטן ביותר של גרביים שחורים שיכולים להיות במגירה? עבור מספר זה של גרביים שחורים, כמה גרביים אדומים נמצאים במגירה?
\item 
מה המספר הקטן ביותר של גרביים שחורים שיכולים להיות במגירה אם נדרוש שהמספר יהיה
\textbf{זוגי}.
עבור מספר זה של גרביים שחורים, כמה גרביים אדומים נמצאים במגירה?
\end{enumerate}
\end{quote}

יהי
$r$
מספר הגרביים האדומים במגירה ויהי
$b$
מספר הגרביים השחורים. ברור ש-%
$r\geq 2, b\geq 1$.

\subsection*{הפתרון שלי}

נכפיל את ההסתברות לשתי שליפות )תזכרו שהגרב הראשון
\textbf{לא}
מוחזר למגירה( ונקבל את המשוואה:
\[
\frac{r}{r+b} \cdot \frac{(r-1)}{(r-1)+b} = \frac{1}{2}\,.
\]
נכפיל ונפשט ונקבל משוואה ריבועית במשתנה 
$r$:
\[
r^2-r(2b+1)-(b^2-b)=0\,.
\]
$b,r$
הם מספרים שלמים חיוביים ולכן הדיסקרימיננט של המשוואה הריבועית:
\[
(2b+1)^2+4(b^2-b)=8b^2+1\,,
\]
חייב להיות ריבוע של מספר שלם.

הדיסקרימיננט הוא ריבוע כאשר
$b=1$
)הערך הקטן ביותר(. לכן:
\[
r= \disfrac{(2\cdot 1 + 1)+ \sqrt{9}}{2}=3\,,
\]
כאשר הפתרון 
$r=0$
אינו מתקבל כי
$r\geq 2$.

בדיקה:
\[
\disfrac{3}{4}\cdot\disfrac{2}{3}=\disfrac{1}{2}\,.
\]

\bigskip

מהו המספר הזוגי הקטן ביותר עבור 
$b$
כדי שהדיסקרימיננט יהיה ריבוע?
\begin{displaymath}
\renewcommand{\arraystretch}{1}
\begin{array}{r|r|r}
b&8b^2+1&\sqrt{8b^2+1}\\
\hline
2&33&5.74\\
4&129&11.36\\
6&289&17
\end{array}
\end{displaymath}
מספר הגרביים האדומים הוא:
\[
r=\disfrac{(2\cdot 6+1)+\sqrt{289}}{2}=15\,.
\]
בדיקה:
\[
\disfrac{15}{21}\cdot\disfrac{14}{20}=\disfrac{\not 3\cdot \not 5}{\not 3\cdot \not 7}\cdot\disfrac{2\cdot \not 7}{4\cdot \not 5}=\disfrac{1}{2}\,.
\]

\subsection*{הפתרון של 
\L{Mosteller}}

האי-שוויון שלהלן נכון:
\[
\frac{r}{r+b} > \frac{r-1}{(r-1)+b}\,.
\]
בגלל ש-%
$r\geq 2, b\geq 1$,
ניתן להכפיל ולפשט. התוצאה היא
$0>-b$,
טענה נכונה.

מכאן ש:
\[
\left(\frac{r}{r+b}\right)^2 = \frac{r}{r+b} \cdot\frac{r}{r+b} > \frac{r}{r+b} \cdot \frac{r-1}{(r-1)+b} = \frac{1}{2}\,,
\]
ובאופן דומה:
\[
\left(\frac{r-1}{(r-1)+b}\right)^2  = \frac{r-1}{(r-1)+b}\cdot \frac{r-1}{(r-1)+b}<  \frac{r}{r+b} \cdot \frac{r-1}{(r-1)+b} = \frac{1}{2}\,.
\]
בשתי האי-שוויונות המכנה שונה מאפס כך שהריבועים סופיים וניתן לחשב שורש ריבועי משתיהן:
\[
\frac{r}{r+b}  > \sqrt{\frac{1}{2}} > \frac{r-1}{(r-1)+b}  \,.
\]
לא קשה להראות שאי-שוויון הראשון שקול ל:
\[
r>\frac{b}{\sqrt{2}-1}=(\sqrt{2}+1)b\,,
\]
ושהאי-שוויון השני שקול ל:
\[
(\sqrt{2}+1)b>r-1\,,
\]
כך ש:
\[
r-1<(\sqrt{2}+1)b<r\,.
\]

עבור 
$b=1$:$\quad$
$2.141 < r< 3.141$ 
וראינו לעיל ש-%
$b=1,r=3$
הוא פתרון.

\bigskip

ננסה מספרים זוגיים עבור
$b$
ונקבל:
\begin{displaymath}
\renewcommand{\arraystretch}{1}
\begin{array}{r|c|c}
b& <r< &r\\
\hline
2&4.8<r<5.8&5\\
4&9.7<r<10.7&10\\
6&14.5<r<15.5&15
\end{array}
\end{displaymath}
נבדוק כמו לעיל ש-%
$b=6,r=15$
הוא פתרון.

\L{Mosteller}
מעיר שיש קשר בין בעיה זו ותורת המספרים המתקדמת, ומביא פתרון נוסף:
$b=35,r=85$
)בדקו!(.

\newpage

\section{משחק טניס}

\begin{quote}
כדי לעודד את קריירת הטניס של אדוה, מציעים לה פרס אם היא מנצחת בלפחות שני משחקונים 
\textbf{ברצף}
מתוך מערכה של שלושה משחקונים. היא משחקת לסירוגין עם אביה ועם אלוף המועדון, והיא רשאית לבחור אם לשחק תחילה עם אביה, אח"כ עם האלוף ואח"כ עם אביה, או תחילה עם האלוף, אח"כ עם אביה ואח"כ עם האלוף. האלוף הוא שחקן
\textbf{טוב יותר}
מאביה. באיזה סדר כדאי לאדוה לבחור?
\end{quote}

אדוה זוכה בפרס אם: )א( היא מנצחת בשני המשחקונים הראשונים ומפסידה בשלישי, )ב( היא מנצחת בשני המשחקונים האחרונים ומפסידה בראשון, )ג( היא מנצחת בשלושת המשחקונים.

תהי 
$f$
ההסתברות לנצחון של אדוה במשחקון נגד אביה, ותהי 
$c$
ההסתברות לנצחון שלה במשחקון נגד האלוף. נתון ש-%
$c<f$.

ההסתברות שאדרה זוכה בפרס אם היא משחקת בסדר אבא-אלוף-אבא היא:
\[
fc(1-f) + (1-f)cf + fcf\,.
\]
ההסתברות שאדרה זוכה בפרס אם היא משחקת בסדר אלוף-אבא-אלוף היא:
\[
cf(1-c)+(1-c)fc+cfc\,.
\]
כדאי לאדוה לבחור את הסדר אבא-אלוף-אבא אם:
\begin{eqnarray*}
fc(1-f) + (1-f)cf + fcf & \stackrel{?}{>}& cf(1-c)+(1-c)fc+cfc\\
-fcf & \stackrel{?}{>}& -cfc\\
-f & \stackrel{?}{>}& -c\\
f & \stackrel{?}{<}& c\,.
\end{eqnarray*}
נתון ש-%
$c<f$,
כך שעדיף שאדוה תבחר את הסדר אלוף-אבא-אלוף.

התוצאה נוגדת את האינטואיציה. לפי האינטואיציה, עדיף שאדוה תשחק יותר משחקונים נגד אביה, ופחות משחקונים נגד האלוף, כי יש לה סיכוי טוב יותר לנצח את אביה. אולם, ברור שאדוה תזכה בפרס רק אם היא תזכה במשחקון
\textbf{האמצעי}.
לכן עליה לבחור בסדר בו היא משחקת נגד אביה -- השחקן החלש יותר -- במשחקון האמצעי.

\end{document}
