% !TeX root = trigonometric-functions.tex

\chapter{Additional Topics}\label{ch.additional}

\section{The secant and cosecant functions}

The secant and cosecant are defined on right triangles as $\sec x = \rule[-10pt]{0pt}{28pt}\disfrac{1}{\cos x}$, $\csc x = \rule[-10pt]{0pt}{28pt}\disfrac{1}{\sin x}$.
Can they be defined geometrically from a construction on the unit circle as we have done for the other trigonometric functions?
In Figure~\ref{fig.secant-and-cosecant} we claim that $OT=|\sec x|$ and $OC=|\csc x|$. 

\begin{figure}[H]
\begin{center}
\begin{tikzpicture}[scale=.75]
\draw[step=2cm,white!50!black,thin] (-6.05,-6.05) grid (6,6);
\draw[thin] (-6,0) -- (6,0);
\draw[thin] (0,-6) -- (0,6);
  \coordinate (A) at (-4,0);
  \coordinate (B) at (4,0);
  \coordinate (D) at (0,4);
  \coordinate (O) at (0,0);
  \node[draw, name path = circle] at (O)
    [circle through = (A)] {};
  \coordinate (point) at (140:7.6);
  \draw[thick,dashed,name path=side] ($(point)!2!(O)$) -- (point);
  \draw[dashed,thick,name path=tangent] (4,-6) -- (4,6);
  \draw[dashed,thick,name path=cotangent] (-6,4) -- (6,4);
  \path [name intersections = {of = circle and side, by = {P}}];
  \draw[thick,blue] (P) -- (P |- O) coordinate (Ps);
  \draw[blue,very thick] (P) -- 
    node[left,xshift=-2pt] {$\sin x$}  (Ps);
  \draw[very thick,teal] (O) -- node[below] {$\cos x$} (Ps);
  \draw (Ps) rectangle +(8pt,8pt);
  \path [name intersections = {of = side and tangent, by = {T}}];
  \draw[very thick] (B) -- node[right] {$\tan x$} (T);
  \path [name intersections = {of = side and cotangent, by = {C}}];
  \draw[very thick] (D) -- node[above] {$\cot x$} (C);
  \draw[thick,red] (B)
    arc[start angle=0, end angle=140, radius=4cm];
  \node[red] at (3.2,3.2) {\textsf{arc}};

  \coordinate (C1) at ($(C)+(230:-1)$);
  \coordinate (O1) at ($(O)+(230:-1)$);
  \coordinate (T1) at ($(T)+(230:-1)$);
  \draw[very thick,blue,dotted,<->] (C1) -- 
    node[above right,yshift=-4pt] {$\csc x$} (O1);
  \draw[very thick,teal,dotted,<->] (O1) -- 
    node[above right,xshift=-4pt] {$\sec x$} (T1);

  \fill (B) circle (2pt) node[above right] {$B$};
  \fill[blue] (Ps) circle (2pt) node[below left,black] {$E$};
  \fill[teal] (O) circle (2pt) node[below left,black] {$O$}
        node[above left,xshift=-10pt,black] {$x$}
        node[below right,xshift=10pt,black] {$x$};;
  \fill (D) circle (2pt) node[above right,yshift=4pt] {$D$};
  \fill (T) circle (2pt) node[right]     {$T$};
  \fill (C) circle (2pt) node[above,xshift=2pt,yshift=4pt] {$C$}
    node[below right,xshift=10pt] {$x$};
  \fill[red] (P) circle (2pt) node[right,xshift=4pt] {$P$};
\end{tikzpicture}
\caption{The secant and cosecant functions}\label{fig.secant-and-cosecant}
\end{center}
\end{figure}

\vspace*{-3ex}

Here is the proof that $OC=\csc x$:
\begin{itemize}
\item $CD \parallel EO$ so by alternate interior angles $\angle DCO=\angle EOC$.
\item In right triangles, if one pair of acute angles are equal, they are similar $\triangle PEO \sim \triangle ODC$.
\item Radii of the unit circle $OP = OD =1$.
\item By construction $PE = \sin x$.
\item Therefore:
\[
\frac{OC}{OP} = \frac{OD}{PE}\,, \quad\quad OC = OP\cdot\frac{OD}{PE}=1\cdot\left|\disfrac{1}{\sin x}\right|=|\csc x|\,.
\]
\end{itemize}
A similar proof shows that $\triangle PEO \sim \triangle TBO$, so $OT = \left|\disfrac{1}{\cos x}\right|=|\sec x|$.



\section{From functions to triangles}

Even when beginning the study of trigonometry with the functional approach, at some point the transition to the functions on right triangles must be done.
This is very easy since the measure of an arc is the same as the measure (in radians) of the central angle it subtends.

\begin{wrapfigure}[14]{r}{.5\textwidth}
\begin{center}
\vspace{-4ex}
\begin{tikzpicture}[scale=.6]
\draw[thin] (-4,0) -- (7.8,0);
\draw[thin] (0,-4) -- (0,4);
  \coordinate[label=below left:$A$] (A) at (-4,0);
  \coordinate[label=below right:$B$] (B) at (4,0);
  \coordinate (O) at (0,0);
  \node[draw, name path = circle] at (O)
    [circle through = (A)] {};
  \coordinate (D) at (30:9);
  \draw[very thick,violet,name path=side] (O) -- (D);
  \path [name intersections = {of = circle and side, by = {P}}];
  \draw[very thick,teal] (P) -- (P |- O) coordinate (C);
  \draw[rotate=90] (C) rectangle +(12pt,12pt);
  \draw[very thick,violet] (D) -- (D |-  O) coordinate (E) -- (O);
  \draw[rotate=90] (E) rectangle +(12pt,12pt);
  \draw[very thick,red] (B)
     arc[start angle=0, end angle=30, radius=4cm];
  \node[red] at (4.2,1.2) {$\alpha$};
  \fill[red] (P) circle (2pt)
    node[above,xshift=2pt,yshift=2pt] {$P$};
  \fill[purple] (E) circle (2pt) node[below] {$E$}; 
  \fill[teal] (C) circle (2pt) node[below] {$C$}; 
  \fill[purple] (D) circle (2pt) node[above] {$D$};;
  \fill (B) circle (2pt);
  \fill[purple] (O) circle (2pt) 
    node[below left] {$O$}
    node[above right,xshift=16pt] {$\alpha$};
\end{tikzpicture}
\caption{From functions to right triangles}\label{fig.functions-to-triangles}
\end{center}
\end{wrapfigure}
Any right triangle can be embedded in a coordinate system such that one side rests of the $x$-axis with the adjacent acute angle at the origin: $\triangle DEO$ in Figure~\ref{fig.functions-to-triangles}.
(Recall that in a right triangle two angles are acute.)
Now construct a unit circle centered at the origin and drop a perpendicular from point $P$, the intersection of the circle and the hypotenuse $OD$, to side $OE$.

For the right triangles, $\triangle DEO\sim \triangle PCO$, since $\angle DOE=\angle POC=\alpha$.
\[
\frac{PC}{OP}=\frac{DE}{OD}\,.
\]
$OP = 1$, the radius of the unit circle, and $PC = \sin \alpha$ by definition, so $\rule[-10pt]{0pt}{28pt}\disfrac{DE}{OD}=\sin\alpha$.
Therefore, the sine of $\alpha$ \emph{in the triangle} $\triangle DEO$ is equal to the opposite side divided by the hypotenuse, as expected.

Similar arguments show that $\disfrac{OE}{OD}=\cos\alpha$ and $\disfrac{DE}{OE}=\tan \alpha$.
