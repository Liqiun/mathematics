% !TeX root = trigonometric-functions.tex

\chapter{Additional Topics}



\section{The secant and cosecant functions}

The functions secant and cosecant are usually defined on right triangles as the inverses $\sec x = \disfrac{1}{\cos x}$, $\csc x = \disfrac{1}{\sin x}$.
Can these functions can be defined geometrically?
In other words, from the construction of the trigonometric functions (Figure~\ref{fig.secant-and-cosecant}), can we find their lengths?
We claim that $OT=|\sec x|$ and $OC=|\csc x|$. 

\begin{figure}[H]
\begin{center}
\begin{tikzpicture}[scale=.75]
\draw[step=2cm,white!80!black,ultra thin] (-6.05,-6.05) grid (6,6);
\draw[thin] (-6,0) -- (6,0);
\draw[thin] (0,-6) -- (0,6);
  \coordinate (A) at (-4,0);
  \coordinate (B) at (4,0);
  \coordinate (D) at (0,4);
  \coordinate (O) at (0,0);
  \node[draw, name path = circle] at (O)
    [circle through = (A)] {};
  \coordinate (point) at (140:7.6);
  \draw[thick,dashed,name path=side] ($(point)!2!(O)$) -- (point);
  \draw[dashed,thick,name path=tangent] (4,-6) -- (4,6);
  \draw[dashed,thick,name path=cotangent] (-6,4) -- (6,4);
  \path [name intersections = {of = circle and side, by = {P}}];
  \draw[thick,blue] (P) -- (P |- O) coordinate (Ps);
  \path[blue] (P) --  node[left,xshift=-2pt] {$\sin x$}  (Ps);
  \draw[very thick,teal] (O) -- node[below] {$\cos x$} (Ps);
  \draw (Ps) rectangle +(8pt,8pt);
  \path [name intersections = {of = side and tangent, by = {T}}];
  \draw[thick] (B) -- node[right] {$\tan x$} (T);
  \path [name intersections = {of = side and cotangent, by = {C}}];
  \draw[thick] (D) -- node[above] {$\cot x$} (C);
  \draw[thick,red] (B)
    arc[start angle=0, end angle=140, radius=4cm];
  \node[red] at (3.2,3.2) {\textsf{arc}};

  \coordinate (C1) at ($(C)+(230:-1)$);
  \coordinate (O1) at ($(O)+(230:-1)$);
  \coordinate (T1) at ($(T)+(230:-1)$);
  \draw[very thick,blue,dotted,<->] (C1) -- 
    node[above right] {$\csc x$} (O1);
  \draw[very thick,teal,dotted,<->] (O1) -- 
    node[above right] {$\sec x$} (T1);

  \fill (B) circle (2pt) node[above right] {$B$};
  \fill[blue] (Ps) circle (2pt) node[below left,black] {$E$};
  \fill[teal] (O) circle (2pt) node[below left,black] {$O$}
        node[above left,xshift=-10pt,black] {$x$}
        node[below right,xshift=10pt,black] {$x$};;
  \fill (D) circle (2pt) node[above right,yshift=4pt] {$D$};
  \fill (T) circle (2pt) node[right]     {$T$};
  \fill (C) circle (2pt) node[above,xshift=2pt,yshift=4pt] {$C$}
    node[below right,xshift=10pt] {$x$};
  \fill[red] (P) circle (2pt) node[right,xshift=4pt] {$P$};
\end{tikzpicture}
\caption{The secant and cosecant functions}\label{fig.secant-and-cosecant}
\end{center}
\end{figure}

Here is the proof that $OC=\csc x$:
\begin{itemize}
\item Radii of the unit circle: $OP = OD =1$.
\item By construction: $PE = \sin x$.
\item $CD \parallel EO$ so the alternate interior angles $\angle DCO, \angle EOC$ are equal.
\item Therefore, $\triangle PEO \sim \triangle ODC$, so:
\[
\frac{OC}{OP} = \frac{OD}{PE}=\left|\frac{1}{\sin x}\right|\,,\;\;\;\;\;
OC = 1\cdot\left|\disfrac{1}{\sin x}\right|=|\csc x|\,.
\]
\end{itemize}
A similar proof shows that $\triangle PEO \sim \triangle TBO$, so $OT = \left|\disfrac{1}{\cos x}\right|=|\sec x|$.



\section{From functions to triangles}

Even when beginning the study of trigonometry with the functional approach, at some point the transition to the functions on right triangles must be done.
This is very easy since the the measure of an arc is the same as the measure (in radians) of the central angle it subtends.

\begin{wrapfigure}[14]{r}{.5\textwidth}
\begin{center}
\vspace{-4ex}
\begin{tikzpicture}[scale=.6]
\draw[thin] (-4,0) -- (7.8,0);
\draw[thin] (0,-4) -- (0,4);
  \coordinate[label=below left:$A$] (A) at (-4,0);
  \coordinate[label=below right:$B$] (B) at (4,0);
  \coordinate (O) at (0,0);
  \fill (A) circle (1.5pt);
  \fill (B) circle (1.5pt);
  \fill[purple] (O) circle (1.5pt) 
    node[below left] {$O$}
    node[above right,xshift=16pt] {$\alpha$};
  \node[draw, name path = circle] at (O)
    [circle through = (A)] {};
  \coordinate (D) at (30:9);
  \draw[thick,violet,name path=side] (O) -- (D);
  \fill[purple] (D) circle (1.5pt) node[above] {$D$};;
  \path [name intersections = {of = circle and side, by = {P}}];
  \fill[red] (P) circle (1pt) node[black,above,xshift=2pt,yshift=2pt] {$P$};
  \draw[thick,teal] (P) -- (P |- O) coordinate (C);
  \fill (C) circle (1.5pt) node[below] {$C$}; 
  \draw[rotate=90] (C) rectangle +(8pt,8pt);
  \draw[thick,violet] (D) -- (D |-  O) coordinate (E) -- (O);
  \fill[purple] (E) circle (1.5pt) node[below] {$E$}; 
  \draw[rotate=90] (E) rectangle +(8pt,8pt);
  \draw[thick,red] (B) arc[start angle=0, end angle=30, radius=4cm];
  \node[red] at (4.2,1.2) {$\alpha$};
\end{tikzpicture}
\caption{From functions to right triangles}\label{fig.functions-to-triangles}
\end{center}
\end{wrapfigure}
Any right triangle can be embedded in a coordinate system such that one side rests of the $x$-axis with the adjacent acute angle at the origin, as shown by $\triangle DEO$ in Figure~\ref{fig.functions-to-triangles}.
(Recall that in a right triangle two angles are acute.)
Now construct a unit circle centered at the origin and drop a perpendicular from point $P$, the intersection of the circle and the hypotenuse $OD$, to side $OE$.

Since $\angle DEO=\angle PCO=\pi$ and $\angle DOE=\angle POC=\alpha$,  $\triangle DEO\sim \triangle PCO$, so:
\[
\frac{PC}{OP}=\frac{DE}{OD}\,.
\]
$OP = 1$, the radius of the unit circle and$PC = \sin \alpha$ by definition, so $\rule[-10pt]{0pt}{28pt}\disfrac{DE}{OD}=\sin\alpha$.
Therefore, the sine of $\alpha$ \emph{in the triangle} $\triangle DEO$ is equal to the opposite side divided by the hypotenuse, as expected.

Similar arguments show that $\disfrac{OE}{OD}=\cos\alpha$ and $\disfrac{DE}{OE}=\tan \alpha$.
