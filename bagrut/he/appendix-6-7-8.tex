% !TeX root = chapter-6-7-8.tex

\selectlanguage{hebrew}


\chapter{הערות על נוסחאות}

אני נוהג לחשב מספר חישובים בדרך שונה מעט ממה שנהוג. נספח זה אני מפרט את החישובים ומנמק את החלטותי.


\paragraph*{חיבור או חיסור של 
$\frac{\pi}{2}$}

חיבור או חיסור של
$\frac{\pi}{2}$
משנה סינוס לקוסינוס ולהיפך. קל להשתכנע כי במשולש ישר זווית, הזוויות החדות הן
$\theta$, $\frac{\pi}{2}-\theta$,
והצלע הנגדי של הזווית המופיע בהגדרת הסינוס מתחלף עם הצלע השכן של הפונקציה המופיע בהגדרת הקוסינוס.

נניח שהתחום הנתון בשאלה הוא
$0\leq x \leq \frac{\pi}{2}$,
ונניח שמקבלים תוצאה כגון
$2x=\frac{\pi}{3}$.
ברור שתשובה אחת היא
$x=\frac{\pi}{6}$,
כאשר
$0\leq \frac{\pi}{6} \leq \frac{\pi}{2}$.
אבל לא לשכוח שגם:
\erh{12pt}
\begin{equationarray*}{rcl}
2x&=&\frac{\pi}{3}+2\pi=\frac{7\pi}{3}\\
x&=&\frac{7\pi}{6}\,,
\end{equationarray*}
היא תשובה, למרות ש-%
$x=\frac{7\pi}{6}>\frac{\pi}{2}$
לא בתחום.

\paragraph*{נגזרות של פונקציות עם גורמים בחזקות לא שלמות וחיוביות}

בנוסחאון נתון:
\[
\textrm{(\R{ממשי} t)} \quad\quad (x^t)' = tx^{t-1}\,,
\]
אולם משתמשים בנוסחה רק עבור 
$t$
שלם וחיובי. אני מעדיף להשתמש בנוסחה עבור כל
$t$
כי קל לזכור את הנוסחה והחישובים פשוטים. למשל, הנוסחה הנתונה 
\[
(\sqrt{x})' = \disfrac{1}{2\sqrt{x}}
\]
מיותרת כי:
\[
(\sqrt{x})' = (x^\frac{1}{2})'= \frac{1}{2}(x^{-\frac{1}{2}}) = \frac{1}{2}\cdot\frac{1}{x^{\frac{1}{2}}}=\disfrac{1}{2\sqrt{x}}\,.
\]
החיסכון בולט יותר כאשר צריכים לחשב נגזרת רציונלית:
\[
\left(\frac{1}{x^t}\right)'\,.
\]
אני רואה שמשתמשים בנוסחה המוסבכת עבור נגזרת של מנה כאשר
$f(x)=1, g(x)=x^t$.
לדעתי, פשוט יותר לחשב:
\[
\left(\frac{1}{x^t}\right)'=(x^{-t})'=-t(x^{-t-1})=\frac{-t}{x^{t+1}}\,.
\]
כאשר במנה יש קבוע ובמכנה יש פונקציה מורכבת עדיין החישוב לא מסובך במיוחד. למשל:
\erh{12pt}
\begin{equationarray*}{rcl}
\left(\frac{14}{x^2-3x+4}\right)'&=&14\left((x^2-3x+4)^{-1}\right)'\\
%&=&14\cdot -1\cdot (x^2-3x+4)^{-2}(x^2-3x+4)'\\
&=&-14(x^2-3x+4)^{-2}(2x-3)\\
&=&\frac{-14(2x-3)}{(x^2-3x+4)^{2}}\,.
\end{equationarray*}

\np

\paragraph*{סינוס וקוסינוס של 
$2x$}

לעתים קרובות מופיעים נוסחאות עם 
$\sin 2x$
ו-%
$\cos 2x$.
אין צורך לזכור בעל פה את הנוסחאות המאפשרות לבטא את הפונקציות הללו עם
$x$
במקום
$2x$
כי ניתן לשחזר אותן תוך מספר שניות מהנוסחאות הטריגונומטריות עבור חיבור של זוויות, נוסחאות הניתנות בנוסחאון:
\erh{0pt}
\begin{equationarray*}{rcl}
\sin 2x &=& \sin (x+x) = \sin x \cos x + \sin x \cos x\\
& =& 2\sin x \cos x\\
\cos 2x &=& \cos (x+x) = \cos x \cos x - \sin x \sin x\\
& =& \cos^2 x - \sin^2 x\\
&=&(1-\sin^2 x) - \sin^2 x = 1 - 2\sin^2 x\,.
\end{equationarray*}

\paragraph*{נקודות חיתוך של שלילה של פונקציה}

אם
$f(x)=0$
אז
$-f(x)=-0=0$.
אמנם התוצאה פשוטה אבל היא שימושית כאשר חושבו נקודות איפוס של נגזרת וצריכים למצוא נקודות איפוס של השלילה של הנגזרת, למשל:
\[
(g(x)-f(x))' = (-1\cdot (f(x)-g(x))' = -1 (f(x)-g(x))'\,,
\]
ולכן אם 
$(f(x_1)-g(x_1))'=0$,
מתקבל מייד
$(g(x_1)-f(x_1))'=-1\cdot 0 = 0$.


\selectlanguage{english}
