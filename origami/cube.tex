% !TeX root = origami-math.tex

%%%%%%%%%%%%%%%%%%%%%%%%%%%%%%%%%%%%%%%%%%%%%%%%%%%%%%%%%%%%%%%%
\chapter{Doubling a Cube}\label{c.cube}

\section{Messer's doubling of a cube}\label{s.cube1}

A cube of volume $V$ has sides of length $\sqrt[3]{V}$. The volume of a cube with twice the volume is $2\cdot V$, so we need to construct the length $\sqrt[3]{2\cdot V}=\sqrt[3]{2}\cdot \sqrt[3]{V}$. If we can construct $\sqrt[3]{2}$, we can multiply by the given length $\sqrt[3]{V}$ to double the cube.

\subsection{Dividing a length into thirds}

Lang \cite{lang} shows efficient constructs for obtaining rational fractions of the length of the side of a square (piece of paper). Here, we need to divide the side of the square into thirds.

First, fold in half to locate the point $J=(1,1/2)$. Next, draw the lines $\overline{AC}$ and $\overline{BJ}$.
\begin{center}
\begin{tikzpicture}[scale=.8]
% Draw square
\coordinate (A) at (0,12);
\coordinate (B) at (0,0);
\coordinate (C) at (12,0);
\coordinate (D) at (12,12);

\fill (A) circle (2pt) node[left]  {$A=(0,1)$};
\fill (B) circle (2pt) node[left]  {$B=(0,0)$};
\fill (C) circle (2pt) node[right] {$C=(1,0)$};
\fill (D) circle (2pt) node[right] {$D=(1,1)$};

\draw [thick] (A)  -- (B) -- (C) -- (D) -- cycle;

% Divide a side in half

\coordinate (M)  at (0,6);
\coordinate (N) at (12,6);
\fill (M) circle (2pt) node[left] {$I=(0,1/2)$};
\fill (N) circle (2pt) node[right] {$J=(1,1/2)$};
\draw [thick,dashed] (M) -- (N);


\draw [thick,dotted,name path=ac] (A) -- 
   node[near start,above,xshift=24pt] {$y=1-x$} (C);
\draw [thick,dotted,name path=be2] (B) -- 
   node[near start,above,xshift=-12pt,yshift=-2pt] {$y=\disfrac{1}{2}x$} (N);

\path [name intersections = {of = ac and be2, by = {I}}];
\fill (I) circle (2pt) 
   node[below,xshift=-6pt,yshift=-8pt] {$K=$}
   node[below,xshift=-6pt,yshift=-20pt] {$(2/3,1/3)$};

\coordinate (E)  at (0,4);
\coordinate (F) at (12,4);
\fill (E) circle (2pt) node[left] {$E=(0,1/3)$};
\fill (F) circle (2pt) node[right] {$F=(1,1/3)$};
\draw [thick,dashed] (E) -- (F);

\coordinate (G)  at (0,8);
\coordinate (H) at (12,8);
\fill (G) circle (2pt) node[left] {$G=(0,2/3)$};
\fill (H) circle (2pt) node[right] {$H=(1,2/3)$};
\draw [very thick,dotted] (G) -- (H);
\end{tikzpicture}
\end{center}

The coordinates of their point of intersection $K$ is obtained by solving the two equations:
\begin{form}{1.8}
y&=&1-x\\
y&=&\disfrac{1}{2}x\,.
\end{form}
The result is $x=2/3, y=1/3$.

Construct the line $\overline{EF}$ perpendicular to $\overline{AB}$ that goes $K$, and construct the reflection $\overline{GH}$ of $\overline{BC}$ around $\overline{EF}$. The side of the square has now been divided into thirds.

\subsection{Computing $\sqrt[3]{2}$}

\begin{center}
\begin{tikzpicture}[scale=.9]
% Draw and label square
\coordinate (A) at (0,12);
\coordinate (B) at (0,0);
\coordinate (C) at (12,0);
\coordinate (D) at (12,12);
\fill (A) circle (2pt) node[left]  {$A$};
\fill (B) circle (2pt) node[left]  {$B$};
\fill (C) circle (2pt) node[right] {$C$};
\fill (D) circle (2pt) node[right] {$D$};
\draw (B) rectangle +(12pt,12pt);
\draw[rotate=90] (C) rectangle +(12pt,12pt);
\draw [thick] (A)  -- (B) -- (C) -- (D) -- cycle;

% Draw line one-third from botton
\coordinate (E)  at (0,4);
\coordinate (F) at (12,4);
\fill (E) circle (2pt) node[left] {$E$};
\fill (F) circle (2pt) node[right] {$F$};
\draw [very thick,dotted,name path=ef] (E) -- (F);

% Draw line two-thirds from bottom
\coordinate (G)  at (0,8);
\coordinate (H) at (12,8);
\fill (G) circle (2pt) node[left] {$G$};
\fill (H) circle (2pt) node[right] {$H$};
\draw[rotate=-90] (G) rectangle +(12pt,12pt);
\draw [very thick,dotted] (G) -- (H);

% Draw reflections of C and F
\coordinate (CP) at (0,5.31);
\coordinate (FP) at (2.96,8);
\fill (CP) circle (2pt)
  node[left] {$C'$}
  node[above right,yshift=8pt] {$\alpha$}
  node[below right,xshift=-2pt,yshift=-10pt] {$\alpha'$};
\fill (FP) circle (2pt)
  node[above] {$F'$}
  node[below left,xshift=-8pt] {$\alpha'$};
\draw[rotate=-50] (CP) rectangle +(12pt,12pt);
\draw[very thick,dotted] (CP) -- (FP);

% Draw fold and fold arrows
% Tangent is y = 2.26x - 10.9
% Crosses x axis at (4.83,0)
\coordinate (J) at (4.83,0);
\fill (J) circle (2pt)
    node[below] {$J$}
    node[above left,xshift=-8pt] {$\alpha$};
\draw [very thick,dashed,name path=jd] (J) -- node[very near end,left] {$l$} (D);
\draw[thick,dotted,bend right=40,->] (C) to ($(CP)+(4pt,0)$);
\draw[thick,dotted,bend right=40,->] (F) to ($(FP)+(4pt,4pt)$);

% Draw hypotenuses of right triangles
\draw[very thick,dotted] (CP) -- (J);
\path (J)  -- (C);

% Labels on BC and hypotenuses
\path (CP) -- node[right] {$(a+1)-b$} (J);
\path (J)  -- node[below] {$(a+1)-b$} (C);
\path (B)  -- node[below] {$b$} (J);
\path (C)  -- node[right] {$\disfrac{a+1}{3}$} (F);
\path (CP) -- node[right,xshift=10pt] {$\disfrac{a+1}{3}$} (FP);

% Labels on AB
\draw[<->] ($(A)+(-1,0)$)    --
  node[fill=white] {$a$} ($(CP)+(-1,0)$);
\draw[<->] ($(CP)+(-1,0)$)   --
  node[fill=white] {$1$} ($(B)+(-1,0)$);
\draw[<->] ($(CP)+(-2.5,0)$) --
  node[fill=white] {$a-\disfrac{a+1}{3}$} ($(G)+(-2.5,0)$);
\draw[<->] ($(A)+(-2.5,0)$) --
  node[fill=white] {$\disfrac{a+1}{3}$} ($(G)+(-2.5,0)$);
\end{tikzpicture}
\end{center}

Label the side of the square by $a+1$. We will show that $a=\sqrt[3]{2}$.

Using Axiom~6 place $C$ at $C'$ on $\overline{AB}$ and $F$ at $F'$ on $\overline{GH}$.  Denote by $J$ the point intersection of the fold with $\overline{BC}$ and denote by $b$ the length of $\overline{BJ}$. The length of $\overline{JC}$ is $(a+1)-b$.

When the fold is performed, the line segment $\overline{JC}$ is reflected onto the line segment $\overline{C'J}$ of the same length, and $\overline{CF}$ is folded onto the line segment $\overline{C'F'}$ of the same length. A simple computation shows that the length of $\overline{GC'}$ is $a-\disfrac{a+1}{3}=\disfrac{2a-1}{3}$. Finally, since $\angle FCJ$ is a right angle, so is $\angle F'C'J$.

$\triangle C'BJ$ is a right triangle so by Pythagoras's theorem:
\begin{form}{1.3}
1^2 + b^2 &=& ((a+1)-b)^2\\
&=& a^2+2a+1 - 2(a+1)b + b^2\\
0&=&a^2+2a - 2(a+1)b\\
b&=&\disfrac{a^2+2a}{2(a+1)}\,.
\end{form}

$\angle GC'F' + \angle F'C'J + \angle JC'B = 180^\circ$ since they form the straight line $\overline{GB}$. Denote $\angle GC'F'$ by $\alpha$.
\[
\angle JC'B=180^\circ - \angle F'C'J - \angle GC'F'= 180^\circ - 90^\circ - \angle GC'F' = 90^\circ-\angle GC'F = 90^\circ -\alpha\,,
\]
which we denote by $\alpha'$. The triangles $\triangle C'BJ$, $\triangle F'GC'$ are right triangles, so $\angle C'JB=\alpha$ and $\angle C'F'G=\alpha'$. Therefore, the triangles are similar and we have:
\[
\disfrac{b}{(a+1)-b}=\disfrac{\disfrac{2a-1}{3}}{\disfrac{a+1}{3}}\,.
\]
Substituting for $b$:
\begin{form}{2}
\disfrac{\disfrac{a^2+2a}{2(a+1)}}{(a+1)-\disfrac{a^2+2a}{2(a+1)}}&=&\disfrac{2a-1}{a+1}\\
%\disfrac{a^2+2a}{(a+1)\cdot 2(a+1)-(a^2+2a)}&=&\disfrac{2a-1}{a+1}\\
\disfrac{a^2+2a}{a^2+2a +2}&=&\disfrac{2a-1}{a+1}\\
%a^3+3a^2+2a&=&(2a-1)(a^2+2a+2)\,.
%&=&2a^3+3a^2+2a-2\,.
\end{form}
Simplifying results in $a^3=2$, $a=\sqrt[3]{2}$.



\newpage

\section{Beloch's doubling of a cube}\label{s.cube2}

In 1936 Margharita P. Beloch was the first to formalize Axiom~6 (often called the \emph{Beloch fold}) and to show that it could be used to solve cubic equations. Here we give her construction for doubling the cube. The construction is treated in more detail in Chapters~\ref{c.lill}, \ref{c.beloch}.

\subsection{The construction}

Place point $A$ at $(-1,0)$ and point $B$ at $(0,-2)$. Let $p$ be the line with equation $x=1$ and let $q$ be the line with equation $y=2$.

Using Axiom~6 construct a fold $l$ that places $A$ at $A'$ on $p$ and $B$ at $B'$ on $q$. Denote the intersection of the fold and the $y$-axis by $X$ and the intersection of the fold and $x$-axis by $Y$.

\begin{center}
\begin{tikzpicture}[scale=1]
% Draw and label square
\coordinate (O) at (0,0);
\coordinate (A) at (-2,0);
\coordinate (B) at (0,-4);
\fill (O) circle (2pt)
  node[below left,xshift=-7pt] {$O$}
  node[below left,yshift=-12pt] {$(0,0)$};
\fill (A) circle (2pt)
  node[above left,xshift=-7pt] {$A$}
  node[below left,xshift=2pt,yshift=0pt] {$(-1,0)$};
\fill (B) circle (2pt)
  node[left,xshift=-12pt] {$B$}
  node[left,yshift=-12pt] {$(0,-2)$};

%\draw[thick] (0,-4.5) -- node[near start, left,yshift=-20pt] {$p$} +(0,10);
%\draw[thick] (-5,0) -- node[very near start, below,xshift=-16pt] {$q$} +(12,0);
\draw[thick] (0,-4.5) --  node[very near end,above left,yshift=12pt] {$y$-\textsf{axis}} +(0,10);
\draw[thick] (-5,0)   -- node[very near start,above left] {$x$-\textsf{axis}} +(12,0);
\draw[very thick] (2,-4.5) -- node[very near start, right,yshift=-10pt] {$p\!:x=1$} +(0,10);
\draw[very thick] (-5,4) -- node[very near start, above,xshift=-16pt] {$q\!: y=2$} +(12,0);

\coordinate (AP) at (2,5);
\fill (AP) circle (2pt) node[above right] {$A'$};
\coordinate (BP) at (6.34,4);
\fill (BP) circle (2pt) node[above right] {$B'$};

% Tangent y = -0.8x + 1.26

\coordinate (X) at (0,2.52);
\coordinate (Y) at (3.15,0);
\fill (X) circle (2pt) node[right,xshift=4pt,yshift=2pt] {$X$};
\fill (Y) circle (2pt) node[above right,xshift=10pt] {$Y$};
\draw [very thick,dashed] ($(X)!-1.1!(Y)$) -- node[very near end,right,xshift=8pt] {$l$} ($(X)!2!(Y)$);

\draw [very thick,dotted] (A) -- (AP);
\draw [very thick,dotted] (B) -- (BP);

\draw[thick,dotted,bend left=40,->] (A) to ($(AP)+(-4pt,0)$);
\draw[thick,dotted,bend left=40,->] (B) to ($(BP)+(-6pt,-3pt)$);

%\node[above left] at (0,4) {$(0,2)$};
%\node[below left] at (2,0) {$(1,0)$};

\end{tikzpicture}
\end{center}

\newpage

\subsection{Proof}

Let us extract a simplified diagram:

\begin{center}
\begin{tikzpicture}[scale=1]
\coordinate (O) at (0,0);
\coordinate (A) at (-2,0);
\coordinate (B) at (0,-4);
\fill (O) circle (2pt)
  node[below left,xshift=-7pt] {$O$};
\fill (A) circle (2pt)
  node[above left,xshift=-7pt] {$A$}
  node[above right,xshift=10pt] {$\alpha$};
\fill (B) circle (2pt)
  node[left,xshift=-12pt] {$B$}
  node[above right,yshift=12pt] {$\alpha'$};

\draw[thick] (0,-4.5) -- +(0,10);
\draw[thick] (-3,0)   -- +(8,0);

\coordinate (AP) at (2,5);
\fill (AP) circle (2pt) node[above right] {$A'$};
\coordinate (BP) at (6.34,4);
\fill (BP) circle (2pt) node[above right] {$B'$};

% Tangent y = -0.8x + 1.26

\coordinate (X) at (0,2.52);
\coordinate (Y) at (3.15,0);
\fill (X) circle (2pt)
  node[right,xshift=4pt,yshift=2pt] {$X$}
  node[below right,yshift=-14pt] {$\alpha$}
  node[below left,xshift=2pt,yshift=-12pt] {$\alpha'$};
\fill (Y) circle (2pt)
  node[above right,xshift=10pt] {$Y$}
  node[below left,xshift=-10pt] {$\alpha$}
  node[above left,xshift=-15pt] {$\alpha'$};
\draw [very thick,dashed] ($(X)!-.4!(Y)$) -- ($(X)!1.2!(Y)$);

\draw [very thick,dotted] (A) -- (AP);
\draw [very thick,dotted] (B) -- (BP);

\draw (0,0) rectangle +(10pt,10pt);
\draw[rotate=-130] (X) rectangle +(10pt,10pt);
\draw[rotate=-130] (Y) rectangle +(10pt,10pt);

\end{tikzpicture}
\end{center}

The fold is the perpendicular bisector of $\overline{AA'}$ and $\overline{BB'}$. Therefore, $\angle AXY$ and $\angle XYB$ are right angles and $\overline{AA'}$ is parallel to $\overline{BB'}$. By alternate interior angles $\angle XAO =\angle BYO=\alpha$. If an acute angle in a right triangle is $\alpha$, the other acute angle must be $90^\circ - \alpha$, which we denote $\alpha'$. The labeling of the angles in all the triangles in the diagram follows immediately.

We have three similar triangles $\triangle AOX\sim \triangle XOY \sim \triangle YOB$. $\overline{OA}=1$, $\overline{OB}=2$ are given, so:
\begin{form}{2}
\disfrac{\overline{OX}}{\overline{OA}}=\disfrac{\overline{OY}}{\overline{OX}}=\disfrac{\overline{OB}}{\overline{OY}}\\
\disfrac{\overline{OX}}{1}=\disfrac{\overline{OY}}{\overline{OX}}=\disfrac{2}{\overline{OY}}\\
\overline{OX}^2=\overline{OY}=\disfrac{2}{\overline{OX}}\;,
\end{form}
resulting in $\overline{OX}^3=2$ and $\overline{OX}=\sqrt[3]{2}$.
