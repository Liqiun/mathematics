% !TeX root = origami-math.tex

\chapter*{References}\label{c.ref}

The following references were used in the preparation of this document.

The axioms are given in the Wikipedia article \cite{hh}, together with parametric equations for the first five axioms. Lee \cite[Chapter~4]{hwa} is a good overview of the mathematics of origami, while Martin \cite[Chapter~10]{martin} is a formal development. Lang \cite{lang} shows how rational numbers, some irrational numbers and approximations to others can be constructing in origami. Trisecting an angle and doubling a cube are described by \cite{newton} and Ben-Lulu \cite{oriah} provides a different proof of the trisection. The construction for doubling the cube is from Newton \cite{newton} and Lee \cite{hwa}. Hull \cite{hull-beloch} presents Beloch's work on solving cubic equations with origami.

\begingroup
\renewcommand\bibname{}
\let\clearpage\relax
\vspace{-4ex}
\bibliographystyle{plain}
\bibliography{origami-math}
\endgroup