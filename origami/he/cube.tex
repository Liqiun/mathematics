% !TeX root = origami-math-he.tex

%%%%%%%%%%%%%%%%%%%%%%%%%%%%%%%%%%%%%%%%%%%%%%%%%%%%%%%%%%%%%%%%
\chapter{הכפלת קוביה}\label{c.cube}

\section{%
הבנייה של 
\L{Messer}
להכפלת קוביה}%
\label{s.cube1}

לקוביה בנפח 
$V$
צלעות באורך
$\sqrt[3]{V}$.
נפח קוביה שנפחה פי שניים הוא
 $2\cdot V$,
 כך שיש לבנות קטע קו באורך
$\sqrt[3]{2\cdot V}=\sqrt[3]{2}\cdot \sqrt[3]{V}$.
אם נוכל לבנות קטע קו באורך
$\sqrt[3]{2}$,
נוכל להכפיל באורך הנתון
$\sqrt[3]{V}$
כדי להכפיל את נפח הקוביה.

\subsection{חלוקת קטע קו לשלוש}

\L{Lang \cite{lang}}
מביא בניות יעילות עבור שברים רציונליים של אורכו של צלע של )דף נייר שהוא( ריבוע. כאן עלינו לחלק צלע של ריבוע לשלושה חלקים.

תחילה, קפל את הריבוע לחצי כדי למצוא את הנוקדה
$J=(1,1/2)$.
אחר כך, בנה את קטעי הקו
$\overline{AC}$
ו-%
$\overline{BJ}$.
\begin{center}
\selectlanguage{english}
\begin{tikzpicture}[scale=.8]
% Draw square
\coordinate (A) at (0,12);
\coordinate (B) at (0,0);
\coordinate (C) at (12,0);
\coordinate (D) at (12,12);

\fill (A) circle (2pt) node[left]  {$A=(0,1)$};
\fill (B) circle (2pt) node[left]  {$B=(0,0)$};
\fill (C) circle (2pt) node[right] {$C=(1,0)$};
\fill (D) circle (2pt) node[right] {$D=(1,1)$};

\draw [thick] (A)  -- (B) -- (C) -- (D) -- cycle;

% Divide a side in half

\coordinate (M)  at (0,6);
\coordinate (N) at (12,6);
\fill (M) circle (2pt) node[left] {$I=(0,1/2)$};
\fill (N) circle (2pt) node[right] {$J=(1,1/2)$};
\draw [thick,dashed] (M) -- (N);


\draw [thick,dotted,name path=ac] (A) -- 
   node[near start,above,xshift=24pt] {$y=1-x$} (C);
\draw [thick,dotted,name path=be2] (B) -- 
   node[near start,above,xshift=-12pt,yshift=-2pt] {$y=\disfrac{1}{2}x$} (N);

\path [name intersections = {of = ac and be2, by = {I}}];
\fill (I) circle (2pt) 
   node[below,xshift=-6pt,yshift=-8pt] {$K=$}
   node[below,xshift=-6pt,yshift=-20pt] {$(2/3,1/3)$};

\coordinate (E)  at (0,4);
\coordinate (F) at (12,4);
\fill (E) circle (2pt) node[left] {$E=(0,1/3)$};
\fill (F) circle (2pt) node[right] {$F=(1,1/3)$};
\draw [thick,dashed] (E) -- (F);

\coordinate (G)  at (0,8);
\coordinate (H) at (12,8);
\fill (G) circle (2pt) node[left] {$G=(0,2/3)$};
\fill (H) circle (2pt) node[right] {$H=(1,2/3)$};
\draw [very thick,dotted] (G) -- (H);
\end{tikzpicture}
\end{center}
אפשר לחשב את הקואורדינטות של נקודה החיתוך 
$K$
על ידי פתרון של שתי המשוואות הללו:
\begin{form}{1.8}
y&=&1-x\\
y&=&\disfrac{1}{2}x\,.
\end{form}
הפתרון הוא:
$x=2/3, y=1/3$.

בנה את הקו
$\overline{EF}$
ניצב ל-%
$\overline{AB}$
כך שהוא עובר דרך 
$K$,
ובנה את 
$\overline{GH}$,
השיקוף של
$\overline{BC}$
סביב
$\overline{EF}$.
הצלע של הריבוע מחולק לשלושה חלקים.

\subsection{בניית $\sqrt[3]{2}$}

\begin{center}
\selectlanguage{english}
\begin{tikzpicture}[scale=.9]
% Draw and label square
\coordinate (A) at (0,12);
\coordinate (B) at (0,0);
\coordinate (C) at (12,0);
\coordinate (D) at (12,12);
\fill (A) circle (2pt) node[left]  {$A$};
\fill (B) circle (2pt) node[left]  {$B$};
\fill (C) circle (2pt) node[right] {$C$};
\fill (D) circle (2pt) node[right] {$D$};
\draw (B) rectangle +(12pt,12pt);
\draw[rotate=90] (C) rectangle +(12pt,12pt);
\draw [thick] (A)  -- (B) -- (C) -- (D) -- cycle;

% Draw line one-third from botton
\coordinate (E)  at (0,4);
\coordinate (F) at (12,4);
\fill (E) circle (2pt) node[left] {$E$};
\fill (F) circle (2pt) node[right] {$F$};
\draw [very thick,dotted,name path=ef] (E) -- (F);

% Draw line two-thirds from bottom
\coordinate (G)  at (0,8);
\coordinate (H) at (12,8);
\fill (G) circle (2pt) node[left] {$G$};
\fill (H) circle (2pt) node[right] {$H$};
\draw[rotate=-90] (G) rectangle +(12pt,12pt);
\draw [very thick,dotted] (G) -- (H);

% Draw reflections of C and F
\coordinate (CP) at (0,5.31);
\coordinate (FP) at (2.96,8);
\fill (CP) circle (2pt)
  node[left] {$C'$}
  node[above right,yshift=8pt] {$\alpha$}
  node[below right,xshift=-2pt,yshift=-12pt] {$\alpha'$};
\fill (FP) circle (2pt)
  node[above] {$F'$}
  node[below left,xshift=-8pt] {$\alpha'$};
\draw[rotate=-50] (CP) rectangle +(12pt,12pt);
\draw[very thick,dotted] (CP) -- (FP);

% Draw fold and fold arrows
% Tangent is y = 2.26x - 10.9
% Crosses x axis at (4.83,0)
\coordinate (J) at (4.83,0);
\fill (J) circle (2pt)
    node[below] {$J$}
    node[above left,xshift=-8pt] {$\alpha$};
\draw [very thick,dashed,name path=jd] (J) -- node[very near end,left] {$l$} (10,12);
\draw[thick,dotted,bend right=40,->] (C) to ($(CP)+(4pt,0)$);
\draw[thick,dotted,bend right=40,->] (F) to ($(FP)+(4pt,4pt)$);

% Draw hypotenuses of right triangles
\draw[very thick,dotted] (CP) -- (J);
\path (J)  -- (C);

% Labels on BC and hypotenuses
\path (CP) -- node[right] {$(a+1)-b$} (J);
\path (J)  -- node[below] {$(a+1)-b$} (C);
\path (B)  -- node[below] {$b$} (J);
\path (C)  -- node[right] {$\disfrac{a+1}{3}$} (F);
\path (CP) -- node[right,xshift=10pt] {$\disfrac{a+1}{3}$} (FP);

% Labels on AB
\draw[<->] ($(A)+(-1,0)$)    --
  node[fill=white] {$a$} ($(CP)+(-1,0)$);
\draw[<->] ($(CP)+(-1,0)$)   --
  node[fill=white] {$1$} ($(B)+(-1,0)$);
\draw[<->] ($(CP)+(-2.5,0)$) --
  node[fill=white] {$a-\disfrac{a+1}{3}$} ($(G)+(-2.5,0)$);
\draw[<->] ($(A)+(-2.5,0)$) --
  node[fill=white] {$\disfrac{a+1}{3}$} ($(G)+(-2.5,0)$);
\end{tikzpicture}
\end{center}

נמסן צלע של הריבוע 
$a+1$.
הבנייה תראה ש-%
$a=\sqrt[3]{2}$.

נשמתש באקסיומה $6$ כדי להניח את 
$C$
ב-%
$C'$
על
$\overline{AB}$,
ולהניח את
$F$
ב-%
$F'$
על
$\overline{GH}$.
סמן את נקודת החיתוך של הקיפול עם
$\overline{BC}$ 
ב-%
$J$,
וסמן את אורכו של
$\overline{BJ}$
ב-%
$b$.
האורך של קטע הקו 
$\overline{JC}$
הוא
$(a+1)-b$.

לאחר ביצוע הקיפול, קטע הקו
$\overline{JC}$
הוא שיקוף של קטע הקו
$\overline{JC'}$
מאותו אורך, וקטע הקו

$\overline{CF}$
הוא שיקוף של קטע הקו
$\overline{C'F'}$
באותו אורך. חישוב פשוט מראה שאורכו של
$\overline{GC'}$
הוא:
\begin{equation}
\selectlanguage{english}
a-\disfrac{a+1}{3}=\disfrac{2a-1}{3}\,.\label{eq.one-third}
\end{equation}
לבסוף,
$\angle FCJ$
היא זווית ישרה, לכן גם
$\angle F'C'J$ 
היא זווית ישרה.

$\triangle C'BJ$
הוא משולש ישר-זווית ולפי משפט פתגורוס:
\begin{form}{1.3}
1^2 + b^2 &=& ((a+1)-b)^2\\
%&=& a^2+2a+1 - 2(a+1)b + b^2\\
a^2+2a - 2(a+1)b&=&0\\
b&=&\disfrac{a^2+2a}{2(a+1)}\,.
\end{form}

$\angle GC'F' + \angle F'C'J + \angle JC'B = 180^\circ$
כי הם מרכיבים קו ישר
$\overline{GB}$.
נסמן
$\alpha=\angle GC'F'$.
אז:
\[
\angle JC'B=180^\circ - \angle F'C'J - \angle GC'F'= 180^\circ - 90^\circ - \angle GC'F' = 90^\circ-\angle GC'F = 90^\circ -\alpha\,.
\]
נסמן
$\alpha'=90^\circ-\alpha$.
המשולשים
$\triangle C'BJ$, $\triangle F'GC'$
הם משולשים ישר-זווית, ולכן 
$\angle C'JB=\alpha$
ו-%
$\angle C'F'G=\alpha'$.
מכאן שהמשולשים דומים וממשוואה%
~\L{\ref{eq.one-third}}
מתקבלת המשוואה:
\[
\disfrac{b}{(a+1)-b}=\disfrac{\disfrac{2a-1}{3}}{\disfrac{a+1}{3}}\,.
\]
נציב עבור
$b$:
\begin{form}{2}
\disfrac{\disfrac{a^2+2a}{2(a+1)}}{(a+1)-\disfrac{a^2+2a}{2(a+1)}}&=&\disfrac{2a-1}{a+1}\\
%\disfrac{a^2+2a}{(a+1)\cdot 2(a+1)-(a^2+2a)}&=&\disfrac{2a-1}{a+1}\\
\disfrac{a^2+2a}{a^2+2a +2}&=&\disfrac{2a-1}{a+1}\,.
%a^3+3a^2+2a&=&(2a-1)(a^2+2a+2)\,.
%&=&2a^3+3a^2+2a-2\,.
\end{form}
נפשט ונקבל
$a^3=2$
ו-%
$a=\sqrt[3]{2}$.



\newpage

\section{הבנייה של
\L{Beloch}
להכפלת קוביה%
}\label{s.cube2}

ב-%
$1936$
\L{Margharita P. Beloch}
נתנה הגדרה פורמלית לאקסיומה~$6$ )הנקרא לעתים הקיפול של
\L{Beloch}(.
היא הראתה שניתן להשמתמש באקסיומה כדי לפתור משוואות ממעלה שלוש. כאן אנחנו נביא את השיטה שלה להכפלת קוביה. נדון בפתרון של משוואות ממעלה שלוש בפרקים%
~\L{\ref{c.lill}, \ref{c.beloch}}.

\subsection{הבנייה}

נסמן את הנקודה
$(-1,0)$
ב-%
$A$
ואת הנקודה
$(0,-2)$
ב-%
$B$.
נסמן ב-%
$p$ 
את הקו 
$x=1$
וב-%
$q$
את הקו
$y=2$.
לפי אקסיומה $6$ ניתן לבנות קיפול 
$l$
המניח את
$A$
ב-%
$A'$
על 
$p$,
והמניח את
$B$
ב-%
$B'$
על
$q$.
נסמן ב-%
$Y$
את נקודת החיתוך של הקיפול עם ציר ה-%
$y$,
ונסמן ב-%
$X$
את נקודת החיתוך של הקיפול עם ציר ה-%
$x$.

\begin{center}
\selectlanguage{english}
\begin{tikzpicture}[scale=1]
% Draw and label square
\coordinate (O) at (0,0);
\coordinate (A) at (-2,0);
\coordinate (B) at (0,-4);
\fill (O) circle (2pt)
  node[below left,xshift=-7pt] {$O$}
  node[below left,yshift=-12pt] {$(0,0)$};
\fill (A) circle (2pt)
  node[above left,xshift=-7pt] {$A$}
  node[below left,xshift=2pt,yshift=0pt] {$(-1,0)$};
\fill (B) circle (2pt)
  node[left,xshift=-12pt] {$B$}
  node[left,yshift=-12pt] {$(0,-2)$};

\draw[thick] (0,-4.5) --  node[very near end,above left,yshift=12pt] {$y$-\textsf{axis}} +(0,10);
\draw[thick] (-5,0)   -- node[very near start,above left] {$x$-\textsf{axis}} +(12,0);
\draw[very thick] (2,-4.5) -- node[very near start, right,yshift=-10pt] {$p\!:x=1$} +(0,10);
\draw[very thick] (-5,4) -- node[very near start, above,xshift=-16pt] {$q\!: y=2$} +(12,0);

\coordinate (AP) at (2,5);
\fill (AP) circle (2pt) node[above right] {$A'$};
\coordinate (BP) at (6.34,4);
\fill (BP) circle (2pt) node[above right] {$B'$};

% Tangent y = -0.8x + 1.26

% Exchanged X and Y 
\coordinate (X) at (0,2.52);
\coordinate (Y) at (3.15,0);
\fill (X) circle (2pt) node[right,xshift=4pt,yshift=2pt] {$Y$};
\fill (Y) circle (2pt) node[above right,xshift=10pt] {$X$};
\draw [very thick,dashed] ($(X)!-1.1!(Y)$) -- node[very near end,right,xshift=8pt] {$l$} ($(X)!2!(Y)$);

\draw [very thick,dotted] (A) -- (AP);
\draw [very thick,dotted] (B) -- (BP);

\draw[thick,dotted,bend left=40,->] (A) to ($(AP)+(-4pt,0)$);
\draw[thick,dotted,bend left=40,->] (B) to ($(BP)+(-6pt,-3pt)$);

\end{tikzpicture}
\end{center}

\newpage

\subsection{הוכחה}

נחלץ איור פשוט יותר:
\begin{center}
\selectlanguage{english}
\begin{tikzpicture}[scale=1]
\coordinate (O) at (0,0);
\coordinate (A) at (-2,0);
\coordinate (B) at (0,-4);
\fill (O) circle (2pt)
  node[below left,xshift=-7pt] {$O$};
\fill (A) circle (2pt)
  node[above left,xshift=-7pt] {$A$}
  node[above right,xshift=10pt] {$\alpha$};
\fill (B) circle (2pt)
  node[left,xshift=-12pt] {$B$}
  node[above right,yshift=12pt] {$\alpha'$};

\draw[thick] (0,-4.5) -- +(0,10);
\draw[thick] (-3,0)   -- +(8,0);

\coordinate (AP) at (2,5);
\fill (AP) circle (2pt) node[above right] {$A'$};
\coordinate (BP) at (6.34,4);
\fill (BP) circle (2pt) node[above right] {$B'$};

% Tangent y = -0.8x + 1.26

% Exchanged X and Y 
\coordinate (X) at (0,2.52);
\coordinate (Y) at (3.15,0);
\fill (X) circle (2pt)
  node[right,xshift=4pt,yshift=2pt] {$Y$}
  node[below right,yshift=-14pt] {$\alpha$}
  node[below left,xshift=2pt,yshift=-12pt] {$\alpha'$};
\fill (Y) circle (2pt)
  node[above right,xshift=10pt] {$X$}
  node[below left,xshift=-10pt] {$\alpha$}
  node[above left,xshift=-15pt] {$\alpha'$};
\draw [very thick,dashed] ($(X)!-.4!(Y)$) -- ($(X)!1.2!(Y)$);

\draw [very thick,dotted] (A) -- (AP);
\draw [very thick,dotted] (B) -- (BP);

\draw (0,0) rectangle +(10pt,10pt);
\draw[rotate=-130] (X) rectangle +(10pt,10pt);
\draw[rotate=-130] (Y) rectangle +(10pt,10pt);

\end{tikzpicture}
\end{center}
הקיפול הוא האנך האמצעי של
$\overline{AA'}$
ן-%
$\overline{BB'}$.
לכן
$\angle AYX$
ו-%
$\angle YXB$
הן זוויות ישר-זווית ו-%
$\overline{AA'}$ 
מקביל ל-%
$\overline{BB'}$.
לפי זוויות מתחלפות
$\angle XAO =\angle BYO=\alpha$.
אם זווית חדה אחת של משולש ישר-זווית היא
$\alpha$,
הזווית החדה השנייה היא
$90^\circ - \alpha$
שנסמן
$\alpha'$.
מכאן מתקבלים סימוני הזוויות האחרות באיור.

יש לנו שלושה משולשים דומים:
$\triangle AOY\sim \triangle YOX \sim \triangle XOB$.
קטעי קו
$\overline{OA}=1$, $\overline{OB}=2$
נתונים ולכן:
\begin{form}{2}
\disfrac{\overline{OY}}{\overline{OA}}=\disfrac{\overline{OX}}{\overline{OY}}=\disfrac{\overline{OB}}{\overline{OX}}\\
\disfrac{\overline{OY}}{1}=\disfrac{\overline{OX}}{\overline{OY}}=\disfrac{2}{\overline{OX}}\\
\overline{OY}^2=\overline{OX}=\disfrac{2}{\overline{OY}}\;,
\end{form}

מכאן ש-%
$\overline{OY}^3=2$
ו-%
$\overline{OY}=\sqrt[3]{2}$.
