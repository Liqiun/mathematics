% !TeX root = origami-math-he.tex

\chapter*{מקורות}\label{c.ref}

להלן רשימת המקורות ששימשו להכנהת המסמך.

האקסיומות נמצאות במאמר בויקיפדיה%
\L{\cite{hh}},
ביחד עם משוואות פרמטריות עבור חמשת האקסיומות הראשונות.
\L{Lee}
\L{\cite[4~\R{פרק}]{hwa}}
מביא סקירה טובה של המתמיטקה של אוריגמי. 
\L{Martin} \L{\cite[10~\R{פרק}]{martin}}
מפתח את המתמטיקה של אוריגמי בצורה פורמלית.
\L{Lang} \L{\cite{lang}}
מראה כיצד ניתן לבנות באמצעות אוריגמי מספרים רציונליים, מקצת מספרים אי-רציונליים וקירובים לאחרים.
חלוקת זווית לשולשה חלקים והכפלת הקוביה ניתנת על ידי
\L{Newton}
\L{\cite{newton}}
ובן לולו מביא הוכחה שונה של חולקת הזווית. הבנייה להכפלת קוביה גם היא מופיעה אצל 
\L{Newton}
\L{\cite{newton}}
ו-%
\L{\cite{hwa}}. 
\L{Hull} \L{\cite{hull-beloch}}
מביא את השיטה של
\L{Lill}
למציאת שורשים של פולינינומים ואת המימוש של 
\L{Beloch}. 
\L{Bradford} \cite{bradford} 
מביא איורים רבים של השיטה של 
\L{Lill}.

\selectlanguage{english}
\begingroup
\renewcommand\bibname{}
\let\clearpage\relax
\vspace{-4ex}
\bibliographystyle{plain}
\bibliography{origami-math-he}
\endgroup