% !TeX root = origami-math-he.tex

\thispagestyle{empty}

\selectlanguage{hebrew}
\begin{center}
\textbf{\Huge\ המתמטיקה של אוריגמי}

\bigskip
\bigskip

\textbf{\LARGE מוטי בן-ארי}

\bigskip
\bigskip
\selectlanguage{english}
\url{http://www.weizmann.ac.il/sci-tea/benari/}

\bigskip
\bigskip

\selectlanguage{hebrew}
{\large
גרסה
$4$
}
\end{center}

\vfill

\selectlanguage{english}
\begin{small}
\begin{center}
\copyright{}\ 2020 Moti Ben-Ari
\end{center}

This work is licensed under the Creative Commons Attribution-ShareAlike 3.0 Unported License. To view a copy of this license, visit \url{http://creativecommons.org/licenses/by-sa/3.0/} or send a letter to Creative Commons, 444 Castro Street, Suite 900, Mountain View, California, 94041, USA.
\end{small}
\selectlanguage{hebrew}
\tableofcontents


%%%%%%%%%%%%%%%%%%%%%%%%%%%%%%%%%%%%%%%%%%%%%%%%%%%%%%%%%%%%%%%%

\chapter{הקדמה}\label{c.introduction}

מסמך זה מפתח את המתמטיקה של אוריגמי תוך שימוש במתמטיקה של בית-ספר תיכון. המשוואות של קווים ניתנים בצורה של שיפוע ונקודת חיתוך
$y=mx+b$.

פרק%
~\L{\ref{c.axioms}}
מפתח את משוואות של שבעת האקסיומות ביחד עם דוגמאות נומריות. באיורים, קווים נתונים מוצגים בקווים רגילים, קיפולים בקווים מקווקווים, קווי עזר בקווים מנוקדים, וחצים מנוקדים מראים את כיוון הקיפול של הנייר.

פעולות הקיפול יכולות לבנות כל אורך שניתן לבנות עם סרגל ומחוגה. נתונים 
$a,b$:
ניתן לבנות

$a+b, a-b, a\times b, a/b, \sqrt{a}$
\L{\cite[4~\R{פרק}]{hwa}}.

פעולת הקיפול חזקה יותר כי ניתן לבנות שורשים ממעולה שלוש. פרק%
~\L{\ref{c.trisection}}
מציג שתי דרכים לחלק זווית לשלושה ופרק%
~\L{\ref{c.cube}}
מציג שתי דרכים להכפלת קוביה.

פרק%
~\L{\ref{c.lill}}
מסביר את השיטה הגיאומטרית של 
\L{Eduard Lill}
למציאת שורשים ממשיים של פולינום. נציג את השיטה עבור פולינומים ממעלה שלוש. פרק%
~\L{\ref{c.beloch}}
מביא את המימוש של
\L{Margharita P. Beloch}
לשיטה של 
\L{Lill}
באמצעות קיפול.

נספח%
~\L{\ref{a.geo}}
מכיל קישורים לפרוייקטים בגיאוגרה המדגימים את האקסיומות. נספח%
~\L{\ref{a.tangent}}
מפתח שוויונות טריגונומטריות. נספח%
~\L{\ref{a.parabola}}
מסביר את ההגדרה הגיאומטרית של פברבולות.

\subsection*{הגדרות}

לפי כל אחד מהאקסיומות, קיים
\textbf{קיפול}
המניח נקודות וקווים על נקודות וקווים, כך שתנאים מסויימים מתקיימים. המונח קיפול בא מהפעולה באוריגמי של קיפול דף נייר, אבל כאן נשתמש בו עבור הקו הגיאומטרי שנוצר על ידי קיפול דף נייר.

ניתן למצוא הגדרות פורמליות ב-%
\L{\cite[10~\R{פרק}]{martin}}.
לתשומת לב הקורא, 
\textbf{לפי ההגדרה},
כתוצאה מקיפול נוצרים 
\textbf{שיקופים}.
נתונה נקודה 
$p$,
השיקוף שלה סביב הקיפול 
$l$
הוא נקודה
$p'$,
כך ש-%
$l$
הוא האנך האמצעי של קטע הקו
$\overline{pp'}$:

\begin{center}
\selectlanguage{english}
\begin{tikzpicture}
\coordinate (P1) at (2,2);
\coordinate (P1P) at (6,4);
\coordinate (mid) at (4,3);
\draw[rotate=30] (mid) rectangle +(8pt,8pt);
\coordinate (m1) at ($(P1)!.5!(mid)$);
\coordinate (m2) at ($(mid)!.5!(P1P)$);
\draw[thick] (m1) -- +(120:4pt);
\draw[thick] (m1) -- +(-60:4pt);
\draw[thick] (m2) -- +(120:4pt);
\draw[thick] (m2) -- +(-60:4pt);
\draw[thick] (P1) -- (P1P);
\draw[very thick,dashed] (4.7,1.6) -- node[very near end,right,yshift=4pt] {$l$} (3.5,4);
\fill (P1) circle(2pt) node[above left] {$p$};
\fill (P1P) circle(2pt) node[above left] {$p'$};
\fill (mid) circle(2pt);% node[below,xshift=-2pt,yshift=-8pt] {$p_i$};
\draw[very thick,dotted,->,bend right=50] (2,1.8) to (6,3.8);
\end{tikzpicture}
\end{center}
