% !TeX root = origami-math-he.tex

%%%%%%%%%%%%%%%%%%%%%%%%%%%%%%%%%%%%%%%%%%%%%%%%%%%%%%%%%%%%%%%%
\chapter{חלוקת זווית לשלושה חלקים}\label{c.trisection}

\section{הבנייה של 
\L{Abe}
לחלוקת זווית לשלושה חלקים%
}\label{s.tri1}

\subsection{הבנייה}

\begin{center}
\selectlanguage{english}
\begin{tikzpicture}[scale=1]

% Place points P, Q, R
\coordinate (P) at (60:10cm); %(5,8.67);
\coordinate (Q) at (0,0);
\coordinate (R) at (10,0);
\fill (P) circle (2pt) node[below right] {$P$};
\fill (Q) circle (2pt) node[left] {$Q$};
\fill (R) circle (2pt) node[right] {$R$};

% Draw PQR
\draw [very thick] (P)  -- (Q) -- (R);

% Draw perpendicular to QR
\draw [thick] (Q) -- node[left,very near end] {$p$} +(0,11);

% Draw parallel to QR and parallel halfway
\coordinate (A) at (0,5);
\coordinate (B) at (0,2.5);
\draw [thick] (A) -- node[above,very near end] {$q$} +(10,0);
\draw [thick] (B) -- node[above,very near end] {$r$} +(10,0);
\fill (A) circle (2pt) node[left] {$A$};
\fill (B) circle (2pt) node[left] {$B$};
\path (Q) -- node[left] {$a$} (B) -- node[left] {$a$} (A);
\draw (A) rectangle +(8pt,8pt);
\draw (B) rectangle +(8pt,8pt);

% Tangent line y = -2.75x + 10.69

% Draw fold
\coordinate (D) at (0,10.69);
\coordinate (fold-x) at (3.89,0);
\coordinate (AP) at (3.65,6.33);
\coordinate (QP) at (6.87,2.5);
\coordinate (BP) at (5.26,4.42);
\fill (D) circle (2pt) node[left] {$D$};
\fill (AP) circle (2pt) node[above,yshift=6pt] {$A'$};
\fill (QP) circle (2pt) node[above,yshift=6pt] {$Q'$};
\fill (BP) circle (2pt) node[above,xshift=2pt,yshift=2pt] {$B'$};
\draw [very thick,dashed] (D) -- node[left,near start] {$l$} (fold-x);

% Draw line of reflections
\draw [very thick, dotted] (D) -- (QP);

% Draw trisecting lines
\draw [very thick,dotted] (Q) -- ($(Q)!1.3!(QP)$);
\draw [very thick,dotted] (Q) -- ($(Q)!1.3!(BP)$);

% Complete triangle
\draw [very thick,dotted] (A) -- (QP);

\end{tikzpicture}
\end{center}
נתונה זווית חדה
$\angle PQR$,
יהי הקו
$p$
ניצב ל-%
$\overline{QR}$
ב-%
$Q$.
יהי הקו
$q$
ניצב ל-%
$p$
ב-% 
$A$
שחותך את
$\overline{PQ}$,
ויהי הקו
$r$
ניצב ל-%
$p$
ב-%
$B$
במחצית הדרך בן
$Q$
ו-%
$A$.

לפי אקסיומה 
$6$
בנה קיפול
$l$
המניח את 
$A$
על
$\overline{PQ}$ 
בנקודה
$A'$,
ומניח את
$Q$
על
$r$
בנקודה
$Q'$.
נסמן ב--%
$B'$
את השיקוף של 
$B$
מסביב ל-%
$l$.

בנה את הקווים
$\overline{QB'}$
ו-%
$\overline{QQ'}$.
טיעון: הזוויות
$\angle PQB'$, $\angle B'QQ'$, $\angle Q'QR$
מחלקות לשלושה חלקים את הזווית
$\angle PQR$.

\subsection{הוכחה ראשונה}

\begin{center}
\selectlanguage{english}
\begin{tikzpicture}[scale=1]

% Place points P, Q, R
\coordinate (P) at (60:10cm);
\coordinate (Q) at (0,0);
\coordinate (R) at (10,0);
\fill (P) circle (2pt) node[below right] {$P$};
\fill (Q) circle (2pt) node[left,xshift=-4pt] {$Q$};
\fill (R) circle (2pt) node[right] {$R$};

% Draw PQR
\draw [very thick] (Q) -- (R);

% Draw perpendicular to QR
\draw [thick] (Q) -- node[left,very near end] {$p$} +(0,11);

% Draw parallel to QR and parallel halfway
\coordinate (A) at (0,5);
\coordinate (B) at (0,2.5);
\draw [thick] (A) -- node[above,very near end] {$q$} +(10,0);
\draw [thick] (B) -- node[above,very near end] {$r$} +(10,0);
\fill (A) circle (2pt) node[left,xshift=-4pt] {$A$};
\fill (B) circle (2pt) node[left,xshift=-4pt] {$B$};
\path (Q) -- node[left,xshift=-4pt] {$a$} (B) -- node[left,xshift=-4pt] {$a$} (A);
\draw (A) rectangle +(8pt,8pt);
\draw (B) rectangle +(8pt,8pt);

% Tangent line y = -2.75x + 10.69

% Draw fold
\coordinate (D) at (0,10.69);
\coordinate (fold-x) at (3.89,0);
\coordinate (AP) at (3.65,6.33);
\coordinate (QP) at (6.87,2.5);
\coordinate (BP) at (5.26,4.42);
\fill (D) circle (2pt) node[left] {$D$};
\fill (AP) circle (2pt) node[above,yshift=6pt] {$A'$};
\fill (QP) circle (2pt) node[above,xshift=2pt,yshift=6pt] {$Q'$};
\fill (BP) circle (2pt) node[above,xshift=4pt,yshift=2pt] {$B'$};
\draw [very thick,dashed] (D) -- node[left,near start] {$l$} (fold-x);
	
% Draw line of reflections
\draw [very thick, dotted] (D) -- (AP);

% Draw trisecting lines
\draw [very thick,dotted] (Q) -- ($(Q)!1.3!(BP)$);

\draw [very thick,loosely dash dot,red] (Q) -- (QP);
\draw [very thick,loosely dash dot,red] (QP) -- (AP);
\draw [very thick,loosely dash dot,red] (AP) -- (Q);
\draw [very thick,loosely dash dot dot,blue] ($(Q)+(0,-4pt)$) -- ($(QP)+(0,-4pt)$);
\draw [very thick,dash dot dot,blue] ($(QP)+(0,-4pt)$) -- ($(A)+(0,-4pt)$);
\draw [very thick,dash dot dot,blue] ($(A)+(-4pt,0)$) -- ($(Q)+(-4pt,0)$);

\draw [thick,dotted] (A) -- (AP);

\node[left,xshift=-40pt,yshift=7pt] at (QP) {$\alpha$};
\node[left,xshift=-40pt,yshift=-6pt] at (QP) {$\alpha$};
\node[right,xshift=40pt,yshift=6pt] at (Q) {$\alpha$};
\node[right,xshift=40pt,yshift=28pt] at (Q) {$\alpha$};
\node[right,xshift=30pt,yshift=42pt] at (Q) {$\alpha$};

\end{tikzpicture}
\end{center}

הנקודות
$A', B', Q'$
הן שיקופים סביב אותו קו 
$l$
של הנקודות
$A,B,Q$
הנמצאות על קו אחד
$\overline{DQ}$,
ולכן גם הן נמצאות על קטע קו אחד
$\overline{DQ'}$.
לפי הבנייה,
$\overline{AB}=\overline{BQ}$, $\overline{BQ'}$
ניצב ל-%
$AQ$
ו-%
$\overline{BQ'}$
הוא צלע משותף, ולכן
$\triangle ABQ'\cong \triangle QBQ'$ 
לפי צלע-זווית-צלע. מכאן ש:
$\angle AQ'B=\angle QQ'B=\alpha$,
כי
$\overline{Q'B}$
הוא האנך האמצעי של המשולש שווי-שוקיים
$\triangle AQ'Q$.

לפי זוויות מתחלפות,
$\angle Q'QR=\angle QQ'B=\alpha$.

לפי שיקוף,
$\triangle AQ'Q\cong \triangle A'QQ'$.\footnote{%
שני המשולשים מודגשים על ידי קווים שונים של מקפים ונקודות, וכן על ידי הצבעים אדום וכחול.%
}
\begin{quote}
הוכחה: הקיפול
$l$
הוא האנך האמצעי של 
$\overline{AA'}$
וגם של
$\overline{QQ'}$;
בנה ניצבים מ-%
$A$
ו-%
$A'$
ל-%
$\overline{QQ'}$;
אזי
$\overline{AQ}=\overline{A'Q'}$
לפי משולשים ישר זווית חופפים.
$\overline{AA'Q'Q}$
הוא טרפז שווי-שוקיים כך שהאלכסונים שווים
$\overline{AQ'}=\overline{A'Q}$.
\end{quote}
מכאן ש-%
$\overline{QB'}$,
השיקוף של
$\overline{Q'B}$,
הוא האנך האמצעי של משולש שווי-שוקיים
$\triangle A'QQ'$,
ו-%
$\angle A'QB'=\angle Q'QB'=\angle QQ'B=\alpha$.

\subsection{הוכחה שנייה}

\begin{center}
\selectlanguage{english}
\begin{tikzpicture}[scale=1]

% Place points P, Q, R
\coordinate (P) at (60:10cm); %(5,8.67);
\coordinate (Q) at (0,0);
\coordinate (R) at (10,0);
\fill (P) circle (2pt) node[below right] {$P$};
\fill (Q) circle (2pt) node[left] {$Q$};
\fill (R) circle (2pt) node[right] {$R$};

% Draw PQR
\draw [very thick] (P)  -- (Q) -- (R);

% Draw perpendicular to QR
\draw [thick] (Q) -- node[left,very near end] {$p$} +(0,11);

% Draw parallel to QR and parallel halfway
\coordinate (A) at (0,5);
\coordinate (B) at (0,2.5);
\draw [thick] (A) -- node[above,very near end] {$q$} +(10,0);
\draw [thick] (B) -- node[above,very near end] {$r$} +(10,0);
\fill (A) circle (2pt) node[left] {$A$};
\fill (B) circle (2pt) node[left] {$B$};
\path (Q) -- node[left] {$a$} (B) -- node[left] {$a$} (A);
\draw (A) rectangle +(8pt,8pt);
\draw (B) rectangle +(8pt,8pt);

% Tangent line y = -2.75x + 10.69

% Draw fold
\coordinate (D) at (0,10.69);
\coordinate (fold-x) at (3.89,0);
\coordinate (AP) at (3.65,6.33);
\coordinate (QP) at (6.87,2.5);
\coordinate (BP) at (5.26,4.42);
\fill (D) circle (2pt) node[left] {$D$};
\fill (AP) circle (2pt) node[above,yshift=6pt] {$A'$};
\fill (QP) circle (2pt) node[above,yshift=6pt] {$Q'$};
\fill (BP) circle (2pt) node[above,xshift=2pt,yshift=2pt] {$B'$};
\draw [very thick,dashed,name path=fold] (D) -- node[left,near start] {$l$} (fold-x);

% Draw line of reflections
\draw [very thick, dotted] (D) -- (QP);

% Draw trisecting lines
\draw [very thick,dotted,name path=Qr] (Q) -- ($(Q)!1.3!(QP)$);
\draw [very thick,dotted,name path=Qq] (Q) -- ($(Q)!1.3!(BP)$);

% Draw indications of right angles
\draw[rotate=-140] (BP) rectangle +(8pt,8pt);
\path [name intersections = {of = fold and Qr, by = {U}}];
\fill (U) circle (2pt) node[above left,xshift=-2pt,yshift=-2pt] {$U$};
\draw[rotate=20] (U) rectangle +(8pt,8pt);
\path [name intersections = {of = fold and Qq, by = {V}}];
\fill (V) circle (2pt) node[above left,xshift=-2pt,yshift=-2pt] {$V$};

\path (Q) -- node[below,near end] {$b$} (U);
\path (U) -- node[below] {$b$} (QP);

\node[left,xshift=-40pt,yshift=-6pt] at (QP) {$\alpha$};
\node[right,xshift=40pt,yshift=6pt] at (Q) {$\alpha$};
\node[right,xshift=40pt,yshift=28pt] at (Q) {$\alpha$};
\node[right,xshift=30pt,yshift=42pt] at (Q) {$\alpha$};
\end{tikzpicture}
\end{center}

הקו
$l$
הוא קיפול, ולכן הוא האנך האמצעי של
$\overline{QQ'}$.
סמן ב-%
$U$
את נקודת החיתוך של
$l$
עם
$\overline{QQ'}$,
וסמן ב-%
$V$
את נקודת החיתוך שלו עם
$\overline{QB'}$.
$\triangle VUQ\cong \triangle VUQ'$
לפי צלע-זווית-צלע כי:
$\overline{VU}$
הוא צלע משותף, הזוויות ב-%
$U$
הן זוויות ישרות, ו-%
$\overline{QU}=\overline{Q'U}=b$.
מכאן ש-%
$\angle VQU=\angle VQ'U=\alpha$
ו-%
$\angle Q'QR=\angle VQ'U=\alpha$
לפי זוויות מתחלפות.

כמו בהוכחה הראשונה, הנקודות
$A', B', Q'$
הן כולן שיקופים סביב
$l$,
לכן הן כולן נמצאות על קטע קו אחד
$\overline{DQ'}$,
ו-%
$\overline{A'B'}=\overline{AB}=\overline{BQ}=\overline{B'Q'}=a$.
מכאן ש-%
$\triangle A'B'Q\cong\triangle Q'B'Q$
ו-%
$\angle A'QB'=\angle Q'QB'=\alpha$.


\newpage

\section{הבנייה של
\L{Martin}
לחלוקת זווית לשלושה חלקים%
}\label{s.tri2}

\subsection{הבנייה}

\begin{center}
\selectlanguage{english}
\begin{tikzpicture}[scale=.9]

% Place points P, Q, R
\coordinate (P) at (60:10cm); %(5,8.67);
\coordinate (Q) at (0,0);
\coordinate (R) at (10,0);
\fill (P) circle (2pt) node[below right] {$P$};
\fill (Q) circle (2pt) node[above left] {$Q$};
\fill (R) circle (2pt) node[right] {$R$};

% Draw PQR
\draw [very thick] (R)  -- (Q);
\draw [very thick,name path=pq] (Q) -- (P);

% M is the midpoint of PQ
\coordinate (M) at (2.5, 4.33);
\fill (M) circle (2pt) node[above left,xshift=2pt] {$M$};
\draw [rotate=-90] (M) rectangle +(8pt,8pt);

% Drop a perpendicular from M to QR and extend the line upwards
% This is the given line p
\coordinate (pQR) at (M |- Q);
\draw [thick,name path=p] (pQR) --
   node[left, very near end,yshift=28pt] {$p$}
   ($(pQR)!2!(M)$);
\draw (pQR) rectangle +(8pt,8pt);

% Construct q perpendicular to p through M
\draw [thick,name path=q] ($(M)+(-2,0)$) --
   node[above, very near start,xshift=-30pt] {$q$}
   ($(M)+(10,0)$);

% Construct the fold line t
% Its equation is y = -2.75x + 18.51, as obtained from Geogebra
\coordinate (t1) at (6.7,.085);
\coordinate (t2) at (3.5,8.89);
\draw [very thick,dashed,name path=t] (t1) --
   node[very near end,left] {$l$}
   (t2);

% Construct a perpendicular to t through P
\coordinate (perp-p) at ($(t1)!(P)!(t2)$);
\path [name path=perp-p] (P) -- ($(P)!2.5!(perp-p)$);

% Get its intersection with t denoted Pt
% and its intersection with p named PP
\path [name intersections = {of = t and perp-p, by = {Pt}}];
\path [name intersections = {of = p and perp-p, by = {PP}}];
\fill (PP) circle(2pt) node[left] {$P'$};
\draw [rotate=22] (Pt) rectangle +(8pt,8pt);

% Draw PT
\draw [very thick,dotted] (P) -- (PP);

% Construct a perpendicular to t through Q
\coordinate (perp-q) at ($(t1)!(Q)!(t2)$);
\path[name path=perp-q] (Q) -- ($(Q)!2.1!(perp-q)$);

% Get its intersection with t denoted V
% and its intersection with q denoted S=Q'
\path [name intersections = {of = t and perp-q, by = {V}}];
\path [name intersections = {of = q and perp-q, by = {QP}}];
\fill (QP) circle(2pt) node[above,yshift=4pt] {$Q'$};
\fill (V) circle(2pt) node[above left,xshift=-4pt,yshift=-2pt] {$V$};
\draw [rotate=22] (V) rectangle +(8pt,8pt);

% Draw Q QP
\draw [very thick,dotted,name path=qs] (Q) -- (QP);

% Get the intersection of QS with p denoted U
\path [name intersections = {of = p and qs, by = {U}}];
\fill (U) circle(2pt) node[above left] {$U$};

% Draw PP QP
\draw [very thick,dotted,name path=ts] (PP) -- (QP);

% Get its intersection with QP denoted W
\path [name intersections = {of = ts and pq, by = {W}}];
\fill (W) circle(2pt) node[right,xshift=4pt,yshift=4pt] {$W$};

% Label line segments
\path (P) -- node[left] {$a$} (M);
\path (M) -- node[left]  {$a$} (Q);
\path (PP) -- node[left]  {$b$} (M);
\path (M) -- node[right] {$b$} (U);
\path (Q) -- node[below,near end] {$c$} (V);
\path (V) -- node[below] {$c$} (QP);

% Label angles
\node [xshift=5pt,yshift=20pt]        at (M) {$\gamma$};
\node [xshift=-5pt,yshift=-20pt]      at (M) {$\gamma$};
\node [xshift=18pt,yshift=15pt]       at (Q) {$\beta$};
\node [xshift=-18pt,yshift=-15pt]     at (P) {$\beta$};
\node [left,xshift=-30pt,yshift=7pt]  at (QP) {$\alpha$};
\node [left,xshift=-30pt,yshift=-7pt] at (QP) {$\alpha$};
\node [right,xshift=34pt,yshift=7pt]  at (Q) {$\alpha$};
\end{tikzpicture}
\end{center}

נתונה זווית חדה
$\angle PQR$,
תהי
$M$
נקודת האמצע של
$\overline{PQ}$.
בנה
$p$
ניצב ל-%
$\overline{QR}$
העובר דרך
$M$
ובנה
$q$
ניצב ל-%
$p$
העובר דרך 
$M$.
$q$
מקביל ל-%
$\overline{QR}$.

לפי אקסיומה $6$, בנה קיפול
$l$
המניח את
$P$
ב-%
$P'$
על
$p$,
ומניח את
$Q$
ב-%
$Q'$
על
$q$.
ייתכן שקיים מספר קיפולים מתאימים; בחר את הקיפול החותך את
$\overline{PM}$.

בנה את קטעי הקו
$\overline{PP'}$
ו-%
$\overline{QQ'}$.
סמן ב-%
$U$
את נקודת החיתוך של
$\overline{QQ'}$
עם
$p$,
וסמן ב-%
$V$
את נקודת החיתוך שלו עם
$l$.
סמן ב-%
$W$
את החיתוכים של 
$\overline{PQ}$
ו-%
$\overline{P'Q'}$
עם
$l$.\footnote{%
לא ברור מאליו ש-%
$\overline{PQ}$
ו-%
$\overline{P'Q'}$
חותכים את
$l$
באותה נקודה. 
$\triangle PP'W\sim\triangle QQ'W$
כך שהגבהים מחלקים את הזוויות 
$\angle PWP', \angle QWQ'$
בצורה דומה וחייבים להיות על אותו קו.%
}

\subsection{הוכחה}

$\triangle QMU\cong \triangle PMP'$
לפי זווית-צלע-זווית:
$\angle P'PM=\angle UQM=\beta$
לפי זוויות מתחלפות;
$\overline{QM}=\overline{MP}=a$
כי 
$M$
היא נקודת האמצע של
$\overline{PQ}$;
$\angle QMU=\angle PMP'$
הן זוויות קודקודיות. מכאן ש-%
$\overline{P'M}=\overline{MU}=b$.

$\triangle P'MQ'\cong\triangle UMQ'$
לפי צלע-זוויות-צלע: הראנו ש-%
$\overline{P'M}=\overline{MU}=b$;
הזוויות ב-%
$M$
הן זוויות ישרות; 
$\overline{MQ'}$
הוא צלע משותף. הגובה של המשולש שווה-שוקיים 
$\triangle P'Q'U$ 
הוא חוצה הזווית
$\angle P'Q'U$
ולכן
$\angle P'Q'M=\angle UQ'M=\alpha$.


$\triangle QWV\cong\triangle Q'WV$
לפי צלע-זווית-צלע:
$\overline{QV}=\overline{VQ'}=c$
והזוויות ב-%
$V$
הן זוויות ישרות כי הקיפול הוא האנך אמצעי של
$\overline{QQ'}$; $\overline{VW}$
הוא צלע משותף. מכאן ש-%
$\angle WQV=\beta=\angle WQ'V=2\alpha$.
הראנו ש-%
$\angle PQR = \beta + \alpha = 2\alpha+\alpha=3\alpha$
ולכן
$\angle Q'QR$
היא שליש מ-%
$\angle PQR$.
