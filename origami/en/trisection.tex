% !TeX root = origami-math-en.tex

%%%%%%%%%%%%%%%%%%%%%%%%%%%%%%%%%%%%%%%%%%%%%%%%%%%%%%%%%%%%%%%%
\chapter{Trisecting an Angle}\label{c.trisection}

\section{Abe's trisection of an angle}\label{s.tri1}

\subsection{The construction}

\begin{center}
\begin{tikzpicture}[scale=1]

% Place points P, Q, R
\coordinate (P) at (60:10cm); %(5,8.67);
\coordinate (Q) at (0,0);
\coordinate (R) at (10,0);
\fill (P) circle (2pt) node[below right] {$P$};
\fill (Q) circle (2pt) node[left] {$Q$};
\fill (R) circle (2pt) node[right] {$R$};

% Draw PQR
\draw [very thick] (P)  -- (Q) -- (R);

% Draw perpendicular to QR
\draw [thick] (Q) -- node[left,very near end] {$p$} +(0,11);

% Draw parallel to QR and parallel halfway
\coordinate (A) at (0,5);
\coordinate (B) at (0,2.5);
\draw [thick] (A) -- node[above,very near end] {$q$} +(10,0);
\draw [thick] (B) -- node[above,very near end] {$r$} +(10,0);
\fill (A) circle (2pt) node[left] {$A$};
\fill (B) circle (2pt) node[left] {$B$};
\path (Q) -- node[left] {$a$} (B) -- node[left] {$a$} (A);
\draw (A) rectangle +(8pt,8pt);
\draw (B) rectangle +(8pt,8pt);

% Tangent line y = -2.75x + 10.69

% Draw fold
\coordinate (D) at (0,10.69);
\coordinate (fold-x) at (3.89,0);
\coordinate (AP) at (3.65,6.33);
\coordinate (QP) at (6.87,2.5);
\coordinate (BP) at (5.26,4.42);
\fill (D) circle (2pt) node[left] {$D$};
\fill (AP) circle (2pt) node[above,yshift=6pt] {$A'$};
\fill (QP) circle (2pt) node[above,yshift=6pt] {$Q'$};
\fill (BP) circle (2pt) node[above,xshift=2pt,yshift=2pt] {$B'$};
\draw [very thick,dashed] (D) -- node[left,near start] {$l$} (fold-x);

% Draw line of reflections
\draw [very thick, dotted] (D) -- (QP);

% Draw trisecting lines
\draw [very thick,dotted] (Q) -- ($(Q)!1.3!(QP)$);
\draw [very thick,dotted] (Q) -- ($(Q)!1.3!(BP)$);

% Complete triangle
\draw [very thick,dotted] (A) -- (QP);

\end{tikzpicture}
\end{center}

Given an acute angle $\angle PQR$, let $p$ be the perpendicular to $\overline{QR}$ at $Q$. Let $q$ be a perpendicular to $p$ that intersects $\overline{PQ}$ at point $A$, and let $r$ be the perpendicular to $p$ at $B$ that is halfway between $Q$ and $A$.

Using Axiom~6, construct a fold $l$ that places $A$ at $A'$ on $\overline{PQ}$ and $Q$ at $Q'$ on $r$. Let $B'$ be the reflection of $B$ around $l$.

Draw the lines $\overline{QB'}$ and $QQ'$. We claim that $\angle PQB'$, $\angle B'QQ'$ and $\angle Q'QR$ are a trisection of $\angle PQR$.

\subsection{First proof}

\begin{center}
\begin{tikzpicture}[scale=1]

% Place points P, Q, R
\coordinate (P) at (60:10cm);
\coordinate (Q) at (0,0);
\coordinate (R) at (10,0);
\fill (P) circle (2pt) node[below right] {$P$};
\fill (Q) circle (2pt) node[left,xshift=-4pt] {$Q$};
\fill (R) circle (2pt) node[right] {$R$};

% Draw PQR
\draw [very thick] (Q) -- (R);

% Draw perpendicular to QR
\draw [thick] (Q) -- node[left,very near end] {$p$} +(0,11);

% Draw parallel to QR and parallel halfway
\coordinate (A) at (0,5);
\coordinate (B) at (0,2.5);
\draw [thick] (A) -- node[above,very near end] {$q$} +(10,0);
\draw [thick] (B) -- node[above,very near end] {$r$} +(10,0);
\fill (A) circle (2pt) node[left,xshift=-4pt] {$A$};
\fill (B) circle (2pt) node[left,xshift=-4pt] {$B$};
\path (Q) -- node[left,xshift=-4pt] {$a$} (B) -- node[left,xshift=-4pt] {$a$} (A);
\draw (A) rectangle +(8pt,8pt);
\draw (B) rectangle +(8pt,8pt);

% Tangent line y = -2.75x + 10.69

% Draw fold
\coordinate (D) at (0,10.69);
\coordinate (fold-x) at (3.89,0);
\coordinate (AP) at (3.65,6.33);
\coordinate (QP) at (6.87,2.5);
\coordinate (BP) at (5.26,4.42);
\fill (D) circle (2pt) node[left] {$D$};
\fill (AP) circle (2pt) node[above,yshift=6pt] {$A'$};
\fill (QP) circle (2pt) node[above,xshift=2pt,yshift=6pt] {$Q'$};
\fill (BP) circle (2pt) node[above,xshift=4pt,yshift=2pt] {$B'$};
\draw [very thick,dashed] (D) -- node[left,near start] {$l$} (fold-x);
	
% Draw line of reflections
\draw [very thick, dotted] (D) -- (AP);

% Draw trisecting lines
\draw [very thick,dotted] (Q) -- ($(Q)!1.3!(BP)$);

\draw [very thick,loosely dash dot,red] (Q) -- (QP);
\draw [very thick,loosely dash dot,red] (QP) -- (AP);
\draw [very thick,loosely dash dot,red] (AP) -- (Q);
\draw [very thick,loosely dash dot dot,blue] ($(Q)+(0,-4pt)$) -- ($(QP)+(0,-4pt)$);
\draw [very thick,dash dot dot,blue] ($(QP)+(0,-4pt)$) -- ($(A)+(0,-4pt)$);
\draw [very thick,dash dot dot,blue] ($(A)+(-4pt,0)$) -- ($(Q)+(-4pt,0)$);

\draw [thick,dotted] (A) -- (AP);

\node[left,xshift=-40pt,yshift=7pt] at (QP) {$\alpha$};
\node[left,xshift=-40pt,yshift=-6pt] at (QP) {$\alpha$};
\node[right,xshift=40pt,yshift=6pt] at (Q) {$\alpha$};
\node[right,xshift=40pt,yshift=28pt] at (Q) {$\alpha$};
\node[right,xshift=30pt,yshift=42pt] at (Q) {$\alpha$};

\end{tikzpicture}
\end{center}


Since $A', B', Q'$ are all reflections around the same line $l$ of the points $A,B,Q$ on one line $DQ$, they are all on one line $\overline{DQ'}$. By construction, $\overline{AB}=\overline{BQ}$, $\overline{BQ'}$ is perpendicular to $AQ$; $\overline{BQ'}$ is a common side, so $\triangle ABQ'\cong \triangle QBQ'$ by side-angle-side. Therefore, $\angle AQ'B=\angle QQ'B=\alpha$, since $\overline{Q'B}$ is the perpendicular bisector of the isoceles triangle $\triangle AQ'Q$.

By alternating interior angles, $\angle Q'QR=\angle QQ'B=\alpha$.

By reflection, $\triangle AQ'Q\cong \triangle A'QQ'$.\footnote{The two triangles have been emphasized using different patterns of dashes and dots, as well as using color.}
\begin{quote}
The fold $l$ is the perpendicular bisector of both $\overline{AA'}$ and $\overline{QQ'}$; drop perpendiculars from $A$ and $A'$ to $\overline{QQ'}$; then $\overline{AQ}=\overline{A'Q'}$ follows by congruent right triangles. $\overline{AA'Q'Q}$ is an isoceles trapezoid so its diagonals are equal $\overline{AQ'}=\overline{A'Q}$.
\end{quote}
Therefore, $\overline{QB'}$, the reflection of $\overline{Q'B}$, is the perpendicular bisector of an isoceles triangle and $\angle A'QB'=\angle Q'QB'=\angle QQ'B=\alpha$.


\subsection{Second proof}

\begin{center}
\begin{tikzpicture}[scale=1]

% Place points P, Q, R
\coordinate (P) at (60:10cm); %(5,8.67);
\coordinate (Q) at (0,0);
\coordinate (R) at (10,0);
\fill (P) circle (2pt) node[below right] {$P$};
\fill (Q) circle (2pt) node[left] {$Q$};
\fill (R) circle (2pt) node[right] {$R$};

% Draw PQR
\draw [very thick] (P)  -- (Q) -- (R);

% Draw perpendicular to QR
\draw [thick] (Q) -- node[left,very near end] {$p$} +(0,11);

% Draw parallel to QR and parallel halfway
\coordinate (A) at (0,5);
\coordinate (B) at (0,2.5);
\draw [thick] (A) -- node[above,very near end] {$q$} +(10,0);
\draw [thick] (B) -- node[above,very near end] {$r$} +(10,0);
\fill (A) circle (2pt) node[left] {$A$};
\fill (B) circle (2pt) node[left] {$B$};
\path (Q) -- node[left] {$a$} (B) -- node[left] {$a$} (A);
\draw (A) rectangle +(8pt,8pt);
\draw (B) rectangle +(8pt,8pt);

% Tangent line y = -2.75x + 10.69

% Draw fold
\coordinate (D) at (0,10.69);
\coordinate (fold-x) at (3.89,0);
\coordinate (AP) at (3.65,6.33);
\coordinate (QP) at (6.87,2.5);
\coordinate (BP) at (5.26,4.42);
\fill (D) circle (2pt) node[left] {$D$};
\fill (AP) circle (2pt) node[above,yshift=6pt] {$A'$};
\fill (QP) circle (2pt) node[above,yshift=6pt] {$Q'$};
\fill (BP) circle (2pt) node[above,xshift=2pt,yshift=2pt] {$B'$};
\draw [very thick,dashed,name path=fold] (D) -- node[left,near start] {$l$} (fold-x);

% Draw line of reflections
\draw [very thick, dotted] (D) -- (QP);

% Draw trisecting lines
\draw [very thick,dotted,name path=Qr] (Q) -- ($(Q)!1.3!(QP)$);
\draw [very thick,dotted,name path=Qq] (Q) -- ($(Q)!1.3!(BP)$);

% Draw indications of right angles
\draw[rotate=-140] (BP) rectangle +(8pt,8pt);
\path [name intersections = {of = fold and Qr, by = {U}}];
\fill (U) circle (2pt) node[above left,xshift=-2pt,yshift=-2pt] {$U$};
\draw[rotate=20] (U) rectangle +(8pt,8pt);
\path [name intersections = {of = fold and Qq, by = {V}}];
\fill (V) circle (2pt) node[above left,xshift=-2pt,yshift=-2pt] {$V$};

\path (Q) -- node[below,near end] {$b$} (U);
\path (U) -- node[below] {$b$} (QP);

\node[left,xshift=-40pt,yshift=-6pt] at (QP) {$\alpha$};
\node[right,xshift=40pt,yshift=6pt] at (Q) {$\alpha$};
\node[right,xshift=40pt,yshift=28pt] at (Q) {$\alpha$};
\node[right,xshift=30pt,yshift=42pt] at (Q) {$\alpha$};
\end{tikzpicture}
\end{center}


Since $l$ is a fold, it is the perpendicular bisector of $\overline{QQ'}$. Denote the intersection of $l$ with $\overline{QQ'}$ by $U$, and its intersection with $\overline{QB'}$ by $V$. $\triangle VUQ\cong \triangle VUQ'$ by side-angle-side since $\overline{VU}$ is a common side,  the angles at $U$ are right angles and $\overline{QU}=\overline{Q'U}=b$. Therefore, $\angle VQU=\angle VQ'U=\alpha$ and then $\angle Q'QR=\angle VQ'U=\alpha$ by alternating interior angles.

As in Proof 1, $A', B', Q'$ are all reflections around $l$, so they are all on one line $\overline{DQ'}$, and $\overline{A'B'}=\overline{AB}=\overline{BQ}=\overline{B'Q'}=a$. Then $\triangle A'B'Q\cong\triangle Q'B'Q$ and $\angle A'QB'=\angle Q'QB'=\alpha$.


\newpage

\section{Martin's trisection of an angle}\label{s.tri2}

\subsection{The construction}

\begin{center}
\begin{tikzpicture}[scale=.9]

% Place points P, Q, R
\coordinate (P) at (60:10cm); %(5,8.67);
\coordinate (Q) at (0,0);
\coordinate (R) at (10,0);
\fill (P) circle (2pt) node[below right] {$P$};
\fill (Q) circle (2pt) node[above left] {$Q$};
\fill (R) circle (2pt) node[right] {$R$};

% Draw PQR
\draw [very thick] (R)  -- (Q);
\draw [very thick,name path=pq] (Q) -- (P);

% M is the midpoint of PQ
\coordinate (M) at (2.5, 4.33);
\fill (M) circle (2pt) node[above left,xshift=2pt] {$M$};
\draw [rotate=-90] (M) rectangle +(8pt,8pt);

% Drop a perpendicular from M to QR and extend the line upwards
% This is the given line p
\coordinate (pQR) at (M |- Q);
\draw [thick,name path=p] (pQR) --
   node[left, very near end,yshift=28pt] {$p$}
   ($(pQR)!2!(M)$);
\draw (pQR) rectangle +(8pt,8pt);

% Construct q perpendicular to p through M
\draw [thick,name path=q] ($(M)+(-2,0)$) --
   node[above, very near start,xshift=-30pt] {$q$}
   ($(M)+(10,0)$);

% Construct the fold line t
% Its equation is y = -2.75x + 18.51, as obtained from Geogebra
\coordinate (t1) at (6.7,.085);
\coordinate (t2) at (3.5,8.89);
\draw [very thick,dashed,name path=t] (t1) --
   node[very near end,left] {$l$}
   (t2);

% Construct a perpendicular to t through P
\coordinate (perp-p) at ($(t1)!(P)!(t2)$);
\path [name path=perp-p] (P) -- ($(P)!2.5!(perp-p)$);

% Get its intersection with t denoted Pt
% and its intersection with p named PP
\path [name intersections = {of = t and perp-p, by = {Pt}}];
\path [name intersections = {of = p and perp-p, by = {PP}}];
\fill (PP) circle(2pt) node[left] {$P'$};
\draw [rotate=22] (Pt) rectangle +(8pt,8pt);

% Draw PT
\draw [very thick,dotted] (P) -- (PP);

% Construct a perpendicular to t through Q
\coordinate (perp-q) at ($(t1)!(Q)!(t2)$);
\path[name path=perp-q] (Q) -- ($(Q)!2.1!(perp-q)$);

% Get its intersection with t denoted V
% and its intersection with q denoted S=Q'
\path [name intersections = {of = t and perp-q, by = {V}}];
\path [name intersections = {of = q and perp-q, by = {QP}}];
\fill (QP) circle(2pt) node[above,yshift=4pt] {$Q'$};
\fill (V) circle(2pt) node[above left,xshift=-4pt,yshift=-2pt] {$V$};
\draw [rotate=22] (V) rectangle +(8pt,8pt);

% Draw Q QP
\draw [very thick,dotted,name path=qs] (Q) -- (QP);

% Get the intersection of QS with p denoted U
\path [name intersections = {of = p and qs, by = {U}}];
\fill (U) circle(2pt) node[above left] {$U$};

% Draw PP QP
\draw [very thick,dotted,name path=ts] (PP) -- (QP);

% Get its intersection with QP denoted W
\path [name intersections = {of = ts and pq, by = {W}}];
\fill (W) circle(2pt) node[right,xshift=4pt,yshift=4pt] {$W$};

% Label line segments
\path (P) -- node[left] {$a$} (M);
\path (M) -- node[left]  {$a$} (Q);
\path (PP) -- node[left]  {$b$} (M);
\path (M) -- node[right] {$b$} (U);
\path (Q) -- node[below,near end] {$c$} (V);
\path (V) -- node[below] {$c$} (QP);

% Label angles
\node [xshift=5pt,yshift=20pt]        at (M) {$\gamma$};
\node [xshift=-5pt,yshift=-20pt]      at (M) {$\gamma$};
\node [xshift=18pt,yshift=15pt]       at (Q) {$\beta$};
\node [xshift=-18pt,yshift=-15pt]     at (P) {$\beta$};
\node [left,xshift=-30pt,yshift=7pt]  at (QP) {$\alpha$};
\node [left,xshift=-30pt,yshift=-7pt] at (QP) {$\alpha$};
\node [right,xshift=34pt,yshift=7pt]  at (Q) {$\alpha$};
\end{tikzpicture}
\end{center}


Given the acute angle $\angle PQR$, let $M$ be the midpoint of $\overline{PQ}$. Construct $p$ the perpendicular to $\overline{QR}$ through $M$ and construct $q$ perpendicular to $p$ through $M$. $q$ is parallel to $\overline{QR}$.

Using Axiom 6, construct a fold $l$ that places $P$ at $P'$ on $p$ and $Q$ at $Q'$ on $q$. More than one fold may be possible; choose the one that intersects $\overline{PM}$.

Draw the lines $\overline{PP'}$ and $\overline{QQ'}$. Denote the intersection of $\overline{QQ'}$ with $p$ by $U$ and its intersection with $l$ by $V$. Denote the intersection of $\overline{PQ}$ and $P'Q'$ with $l$ by $W$.\footnote{It is not immediate that both $\overline{PQ}$ and $P'Q'$ intersect $l$ at the same point. $\triangle PP'W \sim \triangle QQ'W$ so the altitudes divide the vertical angles $\angle PWP', \angle QWQ'$ similarly and thus must be on the same line.}

\subsection{Proof}

$\triangle QMU\cong \triangle PMP'$ by angle-side-angle:  $\angle P'PM=\angle UQM=\beta$ by alternate interior angles; $\overline{QM}=\overline{MP}=a$ since $M$ is the midpoint of $\overline{PQ}$; $\angle QMU=\angle PMP'$ are vertical angles. Therefore, $\overline{P'M}=\overline{MU}=b$.

$\triangle P'MQ'\cong \triangle UMQ'$ by side-angle-side: we have shown that $\overline{P'M}=\overline{MU}=b$; the angles at $M$ are right angles; $\overline{MQ'}$ is a common side. Since the altitude of the isoceles triangle $\triangle P'Q'U$ is the bisector of $\angle P'Q'U$, so $\angle P'Q'M=\angle UQ'M=\alpha$.

$\triangle QWV\cong\triangle Q'WV$ by side-angle-side: $\overline{QV}=\overline{VQ'}=c$ and the angles at $V$ are right angles since the fold is the perpendicular bisector of $\overline{QQ'}$; $\overline{VW}$ is a common side. Therefore, $\angle WQV=\beta=\angle WQ'V=2\alpha$. By alternate interior angles $\angle Q'QR=\angle MQ'Q=\alpha$. We have $\angle PQR = \beta + \alpha = 2\alpha+\alpha=3\alpha$ so $\angle Q'QR$ is one-third of $\angle PQR$.
