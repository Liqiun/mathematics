% !TeX root = origami-math-en.tex

%%%%%%%%%%%%%%%%%%%%%%%%%%%%%%%%%%%%%%%%%%%%%%%%%%%%%%%%%%%%%%%%
\chapter{Beloch's Fold and Beloch's Square}\label{c.beloch}

\section{The Beloch fold}\label{s.beloch-fold}

Margharita P. Beloch discovered a remarkable connection between origami and Lill's method for finding roots of polynomials. She found that one application of the operation of origami Axiom~6 (Section~\ref{s.ax6}) applied to the first path of Lill's method can obtain a real root of any cubic polynomial. The operation is often called the \emph{Beloch fold}.

Consider the polynomial $p(x)=x^3+6x^2+11x+6$ from Section~\ref{s.magic}. In the following diagram we have emphasized the second path and renamed some vertices. All we have to do to solve the equation is to perform a Beloch fold to simultaneously place the points $P',Q'$ on the line segments of lengths $a_2,a_1$, respectively. Unfortunately, if you perform the fold, the path does not solve the equation: $Q'$ is way off to the right, so the angles at $P'$ and $Q'$ are not right angles.
\begin{center}
\begin{tikzpicture}[scale=.8]
% Draw help lines and axes
\draw[step=10mm,white!60!black] (-11,-1) grid (2,7);
\draw[thick] (-11,0) -- (2,0);
\draw[thick] (0,-1) -- (0,7);
\foreach \x in {-10,...,2}
  \node at (\x-.3,-.2) {\sm{\x}};
\foreach \y in {1,...,7}
  \node at (-.2,\y-.3) {\sm{\y}};
  
%Draw first path with five points
\coordinate (A) at (0,0);
\coordinate (B) at (1,0);
\coordinate (C) at (1,6);
\coordinate (D) at (-10,6);
\coordinate (E) at (-10,0);
\foreach \x in {A,B,C,D,E}
  \fill (\x) circle(2pt);
\node[below right,yshift=-6pt]  at (A) {$P$};
\node[below left,yshift=-6pt] at (E) {$Q$};

\draw[thick] (A) -- (B);
\draw[thick,name path=bc] (B) --
  node[right,near end] {$a_2$} (C);
\draw[thick,name path=cd] (C) -- 
  node[above] {$a_1$} (D);
\draw[thick,name path=de] (D) -- (E);

% Draw first segment of second path
\path[name path=a2] (A) -- +(63.4:4);
\path [name intersections = {of = a2 and bc, by = {A2}}];
\fill (A2) circle(2pt) node[above right] {$P'$};
\draw[ultra thick,dotted] (A) -- (A2);
\draw[rotate=153.4] (A2) rectangle +(10pt,10pt);

% Draw second segment of second path
\path[name path=b2] (A2) -- +(153.4:10);
\path [name intersections = {of = b2 and cd, by = {B2}}];
\fill (B2) circle(2pt) node[above left]  {$Q'$};
\draw[ultra thick,dotted] (A2) -- (B2);
\draw[rotate=243.4] (B2) rectangle +(10pt,10pt);

% Draw third segment of second path
\draw[ultra thick,dotted] (B2) -- (E);
\end{tikzpicture}
\end{center}

\newpage

Recall that a fold is the perpendicular bisector of the line segment between any point and its reflection around the fold. We want $\overline{P'Q'}$ to be a fold so that it will be perpendicular to both $\overline{QQ'}$ and $\overline{PP'}$. If $\overline{P'Q'}$ is the perpendicular bisector of $\overline{QQ'}$ and $\overline{PP'}$, then $P',Q'$, the reflections of $P,Q$, must be the same distance away from the fold as $P$ and $Q$, respectively. With some change of notation we have the following diagram.

\begin{center}
\begin{tikzpicture}[scale=.8]
% Draw help lines and axes
\draw[step=10mm,white!60!black] (-11,-1) grid (3,13);
\draw[thick] (-11,0) -- (3,0);
\draw[thick] (0,-1) -- (0,13);
\foreach \x in {-10,...,3}
  \node at (\x-.3,-.2) {\sm{\x}};
\foreach \y in {1,...,13}
  \node at (-.2,\y-.3) {\sm{\y}};
  
% Draw first path with five points
\coordinate (A) at (0,0);
\coordinate (B) at (1,0);
\coordinate (C) at (1,6);
\coordinate (D) at (-10,6);
\coordinate (E) at (-10,0);
\foreach \x in {A,B,C,D,E}
  \fill (\x) circle(2pt);
\node[below right,yshift=-6pt] at (A) {$P$};
\node[below left,yshift=-6pt] at (E) {$Q$};

\draw[thick] (A) -- (B);
\draw[thick,name path=bc] (B) -- node[right,near end] {$a_2$} (C);
\draw[thick,name path=cd] (C) -- node[above] {$a_1$} (D);
\draw[thick,name path=de] (D) -- (E);

% Draw parallel lines
\draw[ultra thick,dotted,name path=bpcp] ($(B)+(1,-1)$) --
  node[above right] {$a_2'$}
  ($(C)+(1,7)$);
\draw[ultra thick,dotted,name path=cpdp] ($(C)+(2,6)$) -- 
  node[above left,xshift=-24pt] {$a_1'$} 
  ($(D)+(-1,6)$);

% Draw first segment of second path
\path[name path=a2] (A) -- +(63.4:4);
\path [name intersections = {of = a2 and bc, by = {A2}}];
\draw[ultra thick,dotted] (A) -- (A2);
\fill (A2) circle(2pt) node[above right,xshift=4pt] {$R$};
\draw[rotate=153.4] (A2) rectangle +(10pt,10pt);

% Draw second segment of second path
\path[name path=b2] (A2) -- +(153.4:10);
\path [name intersections = {of = b2 and cd, by = {B2}}];
\fill (B2) circle(2pt) node[above left]  {$S$};
\draw[very thick,dashed] (A2) -- (B2);
\draw[rotate=243.4] (B2) rectangle +(10pt,10pt);

% Draw third segment of second path
\draw[ultra thick,dotted] (B2) -- (E);

% Locate reflections on parallel lines and draw lines
\coordinate (PP) at ($(A2)+(1,2)$);
\fill (PP) circle(2pt) node[above right] {$P'$};
\draw[ultra thick,dotted] (A2) -- (PP);

\coordinate (QP) at ($(B2)+(3,6)$);
\fill (QP) circle(2pt) node[above right] {$Q'$};
\draw[ultra thick,dotted] (B2) -- (QP);
\end{tikzpicture}
\end{center}


A line $a_2'$ is drawn so that it is parallel to $a_2$ and the same distance from $a_2$ as $a_2$ is from $P$. Similarly, line $a_1'$ is drawn so that it is parallel to $a_1$ and the same distance from $a_1$ as $a_1$ is from $Q$. Apply Axiom~6 to simultaneously place $P$ at $P'$ on $a_2'$ and to place $Q$ at $Q'$ on $a_1'$. The fold $\overline{RS}$ is the perpendicular bisector of the lines $\overline{PP'}$ and $\overline{QQ'}$; therefore, the angles at $R$ and $S$ are right angles as required.

\newpage

Let us try the Beloch fold on the polynomial $x^3-3x^2-3x+1$ from Section~\ref{s.negative}. $a_2$ is the vertical line segment of length $3$ whose equation is $x=1$, and its parallel line is $a_2'$ whose equation is $x=2$, because $P$ is at a distance of $1$ from $a_2$. $a_1$ is the horizontal line segment of length $3$ whose equation is $y=-3$, and its parallel line is $a_1'$ whose equation is $y=-2$ because $Q$ is at a distance of $1$ from $a_1$. The fold $RS$ is the perpendicular bisector of both $\overline{PP'}$ and $\overline{QQ'}$. The line $\overline{PRSQ}$ is the same as the second path in Section~\ref{s.negative}.

\begin{center}
\begin{tikzpicture}[scale=1]
% Draw help lines and axes
\draw[step=10mm,white!50!black] (-1,-5) grid (6,2);
\foreach \x in {0,...,6}
  \node at (\x-.3,-.2) {\sm{\x}};
\foreach \y in {-4,...,-1}
  \node at (-.3,\y-.3) {\sm{\y}};
\foreach \y in {1,...,2}
  \node at (-.3,\y-.3) {\sm{\y}};

% Draw first path
\coordinate (A) at (0,0);
\coordinate (B) at (1,0);
\coordinate (C) at (1,-3);
\coordinate (D) at (4,-3);
\coordinate (E) at (4,-4);
\node[above left] at (A) {$P$};
\node[below right] at (E) {$Q$};
\foreach \x in {A,B,C,D,E}
  \fill (\x) circle(2pt);

\draw[very thick,{Stealth[scale=1.4,inset=2pt,reversed]}-] (A) --
  (B);
\draw[very thick,{Stealth[scale=1.4,inset=2pt]}-,name path=bc] (B) -- 
  node[left] {$a_2$} (C);
\draw[very thick,{Stealth[scale=1.4,inset=2pt]}-,name path=cd] (C) --
  node[above] {$a_1$}(D);
\draw[very thick,{Stealth[scale=1.4,inset=2pt,reversed]}-,name path=de] (D) --
 (E);

% Draw extensions of first path
\draw[very thick,loosely dotted,name path=b] (1,-4) -- (1,2);
\draw[very thick,loosely dotted,name path=c] (-1,-3) -- (6,-3);

% Draw reflected points
\coordinate (PP) at (2,2);
\coordinate (QP) at (6,-2);
\fill (PP) circle(2pt) node[above left] {$P'$};
\fill (QP) circle(2pt) node[below right] {$Q'$};

% Midpoints of bisected lines
\coordinate (R) at (1,1);
\coordinate (S) at (5,-3);
\fill (R) circle(2pt) node[above left] {$R$};
\fill (S) circle(2pt) node[below right] {$S$};

% Draw reflected lines
\draw[ultra thick,dotted] ($(B)+(1,2)$) --
  node[right,very near end,yshift=-8pt] {$a_2'$} ($(C)+(1,-2)$);
\draw[ultra thick,dotted] ($(C)+(-2,1)$) --
  node[above,very near start,xshift=-8pt,yshift=-1pt] {$a_1'$} ($(D)+(2,1)$);
\draw[ultra thick,dotted] (A) -- (PP);
\draw[ultra thick,dotted] (E) -- (QP);

% Draw fold
\draw[very thick,dashed] (R) -- (S);
\draw[rotate=-45] (R) rectangle +(8pt,8pt);
\draw[rotate=45] (S) rectangle +(8pt,8pt);
\end{tikzpicture}
\end{center}

\newpage

\section{The Beloch square}\label{s.beloch-square}

This construction in the previous section can be expressed in terms of a \emph{Beloch square}: Given two points $P,Q$ and two lines $a_2,a_1$, construct a square $\overline{ARSB}$, such that:
\begin{itemize}
\item One side is $\overline{RS}$ where $R$ lies on $a_2$ and $S$ lies on $a_1$;
\item $P$ lies on $\overline{RA}$ and $Q$ lies on $\overline{SB}$.
\end{itemize}
The following diagram extends the construction for $x^3+6x^2+11x+6$ to show the Beloch square. The length of $RS$ is $\sqrt{80}=4\sqrt{5}\approx 8.94$. We can construct the square by adding three sides of this length.
\begin{center}
\begin{tikzpicture}[scale=.8]
% Draw help lines and axes
\draw[step=10mm,white!60!black] (-12,-7) grid (2,7);
%\draw[thick] (-12,0) -- (2,0);
%\draw[thick] (0,-7) -- (0,7);
\foreach \x in {-11,...,3}
  \node at (\x-.3,-.2) {\sm{\x}};
\foreach \y in {1,...,7}
  \node at (-.4,\y-.3) {\sm{\y}};
\foreach \y in {-6,...,-1}
  \node at (-.4,\y-.3) {\sm{\y}};

% Draw first path
\coordinate (A) at (0,0);
\coordinate (B) at (1,0);
\coordinate (C) at (1,6);
\coordinate (D) at (-10,6);
\coordinate (E) at (-10,0);
\foreach \x in {A,B,C,D,E}
  \fill (\x) circle(2pt);
\draw[thick] (A) -- (B);
\draw[thick,name path=bc] (B) -- node[right,near end] {$a_2$} (C);
\draw[thick,name path=cd] (C) -- node[above] {$a_1$} (D);
\draw[thick,name path=de] (D) -- (E);

% Find first segment of second path
\path[name path=a2] (A) -- +(63.4:4);
\path [name intersections = {of = a2 and bc, by = {A2}}];
\draw[rotate=153.4] (A2) rectangle +(10pt,10pt);

% Draw second segment of second path
\path[name path=b2] (A2) -- +(153.4:10);
\path [name intersections = {of = b2 and cd, by = {B2}}];
\draw[very thick,dashed] (A2) -- (B2);
\draw[rotate=243.4] (B2) rectangle +(10pt,10pt);

% Draw square
\draw[very thick,dashed] (B2) -- +(243.4:8.94) coordinate (AB);
\draw[very thick,dashed] (A2) -- +(243.4:8.94) coordinate (BB);
\fill (AB) circle (2pt) node[below left] {$B$};
\fill (BB) circle (2pt) node[below right] {$A$};
\draw[very thick,dashed] (AB) -- (BB);

\path[fill=white!85!black,rotate=-26.6] (AB) rectangle +(8.94,8.94);

% Draw labels of points
\fill (A2) circle(2pt) node[above right,xshift=4pt] {$R$};
\fill (B2) circle(2pt) node[above left]  {$S$};
\fill (A)  circle(2pt) node[below right,yshift=-6pt]  {$P$};
\fill (E)  circle(2pt) node[above left]  {$Q$};
\end{tikzpicture}
\end{center}

