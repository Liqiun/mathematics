% !TeX root = origami-activities-en.tex

%%%%%%%%%%%%%%%%%%%%%%%%%%%%%%%%%%%%%%%%%%%%%%%%%%%%%%%%%%%%%%%%%%
%%%%%%%%%%%%%%%%%%%%%%%%%%%%%%%%%%%%%%%%%%%%%%%%%%%%%%%%%%%%%%%%%%
%%%%%%%%%%%%%%%%%%%%%%%%%%%%%%%%%%%%%%%%%%%%%%%%%%%%%%%%%%%%%%%%%%

\section{Exercises for geometric loci}\label{s.exercises}

The following exercises are rather basic and can be integrated into the activities or used to summarize the activities. Some of them can be solved using algebra alone, but insights from origami can deepened the students' understanding of the axioms.

\begin{enumerate}
\item Find the geometric locus of all the points whose distance from the point $(-8,-6)$ is equal to their distances from $(12,4)$.

\item Find the geometric locus of all the points whose sum or difference of their distances from $P_1=(2,3)$ and $P_2=(6,4)$ is equal to the length of the line segment $\overline{P_1P_2}$.

\item 
\begin{enumerate}
\item Find the geometric locus of all the points whose distance from the line $5x+3y-14=0$ is equal to their distance to the line $3x+5y-34=0$.
\item What is the form of the geometric locus that you found?
\item Explain why all the lines you found in the previous items bisect the angle between the two given lines.
\end{enumerate}

\item Find the geometric locus of all points whose distance from the line $y=2x+6$ is equal to the distance from the line $y=2x-4.5$.

\item 
\begin{enumerate} 
\item Find the geometric locus of all the points whose distance from the point $(3,-6)$ is $9$.
\item What is the form of the geometric locus that you found?
\end{enumerate}

\item Find the geometric locus of all the points on the line $y=x+1$ whose distance from the point $(2,10)$ is $5$.

\item
\begin{enumerate}
\item What is the form of the geometric locus of all points whose distance from the point $(0,3)$ is equal to their distance from the line $y=-3$.
\item What is the geometric locus?
\end{enumerate}

\end{enumerate}

