% !TeX root = origami-activities-en.tex

%%%%%%%%%%%%%%%%%%%%%%%%%%%%%%%%%%%%%%%%%%%%%%%%%%%%%%%%%%%%%%%%%%
%%%%%%%%%%%%%%%%%%%%%%%%%%%%%%%%%%%%%%%%%%%%%%%%%%%%%%%%%%%%%%%%%%
%%%%%%%%%%%%%%%%%%%%%%%%%%%%%%%%%%%%%%%%%%%%%%%%%%%%%%%%%%%%%%%%%%

\section{Solutions to the exercises in Section~\ref{s.exercises}}

\textbf{Question 1} Find the geometric locus of all the points whose distance from the point $(-8,-6)$ is equal to their distances from $(12,4)$.

\textbf{Solution}

The geometric locus of all points equidistant from $P_1$ and $P_2$ is the perpendicular bisector of the segment $\overline{P_1P_2}$. Let us find the equation of the perpendicular bisector of the segment connecting $(-8,-6)$ and $(12,4)$. The slope is the negative inverse of the segment, which is $-2$ since the slope of the segment is:
\[
\disfrac{4-(-6)}{12-(-8)}=\disfrac{1}{2}\,.
\]
The midpoint of $\overline{P_1P_2}$ is:
\[
\left(\disfrac{-8+12}{2},\disfrac{-6+4}{2}\right)=(2,-1)\,.
\]
The equation of the perpendicular bisector is:
\begin{form}{1}
y-(-1)&=&-2(x-2)\\
y&=&-2x+3\,.
\end{form}

\textbf{Question 2} Find the geometric locus of all points whose sum or difference of their distances from $P_1=(2,3)$ and $P_2=(6,4)$ is equal to the length of the line segment $\overline{P_1P_2}$. 

\textbf{Solution} In Activity 3 we saw that the geometric locus of all points whose sum or difference of their distance from $P_1$ and $P_2$ is equal to the length of $\overline{P_1P_2}$ is the line that passes through $P_1$ and $P_2$.

The slope is:
\[
\disfrac{4-3}{6-2}=\disfrac{1}{4}\,,
\]
so the equation of the line is:
\begin{form}{2}
y-3&=&\disfrac{1}{4}(x-2)\\
y&=&\disfrac{2}{3}(x + 1)\,.
\end{form}

\textbf{Question 3}
\begin{enumerate}
\item Find the geometric locus of all points whose distance from the line $5x+3y-14=0$ is equal to their distance to the line $3x+5y-34=0$.
\item What is the form of the geometric locus that you found?
\item Explain why all the lines you found in the previous items bisect the angle between the two given lines.
\end{enumerate}

\textbf{Solution}

\textbf{(1)} Let $(x,y)$ be any point that satisfies the condition. We will use the formula for finding the distance of a point from a line in the plane. We do this for each line and then equate them. The distance of $(x,y)$ from $5x+3y-14$ is:
\[
\left|\disfrac{5x+3y-14}{\sqrt{5^2+3^2}}\right|=\left|\disfrac{5x+3y-14}{\sqrt{34}}\right|\,.
\]
The distance of $(x,y)$ from $3x+5y-34$ is:
\[
\left|\disfrac{3x+5y-34}{\sqrt{3^2+5^2}}\right|=\left|\disfrac{3x+5y-34}{\sqrt{34}}\right|\,.
\]
Equating the formulas and multiplying by $\sqrt{34}$:
\[
|5x+3y-14|=|3x+5y-34|\,.
\]
There are two cases depending on the signs:
\begin{form}{1}
5x+3y-14&=&3x+5y-34\\
x-y+10&=&0\,,
\end{form}
and
\begin{form}{1}
5x+3y-14&=&-(3x+5y-34)\\
x+y-6&=&0\,.
\end{form}

\textbf{(2)} The geometric locus is two lines.

\textbf{(3)} This was proved in the Activity.


\textbf{Question 4}  Find the geometric locus of all points whose distance from the line $y=2x+6$ is equal to the distance from the line $y=2x-4.5$.

\textbf{Solution}

Both of the given lines have the same slope so they are parallel. We saw that when two lines are parallel, the geometric locus of the points equidistant from them is the line parallel to them and equidistant from them. 

The line with have an equation of the form $-2x-y+c=0$. Since it is equidistant from them, we use the equation for the distance between two parallel lines and equate them:
\[
\disfrac{|6-c|}{\sqrt{(-2)^2+(-1)^2}}=\disfrac{|-4.5-c|}{\sqrt{(-2)^2+(-1)^2}}\,.
\]
It follows that:
\[
|6-c|=|-4.5-c|\,,
\]
so the equation has the single solution $c=\disfrac{3}{4}$ and the line of the equation is $-2x-y-\disfrac{3}{4}$.

\newpage

\textbf{Question 5}
\begin{enumerate} 
\item Find the geometric locus of all the points whose distance from the point $(3,-6)$ is $9$.
\item What is the form of the geometric locus that you found?
\end{enumerate}

\textbf{Solution}

\textbf{(1)} Let $(x,y)$ be any point that satisfies the condition. We will use the formula for finding the distance between two points:
\begin{form}{1.3}
9&=&\sqrt{(x-3)^2+(y+6)^2}\\
81&=&(x-3)^2+(y+6)^2\,.
\end{form}

\textbf{(2)} The geometric locus is a circle.

\textbf{Question 6}  Find the geometric locus of all points on the line $y=x+1$ whose distance from the point $(2,10)$ is $5$.

\textbf{Solution} The geometric locus of all points whose distance from $(2,10)$ is $5$ is a circle whose equation is $(x-2)^2+(y-10)^2=25$. We saw that the geometric locus that we seek is the point or points of intersection of the line with the circle. Substitute the equation for the line into the equation of the circle:
\begin{form}{1.5}
(x-2)^2+((x+1)-10)^2&=&25\\
(x-2)^2+(x-9)^2&=&25\\
x^2-4x+4+x^2-18x+81&=&25\\
2x^2-22x+60&=&0\\
x=&=&5,\;6\,.
\end{form}
By substituting these values into the equation for the line we obtain that the two points of intersection are $(6,7),\:(5,6)$.

\textbf{Question $7$} 
\begin{enumerate}
\item What is the form of the geometric locus of all point whose distance from the point $(0,3)$ is equal to their distance from the line $y=-3$.
\item What is the geometric locus?
\end{enumerate}


\textbf{Solution}

\textbf{(1)} The geometric locus is a parabola whose focus is $(0,3)$ and whose directrix is $y=-3$.

\textbf{(2)} We use the formula for the equation of a parabola:
\begin{form}{1}
x^2=4\cdot 3y\\
x^2&=&12y\,.
\end{form}


