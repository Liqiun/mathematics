% !TeX root = origami-activities-en.tex
\section{Geogebra constructions for the origami axioms}

%%%%%%%%%%%%%%%%%%%%%%%%%%%%%%%%%%%%%%%%%%%%%%%%%%%%%%%%%%%%%%%%%%
%%%%%%%%%%%%%%%%%%%%%%%%%%%%%%%%%%%%%%%%%%%%%%%%%%%%%%%%%%%%%%%%%%
%%%%%%%%%%%%%%%%%%%%%%%%%%%%%%%%%%%%%%%%%%%%%%%%%%%%%%%%%%%%%%%%%%

The italicized words refer to menu selections in Geogebra.

\textbf{Axiom $1$} Construct two \emph{points}. Construct a \emph{line} throught the points.

\textbf{Axiom $2$} Construct two \emph{points}. Construct the \emph{perpendicular bisector} of the line \emph{segment} connecting the two points. Construct the \emph{reflection} of one point relative to the line to check that it is placed on the other point.

\textbf{Axiom $3$} For the case where $l_1,l_2$ intersect: construct two \emph{segments} that intersect or that would intersect if extended. Construct the \emph{angle bisector} between the segments. (You can extend the segments so that they intersect but that is not necessary.) Construct the \emph{reflection} of one line relative to the bisector and check if it is placed on the other line. (You can change the color of one line to make it clear that they are placed on each other.)

For the case where $l_1,l_2$ are parallel: construct two parallel \emph{segments}. Construct a \emph{perpendicular} to one segment and construct its \emph{intersection} with the second segment (extended if necessary). Construct the \emph{perpendicular bisector} of the segment between between the two lines. Construct the \emph{reflection} of one line relative to the bisector and check that it is placed on the other line. (You can change the color of one line to make it clear that they are placed on each other.)

\textbf{Axiom $4$} Construct a \emph{line} and a \emph{point} not on the line. Construct a \emph{perpendicular} to the line through the point. Construct a \emph{reflection} of some point on the line relative to the perpendicular to check that it is placed on the line.

\textbf{Axiom $5$} Construct a \emph{line} $l$ and \emph{points} $P_1,P_2$. Construct a \emph{circle through point $P_1$} with center $P_2$. Construct the \emph{intersections} $A,B$ of $l$ and the circle. Construct line \emph{segments} $\overline{P_1A}, \overline{P_1B}$. Construct a \emph{perpendiculars} through $P_2$ those segments. Construct a \emph{reflection} of $P_1$ relative to one of the perpendiculars to check that it is placed on the line. Repeat with the other perpendicular.

\textbf{Axiom $6$} Construct two \emph{lines} $l_1,l_2$ and two \emph{points} $P_1,P_2$. Construct a \emph{parabola} with focus $P_1$ and directrix $l_1$ and \emph{parabola} with focus $P_2$ and directrix $l_2$. Construct \emph{tangents} to both parabolas. Construct the \emph{reflections} of $P_1,P_2$ relative to one of the tangents and check that they are placed on the lines $l_1,l_2$.

\textbf{Axiom $7$} Construct two line \emph{segments} $l_1,l_2$ and a \emph{point} $P$. Construct a \emph{parallel} to $l_2$ through $p$, and construct $P_1'$, the \emph{intersection} of the parallel and $l_1$. (Extend $l_1$ if necessary.) Construct the  \emph{segment} $\overline{PP'}$ and construct its \emph{perpendicular bisector}. Construct the \emph{reflection} of $P$ relative to the bisector and check that it is placed on $P'$.
