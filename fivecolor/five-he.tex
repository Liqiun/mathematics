\documentclass[12pt,a4paper]{article}

\usepackage[utf8x]{inputenc}
\usepackage[english,hebrew]{babel}

\usepackage{verbatim}
\usepackage{url}
\usepackage{bm}

\newtheorem{theorem}{משפט}
\newtheorem{definition}{הגדרה}

\usepackage{tikz}
\usetikzlibrary{intersections,calc,through,arrows.meta,shapes.geometric}
\tikzset {>=Stealth}

\textwidth=15cm
\textheight=23cm
\topmargin=0pt
\headheight=0pt
\oddsidemargin=2em
\headsep=0pt
\parindent=0pt
\renewcommand{\baselinestretch}{1.15}
\setlength{\parskip}{0.3\baselineskip plus 1pt minus 1pt}

\newenvironment{form}[1]{%
\begin{displaymath}%
\renewcommand{\arraystretch}{#1}%
\begin{array}{lcl}}%
{\end{array}%
\end{displaymath}%
}

\newcommand*{\qed}{\hfill\rule{1ex}{1.5ex}}

\begin{document}
\thispagestyle{empty}
\selectlanguage{hebrew}
\begin{center}
\textbf{\LARGE משפט חמשת הצבעים}

\bigskip

\textbf{\Large מוטי בן-ארי\\\bigskip\url{http://www.weizmann.ac.il/sci-tea/benari/}}
\smallskip

גרסה
$1.1$
\end{center}

\begin{footnotesize}
\begin{center}
\L{\copyright{}\  2021} מוטי בן-ארי. 
\end{center}
\selectlanguage{english}
This work is licensed under the Creative Commons Attribution-ShareAlike 3.0 Unported License. To view a copy of this license, visit \url{http://creativecommons.org/licenses/by-sa/3.0/} or send a letter to Creative Commons, 444 Castro Street, Suite 900, Mountain View, California, 94041, USA.

\end{footnotesize}
\selectlanguage{hebrew}


\section{מפות מישוריות וגרפים מישוריים}\label{s.planar}

\begin{theorem}
ניתן לצבוע מפה מישורית עם ארבעה צבעים כך ששני שטחים שכנים צבועים בצבע שונה.
\end{theorem}

הוכחת משפט זה קשה ביותר; כאן נוכיח משפט הרבה יותר קל, משפט חמשת הצבעים, משפט שהוכח במאה התשע-עשרה.

\begin{theorem}
ניתן לצבוע מפה מישורית עם חמישה צבעים כך ששני שטחים שכנים צבועים בצבעים שונים.
\end{theorem}

\begin{definition}
\textbf{מפה מישורית}
היא אוסף של שטחים במישור עם גבולות משותפים.
\textbf{צביעה}
של מפה היא השמה של צבעים לשטחים כך שכל זוג שטחים שיש להם גבולות  משותפים צבועים בצבעים שונים.%
\footnote{%
שטחים שאין להם גבול משותף יכולים להיחשב כ-"אותו שטח", למשל, מדינת אלסקה ומדינת וושינגטון נחשבות כחלק מארה"ב למרות שאין להם גבול משותף, ואי אפשר לנסוע מאחד לשני בלי לעבור דרך ארץ אחרת (קנדה). בבעיה המתימטית, נתייחס אל שתי המדינות כשטחים שונים שניתן לצבוע באותו צבע או בצבע שונה.%
}
\end{definition}

האיורים שלהלן מראים מפה מישורית עם עשרה שטחים. האיור משמאל מראה צביעה עם חמישה צבעים והאיור מימין מראה צביעה עם ארבעה צבעים.
\begin{center}
\selectlanguage{english}
\begin{tikzpicture}[scale=.667]
\draw[fill=blue!80] (-3.5,-3.5) rectangle +(7,7);

\draw[fill=green] (0:1) 
  arc [start angle=0,  end angle=360, radius=1];

\draw[fill=green] (45:2) --
      (45:3)  arc[start angle=45,  end angle=135, radius=3] --
      (135:2) arc[start angle=135, end angle=45,  radius=2];
\draw[fill=orange] (-45:2) --
      (-45:3)  arc[start angle=-45,  end angle=-135, radius=3] --
      (-135:2) arc[start angle=-135, end angle=-45,  radius=2];
\draw[fill=yellow] (45:2) --
      (45:3)  arc[start angle=45,  end angle=-45, radius=3] --
      (-45:2) arc[start angle=-45, end angle=45,  radius=2];
\draw[fill=red] (135:2) --
      (135:3)  arc[start angle=135,  end angle=225, radius=3] --
      (225:2) arc[start angle=225, end angle=135,  radius=2];

\draw[fill=orange] (0:1) --
      (0:2)  arc[start angle=0,  end angle=90, radius=2] --
      (90:1) arc[start angle=90, end angle=0,  radius=1];
\draw[fill=red] (0:1) --
      (0:2)  arc[start angle=0,  end angle=-90, radius=2] --
      (-90:1) arc[start angle=-90, end angle=0,  radius=1];
\draw[fill=yellow] (90:1) --
      (90:2)  arc[start angle=90,  end angle=180, radius=2] --
      (180:1) arc[start angle=180, end angle=90,  radius=1];
\draw[fill=blue!80] (180:1) --
      (180:2)  arc[start angle=180,  end angle=270, radius=2] --
      (270:1) arc[start angle=270, end angle=180,  radius=1];

\begin{scope}[xshift=10cm]
\draw[fill=blue!80] (-3.5,-3.5) rectangle +(7,7);

\draw[fill=green] (0:1) 
  arc [start angle=0,  end angle=360, radius=1];

\draw[fill=green] (45:2) --
      (45:3)  arc[start angle=45,  end angle=135, radius=3] --
      (135:2) arc[start angle=135, end angle=45,  radius=2];
\draw[fill=green] (-45:2) --
      (-45:3)  arc[start angle=-45,  end angle=-135, radius=3] --
      (-135:2) arc[start angle=-135, end angle=-45,  radius=2];
\draw[fill=yellow] (45:2) --
      (45:3)  arc[start angle=45,  end angle=-45, radius=3] --
      (-45:2) arc[start angle=-45, end angle=45,  radius=2];
\draw[fill=red] (135:2) --
      (135:3)  arc[start angle=135,  end angle=225, radius=3] --
      (225:2) arc[start angle=225, end angle=135,  radius=2];

\draw[fill=blue!80] (0:1) --
      (0:2)  arc[start angle=0,  end angle=90, radius=2] --
      (90:1) arc[start angle=90, end angle=0,  radius=1];
\draw[fill=red] (0:1) --
      (0:2)  arc[start angle=0,  end angle=-90, radius=2] --
      (-90:1) arc[start angle=-90, end angle=0,  radius=1];
\draw[fill=yellow] (90:1) --
      (90:2)  arc[start angle=90,  end angle=180, radius=2] --
      (180:1) arc[start angle=180, end angle=90,  radius=1];
\draw[fill=blue!80] (180:1) --
      (180:2)  arc[start angle=180,  end angle=270, radius=2] --
      (270:1) arc[start angle=270, end angle=180,  radius=1];
\end{scope}
\end{tikzpicture}
\end{center}

\begin{definition}
\textbf{גרף}
הוא קבוצה של 
\textbf{צמתים}
$V$
וקבוצה של 
\textbf{קשתות}
$E$,
כך שכל קשת מחבר בדיוק שני צמתים.

\textbf{גרף מישורי}
הוא גרף בו שתי קשתות לא חותכות אחת את השניה. בגרף מישורי קטע מהמישור התחום על ידי קבוצה של קשתות נקרא 
\textbf{שטח}.

\textbf{צביעה}
של גרף מישורי היא השמה של צבעים לצמתים, כך ששני צמתים המחוברים על ידי קשת צבועים בצבעים שונים.
\end{definition}

מפות וגרפים דואליים ונוח יותר לטפל בבעיות צביעה בגרפים ולא במפות.
\begin{theorem}
נתונה מפה מישורית, ניתן לבנות גרף מישורי כך שעבור כל צביעה של שטחים במפה קיימת צביעה של הצמתים בגרף, ולהיפך.
\end{theorem}

\textbf{הוכחה}
בנו צומת עבור כל שטח במפה ובנו קשת בין שני צמתים או"א קיים גבול בין שני השטחים.
\qed

האיור שלהלן מראה את הגרף המישורי שניתן לבנות מהמפה המישורית שהבאנו לעיל.
\begin{center}
\selectlanguage{english}
\begin{tikzpicture}[scale=.667]

\draw[fill=blue!80] (-3.5,-3.5) rectangle +(7,7);

\draw[fill=green] (0:1) 
  arc [start angle=0,  end angle=360, radius=1];

\draw[fill=green] (45:2) --
      (45:3)  arc[start angle=45,  end angle=135, radius=3] --
      (135:2) arc[start angle=135, end angle=45,  radius=2];
\draw[fill=green] (-45:2) --
      (-45:3)  arc[start angle=-45,  end angle=-135, radius=3] --
      (-135:2) arc[start angle=-135, end angle=-45,  radius=2];
\draw[fill=yellow] (45:2) --
      (45:3)  arc[start angle=45,  end angle=-45, radius=3] --
      (-45:2) arc[start angle=-45, end angle=45,  radius=2];
\draw[fill=red] (135:2) --
      (135:3)  arc[start angle=135,  end angle=225, radius=3] --
      (225:2) arc[start angle=225, end angle=135,  radius=2];

\draw[fill=blue!80] (0:1) --
      (0:2)  arc[start angle=0,  end angle=90, radius=2] --
      (90:1) arc[start angle=90, end angle=0,  radius=1];
\draw[fill=red] (0:1) --
      (0:2)  arc[start angle=0,  end angle=-90, radius=2] --
      (-90:1) arc[start angle=-90, end angle=0,  radius=1];
\draw[fill=yellow] (90:1) --
      (90:2)  arc[start angle=90,  end angle=180, radius=2] --
      (180:1) arc[start angle=180, end angle=90,  radius=1];
\draw[fill=blue!80] (180:1) --
      (180:2)  arc[start angle=180,  end angle=270, radius=2] --
      (270:1) arc[start angle=270, end angle=180,  radius=1];


\foreach \x/\y/\name in {
    0/0/O,
    3/3/Z,
    1/1/E,-1/1/F,-1/-1/G,1/-1/H,
    0/2.5/A,2.5/0/B,0/-2.5/C,-2.5/0/D,
    } {
  \fill (\x,\y) coordinate(\name) circle(4pt);
}

\draw (E) -- (O) -- (F);
\draw (G) -- (O) -- (H);
\draw (E) to [bend right=30] (F) to [bend right=30] (G) 
          to [bend right=30] (H) to [bend right=30] (E);
\draw (A) -- (E) -- (B) -- (H) -- (C) -- (G) -- (D) -- (F);
\draw (A) to [bend right=30] (D) to [bend right=30] (C) 
          to [bend right=30] (B) to [bend right=30] (A);

\draw (F) -- (A) -- (Z) -- (B);
\draw (C) .. controls (3.5,-3.2) .. (Z);
\draw (D) .. controls (-3.5,3.5) .. (Z);

\begin{scope}[xshift=10cm]

\foreach \x/\y/\name in {
    0/0/O,
    3/3/Z,
    1/1/E,-1/1/F,-1/-1/G,1/-1/H,
    0/2.5/A,2.5/0/B,0/-2.5/C,-2.5/0/D,
    } {
  \coordinate(\name) at (\x,\y);
}

\draw (E) -- (O) -- (F);
\draw (G) -- (O) -- (H);
\draw (E) to [bend right=30] (F) to [bend right=30] (G) 
          to [bend right=30] (H) to [bend right=30] (E);
\draw (A) -- (E) -- (B) -- (H) -- (C) -- (G) -- (D) -- (F);
\draw (A) to [bend right=30] (D) to [bend right=30] (C) 
          to [bend right=30] (B) to [bend right=30] (A);

\draw (F) -- (A) -- (Z) -- (B);
\draw (C) .. controls (3.5,-3.2) .. (Z);
\draw (D) .. controls (-3.5,3.5) .. (Z);

\foreach \cl/\x/\y in {
    green/0cm/0cm,
    blue!80/3cm/3cm,
    blue!80/1cm/1cm,
    yellow/-1cm/1cm,
    blue!80/-1cm/-1cm,
    red/1cm/-1cm,
    green/0cm/2.5cm,
    yellow/2.5cm/0cm,
    green/0cm/-2.5cm,
    red/-2.5cm/0cm
    }
  \fill[\cl] (\x,\y) circle (4pt);
\end{scope}

\end{tikzpicture}
\end{center}

ניתן להגביל את עצמנו לגרפים שהשטחים שלו
\textbf{משולשיים}.%
\footnote{%
השטחים הם לא בהכרח 
\textbf{משולשים}
כי הקשתות יכולות להיות עקומות. לפי משפט
\L{F\'{a}ry},
כל גרף מישורי משולשי ניתן להפוך לגרף מישורי שקול עם קשתות ישרות.%
}

האיור השמאלי שלהלן מראה שניתן לצבוע ריבוע עם שני צבעים, אבל אם מתלתים
\L{(triangulate)} 
אותו---ראו איור מרכזי, חייבים להשתמש בארבעה צבעים. היעד הוא להוכיח 
\textbf{שכל}
גרף ניתן לצבוע בחמישה צבעים, כך שאם הדבר אפשרי בגרף המשולשי, הוא אפשרי גם בגרף המקורי, כי מחיקת הקשתות הנוספות לא מקלקל את הצביעה )איור ימני(.
\begin{center}
\selectlanguage{english}
\begin{tikzpicture}[scale=.33]
\draw (-3.5,-3.5) rectangle +(7,7);
\fill[red] (-3.5,-3.5) circle(7pt);
\fill[green] (-3.5,3.5) circle(7pt);
\fill[green] (3.5,-3.5) circle(7pt);
\fill[red] (3.5,3.5) circle(7pt);
\begin{scope}[xshift=11cm]
\draw (-3.5,-3.5) -- (3.5,3.5);
\draw (-3.5,3.5) .. controls (6,6) .. (3.5,-3.5);
\draw (-3.5,-3.5) rectangle +(7,7);
\fill[red] (-3.5,-3.5) circle(7pt);
\fill[green] (-3.5,3.5) circle(7pt);
\fill[yellow] (3.5,-3.5) circle(7pt);
\fill[blue!80] (3.5,3.5) circle(7pt);
\end{scope}
\begin{scope}[xshift=22cm]
\draw (-3.5,-3.5) rectangle +(7,7);
\fill[red] (-3.5,-3.5) circle(7pt);
\fill[green] (-3.5,3.5) circle(7pt);
\fill[yellow] (3.5,-3.5) circle(7pt);
\fill[blue!80] (3.5,3.5) circle(7pt);
\end{scope}
\end{tikzpicture}
\end{center}

\section{הנוסחה של
\L{Euler}}

\begin{theorem}[\L{Euler}]\label{thm.euler}
יהי
$G$
גרף מישורי מקושר עם
$V$
צמתים,
$E$
קשתות ו-%
$F$
שטחים. אזי
$V-E+F=2$.
\end{theorem}

\textbf{הוכחה} 
באינדוקציה על מספר הקשתות. אם מספר הקשתות בגרף מישורי מקושר הוא אפס, קיים רק צומת אחד ושטח אחד, כך ש-%
$1-0+1=2$.

יהי 
$G$
גרף מישורי מקושר עם 
$V$
צמתים, 
$E$
קשתות ו-%
$F$
שטחים, ומחק קשת 
$e$
המחבר את הצמתים
$v_1,v_2$.
יש שני מקרים:

\textbf{מקרה 1}
הגרף מפסיק להיות מקושר. זהה את
$v_1$
עם
$v_2$.
ל-%
$G'$,
הגרף הנוצר, פחות קשתות מ-%
$G$,
הוא שוב גרף מישורי מקושר,
ולכן לפי הנחת האינדוקציה,
$(V-1)-(E-1)+F=2$
כי יש גם צומת אחד פחות. נפשט ונקבל
$V-E+F=2$
עבור
$G$.

\begin{center}
\selectlanguage{english}
\begin{tikzpicture}
\foreach \x/\y in {0/0, 2/0, 1/1.5,4/0,6/0,5/1.5}
  \fill (\x,\y) circle (2pt);
\draw (2,0) -- (1,1.5) -- (0,0) -- (2,0) node[below] {$v_1$} -- node[above] {$e$} (4,0) node[below] {$v_2$} -- (6,0) -- (5,1.5) -- (4,0);
\begin{scope}[xshift=8cm]
\foreach \x/\y in {0/0, 2/0, 1/1.5,4/0,3/1.5}
  \fill (\x,\y) circle (2pt);
\draw (2,0) -- (1,1.5) -- (0,0) -- (2,0) node[below] {$v_1,v_2$} -- (4,0) -- (3,1.5) -- (2,0);
\end{scope}
\end{tikzpicture}
\end{center}

\textbf{מקרה 2}
הגרף נשאר מקושר. לגרף הנוצר
$G'$
פחות קשתות מ-%
$G$,
ולכן לפי הנחת האינדוקציה,
$V-(E-1)+(F-1)=2$
כי מחיקת קשת אחת מאחדת שני שטחים לאחד. נפשט ונקבל
$V-E+F=2$ 
עבור
$G$.

\begin{center}
\selectlanguage{english}
\begin{tikzpicture}
\foreach \x/\y in {0/0, 2/0, 1/1.5,4/0,6/0,5/1.5}
  \fill (\x,\y) circle (2pt);
\draw (2,0) -- (1,1.5) -- (0,0) -- (2,0) -- node[above] {$e$} (4,0) -- (6,0) -- (5,1.5) -- (4,0);
\draw (1,1.5) -- (5,1.5);
\begin{scope}[xshift=8cm]
\foreach \x/\y in {0/0, 2/0, 1/1.5,4/0,6/0,5/1.5}
  \fill (\x,\y) circle (2pt);
\draw (2,0) -- (1,1.5) -- (0,0) -- (2,0);
\draw (4,0) -- (5,1.5) -- (6,0) -- cycle;
\draw (1,1.5) -- (5,1.5);
\end{scope}
\end{tikzpicture}
\end{center}
\qed

\begin{theorem}
יהי
$G$
גרף מישורי מקושר ומתולת. אזי
$E= 3V-6$.
\end{theorem}

למשל, בגרף המישורי בסעיף
\L{~\ref{s.planar}}
יש
$10$
צמתים ו-%
$24= 3\cdot 10-6$
קשתות.

\textbf{הוכחה}
כל שטח חסום על ידי שלוש קשתות, כך ש-%
$E=3F/2$
כי כל קשת נספר פעמיים, פעם אחת לכל שטח שהיא חוסמת. לפי נוסחת
\L{Euler}:
\begin{form}{1}
E&=&V+F-2\\
E&=&V+2E/3-2\\
E&=&3V-6\,.
\end{form}
\vspace*{-5ex}
\qed

\newpage

\begin{theorem}\label{thm.count}
יהי
$G$
גרף מישורי מקושר. אזי
$E\leq 3V-6$.
\end{theorem}

עבור הגרף באיור שלהלן,
$E=8\leq 3\cdot 6 - 6= 12$.

\begin{center}
\selectlanguage{english}
\begin{tikzpicture}
\foreach \x/\y in {0/0, 2/0, 1/1.5,4/0,6/0,5/1.5}
  \fill (\x,\y) circle (2pt);
\draw (2,0) -- (1,1.5) -- (0,0) -- (2,0) -- (4,0) -- (6,0) -- (5,1.5) -- (4,0);
\draw (1,1.5) -- (5,1.5);
\end{tikzpicture}
\end{center}

\textbf{הוכחה}
תלתו את
$G$
כדי לקבל
$G'$.
ב-%
$G'$, $E= 3V-6$
לפי משפט
\L{~\ref{thm.count}}.
כעת, מחק קשתות מ-%
$G'$
כדי לקבל את
$G$.
מספר הצמתים לא משתנה כך ש-%
$E\leq 3V-6$.
\qed

הנה הגרף המתולת שעבורו
$E=3\cdot 6 - 6= 12$.
\begin{center}
\selectlanguage{english}
\begin{tikzpicture}
\foreach \x/\y in {0/0, 2/0, 1/1.5,4/0,6/0,5/1.5}
  \fill (\x,\y) circle (2pt);
\draw (2,0) -- (1,1.5) -- (0,0) -- (2,0) -- (4,0) -- (6,0) -- (5,1.5) -- (4,0);
\draw (1,1.5) -- (5,1.5);
\draw (2,0) -- (5,1.5);
\draw (2,0) .. controls (-1,-1) and (-1,1) .. (1,1.5);
\draw (2,0) .. controls (3,-1) .. (6,0) .. controls (7,2) and (4,2) .. (1,1.5);
\end{tikzpicture}
\end{center}

\section{גרפים שאינם מישוריים}
נסטה מעט מהסיפור כדי להראות איך ניתן להשתמש במשפטים 
\L{~\ref{thm.euler}}
ו-%
\L{~\ref{thm.count}}
כדי להוכיח שגרפים מסויימים אינם מישוריים.

\begin{theorem}
$K_5$,
הגרף השלם עם חמישה צמתים, אינו מישורי.
\end{theorem}

\begin{center}
\selectlanguage{english}
\begin{tikzpicture}[scale=.8]
\node (pentagon) [minimum size=4cm,regular polygon,regular polygon sides=5] at (0,0) {};
\draw (pentagon.corner 1) -- (pentagon.corner 2);
\draw (pentagon.corner 2) -- (pentagon.corner 3);
\draw (pentagon.corner 3) -- (pentagon.corner 4);
\draw (pentagon.corner 4) -- (pentagon.corner 5);
\draw (pentagon.corner 5) -- (pentagon.corner 1);
\draw (pentagon.corner 1) -- (pentagon.corner 3);
\draw (pentagon.corner 1) -- (pentagon.corner 4);
\draw (pentagon.corner 2) -- (pentagon.corner 4);
\draw (pentagon.corner 2) -- (pentagon.corner 5);
\draw (pentagon.corner 3) -- (pentagon.corner 5);

\foreach \corner in {1,2,3,4,5}
  \fill (pentagon.corner \corner) circle (2pt);

\begin{scope}[xshift=8cm]
\node (pentagon) [minimum size=4cm,regular polygon,regular polygon sides=5] at (0,0) {};
\draw (pentagon.corner 1) -- (pentagon.corner 2);
\draw (pentagon.corner 2) -- (pentagon.corner 3);
\draw (pentagon.corner 3) -- (pentagon.corner 4);
\draw (pentagon.corner 4) -- (pentagon.corner 5);
\draw (pentagon.corner 5) -- (pentagon.corner 1);
\draw (pentagon.corner 1) .. controls (-4,1) .. 
      (pentagon.corner 3);
\draw (pentagon.corner 1) .. controls (4,1) ..
      (pentagon.corner 4);
\draw (pentagon.corner 2) -- (pentagon.corner 4);
\draw (pentagon.corner 2) -- (pentagon.corner 5);
\draw (pentagon.corner 3) -- (pentagon.corner 5);

\foreach \corner in {1,2,3,4,5}
  \fill (pentagon.corner \corner) circle (2pt);
\draw (0,-.95) circle(5pt);
\end{scope}
\end{tikzpicture}
\end{center}
\textbf{הוכחה}
עבור
$K_5$, $V=5$
ן-%
$E=10$.
אבל
$10 \not\leq 3\cdot 5 -6=9$.\qed

\newpage

\begin{theorem}
$K_{3,3}$,
הגרף הדו-אזורי עם שלושה צמתים בכל אזור, אינו מישורי.
\end{theorem}

\begin{center}
\selectlanguage{english}
\begin{tikzpicture}[scale=.8]
\foreach \x/\y in {0/0,0/2,0/4,3/0,3/2,3/4}
  \fill (\x,\y) circle (2pt);
\draw (0,0) -- (3,0);
\draw (0,2) -- (3,2);
\draw (0,4) -- (3,4);
\draw (0,0) -- (3,2);
\draw (0,2) -- (3,4);
\draw (0,4) -- (3,0);
\draw (0,0) -- (3,4);
\draw (0,2) -- (3,0);
\draw (0,4) -- (3,2);
\begin{scope}[xshift=8cm]
\foreach \x/\y in {0/0,0/2,0/4,3/0,3/2,3/4}
  \fill (\x,\y) circle (2pt);
\draw (0,4) -- (3,4);
c\draw (0,2) -- (3,4);
\draw (0,4) .. controls (-1,1) .. (3,0);
\draw (0,0) .. controls (-3,5) .. (3,4);
\draw (0,2) -- (3,0);
\draw (0,4) -- (3,2);

\draw[fill=green] (0,0) -- (3,0) -- (3,0) -- (0,2)  -- (3,2) .. controls (4,-1) .. (0,0);
\fill (3,0) circle (2pt);
\draw (1.5,3) circle(5pt);
\end{scope}
\end{tikzpicture}
\end{center}
\textbf{הוכחה}
$V=6$
ו-%
$E=9$.
לפי משפט
\L{~\ref{thm.euler}},
$F=E-V+2=9-6+2=5$.
אבל כל שטח תחום על ידי ארבע קשתות, ולכן
$E=4F/2=(4\cdot 5)/2\neq 9$.\qed

\section{המעלה של הצמתים}

\begin{definition}
$d(v)$,
\textbf{המעלה}
של צומת
$v$,
היא מספר הקשתות הנפגשות ב-%
$v$.
\end{definition}
עבור הגרף בסעיף
\L{~\ref{s.planar}},
קיימים 
$8$
צמתים בתוך הטבעות, כל אחד ממעלה
$5$.
המעלה של השטח החיצוני ושל המעגל הפנימי הוא 
$4$.
לכן:
\[
\sum_{v\in V} d(v) = 5\cdot 8 + 4\cdot 2=48\,.
\]
כדי לקבל את מספר הקשתות בגרף, עלינו לחלק ב-%
$2$
כי כל קשת נספרה פעמיים, פעם אחת עבור כל צומת שהיא נוגעת בו.
על ידי הכללת הטיעונים הללו נקבל:
\begin{theorem}\label{thm.degrees}
יהי
$d_i, i=1,2,3,\ldots,k$
מספרי הצמתים ממעלה
$i$
בגרף מישורי מקושר עם
$V$
צמתים ו-%
$E$ 
קשתות, כאשר
$k$
הוא המעלה הגבוהה ביותר של צומת ב-%
$V$.
אזי:
\[
\sum_{v\in V} d(v) =\sum_{i=1}^{k} i\cdot d_i=2E\,.
\]
\end{theorem}
\vspace*{-6ex}
\begin{theorem}\label{thm.degree5}
יהי
$G$
גרף מישורי מקושר עם
$E$
קשתות ו-%
$V$
צמתים, ויהי
$d_i,i=1,\ldots,k$
מספרי ההצמתים ממעלה
$i$,
כאשר
$k$
הוא המעלה הגבוהה ביותר של צומת ב-%
$V$.
אזי חייב להיות צומת
$v$
ב-%
$V$
כך ש-%
$d(v) \leq 5$.
\end{theorem}
\textbf{הוכחה 1}
ברור שאם יש 
$d_1$
צמתים ממעלה
$1$, $d_2$ 
צמתים ממעלה
$2$, \ldots, $d_k$
צמתים ממעלה
$k$, 
אזי
$V=\sum_{i=1}^{k}d_i$. 
מהמשפטים
\L{~\ref{thm.count}}
ו-%
\L{~\ref{thm.degrees}}:
\[
\sum_{i=1}^{k} i\cdot d_i=2E\leq 2(3V-6) = 6V-12=6\sum_{i=1}^{k} d_i -12\,.
\]
מכאן ש:
\[
\sum_{i=1}^{k} i\cdot d_i \leq 6\sum_{i=1}^{k} d_i -12\,,
\]
ו:
\[
\sum_{i=1}^{k} (6-i)d_i> 12\,.
\]
בגלל ש-%
$12>0$,
ל-%
$i$
אחד לפחות,
$6-i>0$
ועבור 
$i$
זה,
$i<6$. \qed

\textbf{הוכחה 2}
נחשב את 
\textbf{הממוצע}
של המעלות של הצמתים: סכום המעלות לחלק למספר הצמתים:
\[
d_{\textit{\footnotesize avg}}=\frac{\sum_{i=1}^{k} i\cdot d_i}{V}\,.
\]
אבל סכום המעלות הוא פעמיים מספר הקשתות, ולפי משפט
\L{~\ref{thm.count}}
נקבל:
\[
d_{\textit{\footnotesize avg}}=\frac{2E}{V}\leq \frac{6V-12}{V}=6-\frac{6}{V}<6\,.
\]
אם ממוצע המעלות הוא פחות משש, חייב להיות צומת אחד לפחות ממעלה פחות משש.
\qed

עבור הגרף בסעיף
\L{~\ref{s.planar}},
סכום המעלות הוא
$8\cdot 5 + 2\cdot 4=48$.
יש 
$10$
צמתים, כך שממוצע המעלות שלו הוא
$\frac{48}{10}=4.8$
וחייב להיות צומת ממעלה 
$4$
או פחות.

\section{משפט ששת הצבעים}

\begin{theorem}\label{thm.sixcolor}
כל גרף מישורי ניתן לצביעה בששה צבעים.
\end{theorem}
\textbf{הוכחה}
באינדוקציה על מספר הצמתים ב-%
$G$.
אם לגרף ששה צמתים או פחות, ברור שניתן לצבוע את הגרף בששה צבעים.

עבור הצעד האינדוקטיבי, יהי
$G$
גרף מישורי. לפי משפט
\L{~\ref{thm.degree5}}
קיים צומת
$v$
ממעלה חמש או פחות. מחק צומת
$v$
כדי לקבל את הגרף
$G'$.
לפי הנחת האינדוקציה, ניתן לצבוע את
$G'$
עם ששה צבעים, אבל ל-%
$v$
חמישה שכנים לכל היותר שצבועים בחמישה 
צבעים לכל היותר, כך שנשאר צבע ששי שניתן לצבוע בו את
$v$. \qed

\begin{center}
\selectlanguage{english}
\begin{tikzpicture}[scale=.75,minimum size=6mm,inner sep=0pt]
\foreach \name/\color/\theta in
    {A/red/18,B/green/90,C/blue!80/162,D/yellow/234,E/orange/306}
  \node[circle,draw,fill=\color] (\name) at (\theta:3) {};
\node[circle,draw] (O) at (0,0) {};
\node[above right] at (O) {$\bm{v}$};
\foreach \name in {A,B,C,D,E}
  \draw (O) -- (\name);
\foreach \i/\j in {A/B,B/C,D/E}
  \draw (\i) -- (\j);
\draw (B) .. controls (-5,1) .. (D);
\begin{scope}[xshift=9cm]
\foreach \name/\color/\theta in
    {A/red/18,B/green/90,C/blue!80/162,D/yellow/234,E/orange/306}
  \node[circle,draw,fill=\color] (\name) at (\theta:3) {};
\node[circle,draw,fill=brown] (O) at (0,0) {};
\node[above right] at (O) {$\bm{v}$};
\foreach \name in {A,B,C,D,E}
  \draw (O) -- (\name);
\foreach \i/\j in {A/B,B/C,D/E}
  \draw (\i) -- (\j);
\draw (B) .. controls (-5,1) .. (D);
\end{scope}
\end{tikzpicture}
\end{center}


\section{משפט חמשת הצבעים}

\begin{definition}
יהי
$G$
גרף מישורי מקושר צבוע. 
$G'$
הוא
\textbf{שרשרת}
או"א
$G'$
הוא תת-גרף מקסימלי של
$G$
הצבוע בשני צבעים.%
\footnote{%
השרשרת נקראת גם
\textbf{שרשרת \L{Kempe}}
כי היא הוגדרה על ידי
\L{Alfred Kempe}
בהוכחה השגויה שלו למשפט ארבעת הצבעים. ראו סעיף
\L{~\ref{s.kempe}}.}
\end{definition}

\begin{theorem}\label{thm.fivecolor}
כל גרף מישורי 
$G$
ניתן לצבוע בחמישה צבעים.
\end{theorem}

\textbf{הוכחה}
באינדקציה על מספר הצמתים. נכונות המשפט ברורה עבור גרף מישורי עם חמישה צמתים או פחות.

עבור הצעד האינדוקטיבי, יהי
$G$
גרף מישורי. לפי משפט
\L{~\ref{thm.degree5}}
קיים צומת
$v$
ממעלה חמש או פחות. מחק את הצומת
$v$
כדי לקבל את הגרף
$G'$.
לפי הנחת האינדוקציה, ניתן לצבוע את
$G'$
עם חמישה צבעים או פחות. ב-%
$G$,
אם המעלה של
$v$
היא פחות מחמש,
או אם
$v_1,\ldots,v_5$,
השכנים של
$v$,
צבועים עם ארבעה צבעים או פחות, ניתן לצבוע את
$v$
עם הצבע החמישי.

אחרת, הצמתים
$v_1,\ldots,v_5$
צבועים בצבעים שונים ב-%
$G'$. 
\begin{center}
\selectlanguage{english}
\begin{tikzpicture}[scale=.75,minimum size=6mm,inner sep=0pt]
\foreach \name/\color/\theta in
    {A/red/18,B/green/90,C/blue!80/162,D/yellow/234,E/orange/306}
  \node[circle,draw,fill=\color] (\name) at (\theta:3) {};
\node[circle,draw] (O) at (0,0) {};
\node[above right] at (O) {$\bm{v}$};

\node[right,xshift=8pt] at (A) {$\bm{v_3}$};
\node[right,xshift=8pt] at (B) {$\bm{v_2}$};
\node[left,,xshift=-8pt] at (C) {$\bm{v_1}$};
\node[left,,xshift=-8pt] at (D) {$\bm{v_5}$};
\node[right,xshift=8pt] at (E) {$\bm{v_4}$};

\foreach \name in {A,B,C,D,E}
  \draw (O) -- (\name);
  
\node[circle,draw,fill=red]  (X1) at (126:5) {};
\node[circle,draw,fill=blue!80] (X2) at (90:7)  {};
\node[circle,draw,fill=red]  (X3) at (36:10) {};
\node[circle,draw,fill=blue!80] (X4) at (18:12) {};
\node[circle,draw,fill=red]  (X5) at (0:10) {};
\node[circle,draw,fill=blue!80] (X6) at (-18:8) {};
\draw (C)  -- (X1);
\draw (X1) -- (X2);
\draw (X2) -- (X3);
\draw (X3) -- (X4);
\draw (X4) -- (X5);
\draw (X5) -- (X6);
\draw (X6) -- (A);

\node[circle,draw,fill=orange]  (Y1)  at (80:5) {};
\node[circle,draw,fill=green]   (Y2)  at (50:7)  {};
\node[circle,draw,fill=orange]  (Y3A) at (20:8) {};
\node[circle,draw,fill=orange]  (Y3B) at (30:5) {};
\node[circle,draw,fill=green]   (Y4A) at (10:6) {};
\node[circle,draw,fill=yellow]   (Y4B) at (10:8) {};
\node[circle,draw,fill=green]   (Y4C) at (15:10) {};
\node[circle,draw,fill=green]   (Y5)  at (-35:7) {};
\node[circle,draw,fill=yellow]  (Y6A) at (-20:12) {};
\node[circle,draw,fill=orange]  (Y6B) at (-20:4) {};
\draw (B)  -- (Y1);
\draw (Y1) -- (Y2);
\draw (Y2) -- (Y3A);
\draw (Y2) -- (Y3B);
\draw (Y3A) -- (Y4A);
\draw (Y3A) -- (Y4B);
\draw (Y3A) -- (Y4C);
\draw (E)  -- (Y5);
\draw (Y5) -- (Y6A);
\draw (Y5) -- (Y6B);
\draw (A) -- (Y1);
\draw (X2) -- (Y1);
\draw (X1) -- (Y1);
\draw (D) -- (E);
\draw (D) -- (C);
\end{tikzpicture}
\end{center}
נתבונן בצומת
$v_1$
הצבוע בכחול ובצומת 
$v_3$
הצבוע באדום, ונתבונן בשרשרת הכחול-אדום המכילה אותם. על ידי הוספת הצומת 
$v$
והקשתות
$\overline{vv_1},\overline{vv_3}$
לשרשרת, נקבל מסלול סגור
$P$
)המסומן בקו כפול( שמחלק את המישור לשטח "פנימי" ולשטח "חיצוני".
\begin{center}
\selectlanguage{english}
\begin{tikzpicture}[scale=.75,minimum size=6mm,inner sep=0pt]
\foreach \name/\color/\theta in
    {A/red/18,B/green/90,C/blue!80/162,D/yellow/234,E/orange/306}
  \node[circle,draw,fill=\color] (\name) at (\theta:3) {};
\node[circle,draw] (O) at (0,0) {};
\node[above right] at (O) {$\bm{v}$};

\node[right,xshift=8pt] at (A) {$\bm{v_3}$};
\node[right,xshift=8pt] at (B) {$\bm{v_2}$};
\node[left,,xshift=-8pt] at (C) {$\bm{v_1}$};
\node[left,,xshift=-8pt] at (D) {$\bm{v_5}$};
\node[right,xshift=8pt] at (E) {$\bm{v_4}$};

\foreach \name in {A,B,C,D,E}
  \draw (O) -- (\name);
  
\node[circle,draw,fill=red]  (X1) at (126:5) {};
\node[circle,draw,fill=blue!80] (X2) at (90:7)  {};
\node[circle,draw,fill=red]  (X3) at (36:10) {};
\node[circle,draw,fill=blue!80] (X4) at (18:12) {};
\node[circle,draw,fill=red]  (X5) at (0:10) {};
\node[circle,draw,fill=blue!80] (X6) at (-18:8) {};
\draw[thick,double distance=2pt] (C)  -- (X1);
\draw[thick,double distance=2pt] (X1) -- (X2);
\draw[thick,double distance=2pt] (X2) -- (X3);
\draw[thick,double distance=2pt] (X3) -- (X4);
\draw[thick,double distance=2pt] (X4) -- (X5);
\draw[thick,double distance=2pt] (X5) -- (X6);
\draw[thick,double distance=2pt] (X6) -- (A);
\draw[thick,double distance=2pt] (A) -- (O) -- (C);

\node[circle,draw,fill=orange]  (Y1)  at (80:5) {};
\node[circle,draw,fill=green]   (Y2)  at (50:7)  {};
\node[circle,draw,fill=orange]  (Y3A) at (20:8) {};
\node[circle,draw,fill=orange]  (Y3B) at (30:5) {};
\node[circle,draw,fill=green]   (Y4A) at (10:6) {};
\node[circle,draw,fill=yellow]   (Y4B) at (10:8) {};
\node[circle,draw,fill=green]   (Y4C) at (15:10) {};
\node[circle,draw,fill=green]   (Y5)  at (-35:7) {};
\node[circle,draw,fill=yellow]  (Y6A) at (-20:12) {};
\node[circle,draw,fill=orange]  (Y6B) at (-20:4) {};
\draw (B)  -- (Y1);
\draw (Y1) -- (Y2);
\draw (Y2) -- (Y3A);
\draw (Y2) -- (Y3B);
\draw (Y3A) -- (Y4A);
\draw (Y3A) -- (Y4B);
\draw (Y3A) -- (Y4C);
\draw (E)  -- (Y5);
\draw (Y5) -- (Y6A);
\draw (Y5) -- (Y6B);
\draw (A) -- (Y1);
\draw (X2) -- (Y1);
\draw (X1) -- (Y1);
\draw (D) -- (E);
\draw (D) -- (C);
\end{tikzpicture}
\end{center}
כעת נתבונן בצומת
$v_2$
הצבוע ירוק ובצומת
$v_4$
הצבוע כתום. הצמתים הללו 
\textbf{אינם}
יכולים להיות בשרשרת ירוק-כתום אחת, כי 
$v_2$
נמצא 
\textbf{בתוך}
$P$
ו-%
$v_4$
נמצא
\textbf{מחוץ}
ל-%
$P$,
ולכן כל מסלול המחבר אותם חייב לחתוך את
$P$,
הסותר את ההנחה שהגרף מישורי.%
\footnote{%
טענה זו נובעת מה-%
\L{\emph{Jordan curve theorem}},
שהוא ברור באופן אינטואיטיבי, אבל קשה מאוד להוכיח.%
}
בתרשים שלהלן אפשר לראות שתי שרשרות ירוק-כתום המכילות את
$v_2$
ו-%
$v_4$
שאינן מחוברות מסומנות בקו מקווקוו כפול.
\begin{center}
\selectlanguage{english}
\begin{tikzpicture}[scale=.75,minimum size=6mm,inner sep=0pt]
\foreach \name/\color/\theta in
    {A/red/18,B/green/90,C/blue!80/162,D/yellow/234,E/orange/306}
  \node[circle,draw,fill=\color] (\name) at (\theta:3) {};
\node[circle,draw] (O) at (0,0) {};
\node[above right] at (O) {$\bm{v}$};

\node[right,xshift=8pt] at (A) {$\bm{v_3}$};
\node[right,xshift=8pt] at (B) {$\bm{v_2}$};
\node[left,,xshift=-8pt] at (C) {$\bm{v_1}$};
\node[left,,xshift=-8pt] at (D) {$\bm{v_5}$};
\node[right,xshift=8pt] at (E) {$\bm{v_4}$};

\foreach \name in {A,B,C,D,E}
  \draw (O) -- (\name);
  
\node[circle,draw,fill=red]  (X1) at (126:5) {};
\node[circle,draw,fill=blue!80] (X2) at (90:7)  {};
\node[circle,draw,fill=red]  (X3) at (36:10) {};
\node[circle,draw,fill=blue!80] (X4) at (18:12) {};
\node[circle,draw,fill=red]  (X5) at (0:10) {};
\node[circle,draw,fill=blue!80] (X6) at (-18:8) {};

\draw[thick,double distance=2pt] (C)  -- (X1);
\draw[thick,double distance=2pt] (X1) -- (X2);
\draw[thick,double distance=2pt] (X2) -- (X3);
\draw[thick,double distance=2pt] (X3) -- (X4);
\draw[thick,double distance=2pt] (X4) -- (X5);
\draw[thick,double distance=2pt] (X5) -- (X6);
\draw[thick,double distance=2pt] (X6) -- (A);
\draw[thick,double distance=2pt] (A) -- (O) -- (C);

\node[circle,draw,fill=orange]  (Y1)  at (80:5) {};
\node[circle,draw,fill=green]   (Y2)  at (50:7)  {};
\node[circle,draw,fill=orange]  (Y3A) at (20:8) {};
\node[circle,draw,fill=orange]  (Y3B) at (30:5) {};
\node[circle,draw,fill=green]   (Y4A) at (10:6) {};
\node[circle,draw,fill=yellow]   (Y4B) at (10:8) {};
\node[circle,draw,fill=green]   (Y4C) at (15:10) {};
\node[circle,draw,fill=green]   (Y5)  at (-35:7) {};
\node[circle,draw,fill=yellow]  (Y6A) at (-20:12) {};
\node[circle,draw,fill=orange]  (Y6B) at (-20:4) {};
\draw[thick,dashed,double distance=2pt] (B)  -- (O) -- (E);
\draw[thick,dashed,double distance=2pt] (B)  -- (Y1);
\draw[thick,dashed,double distance=2pt] (Y1) -- (Y2);
\draw[thick,dashed,double distance=2pt] (Y2) -- (Y3A);
\draw[thick,dashed,double distance=2pt] (Y2) -- (Y3B);
\draw[thick,dashed,double distance=2pt] (Y3A) -- (Y4A);
\draw[thick,dashed,double distance=2pt] (Y3A) -- (Y4B);
\draw[thick,dashed,double distance=2pt] (Y3A) -- (Y4C);
\draw[thick,dashed,double distance=2pt] (E)  -- (Y5);
\draw[thick,dashed,double distance=2pt] (Y5) -- (Y6B);
\draw (Y5) -- (Y6A);
\draw (A) -- (Y1);
\draw (X2) -- (Y1);
\draw (X1) -- (Y1);
\draw (D) -- (E);
\draw (D) -- (C);
\end{tikzpicture}
\end{center}
\newpage
נחליף ביניהם את שני ההצבעים בשרשרת המכילה את
$v_2$.
זה לא ישנה את העובדה שניתן לצבוע את
$G'$
עם חמישה צבעים.
$v_2$
ו-%
$v_4$
שניהם צבועים בכתום, וניתן לצבוע את
$v$
בירוק כדי לקבל צביעה של
$G$
עם חמישה צבעים.
\qed
\begin{center}
\selectlanguage{english}
\begin{tikzpicture}[scale=.75,minimum size=6mm,inner sep=0pt]
\foreach \name/\color/\theta in
    {A/red/18,B/orange/90,C/blue!80/162,D/yellow/234,E/orange/306}
  \node[circle,draw,fill=\color] (\name) at (\theta:3) {};
\node[circle,draw,fill=green] (O) at (0,0) {};
\node[above right] at (O) {$\bm{v}$};

\node[right,xshift=8pt] at (A) {$\bm{v_3}$};
\node[right,xshift=8pt] at (B) {$\bm{v_2}$};
\node[left,,xshift=-8pt] at (C) {$\bm{v_1}$};
\node[left,,xshift=-8pt] at (D) {$\bm{v_5}$};
\node[right,xshift=8pt] at (E) {$\bm{v_4}$};

\foreach \name in {A,B,C,D,E}
  \draw (O) -- (\name);
  
\node[circle,draw,fill=red]  (X1) at (126:5) {};
\node[circle,draw,fill=blue!80] (X2) at (90:7)  {};
\node[circle,draw,fill=red]  (X3) at (36:10) {};
\node[circle,draw,fill=blue!80] (X4) at (18:12) {};
\node[circle,draw,fill=red]  (X5) at (0:10) {};
\node[circle,draw,fill=blue!80] (X6) at (-18:8) {};

\draw[thick,double distance=2pt] (C)  -- (X1);
\draw[thick,double distance=2pt] (X1) -- (X2);
\draw[thick,double distance=2pt] (X2) -- (X3);
\draw[thick,double distance=2pt] (X3) -- (X4);
\draw[thick,double distance=2pt] (X4) -- (X5);
\draw[thick,double distance=2pt] (X5) -- (X6);
\draw[thick,double distance=2pt] (X6) -- (A);
\draw[thick,double distance=2pt] (A) -- (O) -- (C);

\draw[thick,dashed,double distance=2pt] (B)  -- (O) -- (E);
\draw[thick,dashed,double distance=2pt] (B)  -- (Y1);
\draw[thick,dashed,double distance=2pt] (Y1) -- (Y2);
\draw[thick,dashed,double distance=2pt] (Y2) -- (Y3A);
\draw[thick,dashed,double distance=2pt] (Y2) -- (Y3B);
\draw[thick,dashed,double distance=2pt] (Y3A) -- (Y4A);
\draw[thick,dashed,double distance=2pt] (Y3A) -- (Y4B);
\draw[thick,dashed,double distance=2pt] (Y3A) -- (Y4C);
\draw[thick,dashed,double distance=2pt] (E)  -- (Y5);
\draw[thick,dashed,double distance=2pt] (Y5) -- (Y6B);

\node[circle,draw,fill=green]  (Y1)  at (80:5) {};
\node[circle,draw,fill=orange]   (Y2)  at (50:7)  {};
\node[circle,draw,fill=green]  (Y3A) at (20:8) {};
\node[circle,draw,fill=green]  (Y3B) at (30:5) {};
\node[circle,draw,fill=orange]   (Y4A) at (10:6) {};
\node[circle,draw,fill=yellow]   (Y4B) at (10:8) {};
\node[circle,draw,fill=orange]   (Y4C) at (15:10) {};
\node[circle,draw,fill=green]   (Y5)  at (-35:7) {};
\node[circle,draw,fill=yellow]  (Y6A) at (-20:12) {};
\node[circle,draw,fill=orange]  (Y6B) at (-20:4) {};


\draw (Y5) -- (Y6A);
\draw (A) -- (Y1);
\draw (X2) -- (Y1);
\draw (X1) -- (Y1);
\draw (D) -- (E);
\draw (D) -- (C);
\end{tikzpicture}
\end{center}

%%%%%%%%%%%%%%%%%%%%%%%%%%%%%%%%%%%%%%%%%%%%%%%%%%%%%%%%%%%
%%%%%%%%%%%%%%%%%%%%%%%%%%%%%%%%%%%%%%%%%%%%%%%%%%%%%%%%%%%
%%%%%%%%%%%%%%%%%%%%%%%%%%%%%%%%%%%%%%%%%%%%%%%%%%%%%%%%%%%
%%%%%%%%%%%%%%%%%%%%%%%%%%%%%%%%%%%%%%%%%%%%%%%%%%%%%%%%%%%


\section{ההוכחה השגויה של
\L{Kempe}
למשפט ארבעת הצבעים}
\label{s.kempe}

משפט ארבעת הצבעים הוצג כהשערה ב-%
$1852$.
ב-%
$1879$, 
\L{Alfred B. Kempe}
פרסם הוכחה של המשפט, אבל לאחר אחת-עשר שנים, ב-%
$1890$,
\L{Percy J. Heawood}
מצא שגיאה בהוכחה. למרות זאת, העבודה של
\L{Kempe}
חשובה כי: 
$(1)$
ההוכחה נכונה עבור חמישה צבעים, ו-%
$(2)$
בהוכחה שלו הוא המציא את הרעיונות הבסיסיים ששימשו את
\L{Kenneth Appel}
ו-%
\L{Wolfgang Haken}
בהוכחה הנכונה שלהם שפורסמה ב-%
$1976$.

\textbf{הוכחה}
רוב ההוכחה זהה להוכחה של משפט חמשת הצבעים. המקרה החדש שיש לטפל בו הוא כאשר יש צומת 
$v$
עם חמישה שכנים, כאשר לפי ההנחה האינדוקטיבית ניתן לצבוע אותם בארבעה צבעים לאחר מחיקת הצומת
$v$.

באיור השמאלי שלהלן, קיימים שני צמתים
$v_2,v_5$
הצבועים בכחול. נתבונן עכשיו בשרשרת הכחול-ירוק המכילה את 
$v_2$
ובשרשרת הכחול-צהוב המכילה את
$v_5$.
השרשרת הכחול-ירוק נמצאת מתוך המסלול הסגור המוגדר על ידי השרשרת האדום-צהוב שמכילה את
$v_1,v_3$,
והשרשרת הכחול-צהוב נמצאת בתוך המסלול הסגור המוגדר על ידי השרשרת האדום-ירוק המכילה את
$v_1,v_4$.

נחליף את הצבעים בשרשרת הכחול-ירוק ובשרשרת הכחול-צהוב )איור ימני(. התוצאה היא שהשכנים של
$v$
צבועים בשלושה צבעים אדום, ירוק וצהוב, וניתן לצבוע את
$v$
בכחול.

\begin{center}
\selectlanguage{english}
\begin{tikzpicture}[scale=.6,minimum size=6mm,inner sep=0pt]

% Draw center node and adjacent nodes
\foreach \name/\color/\theta in
    {A/yellow/18,B/blue!80/90,C/red/162,D/blue!80/234,E/green/306}
  \node[circle,draw,fill=\color] (\name) at (\theta:3) {};
\node[circle,draw] (O) at (0,0) {};
\node[above right]     at (O) {$\bm{v}$};

\node[right,xshift=10pt] at (A) {$\bm{v_3}$};
\node[left,xshift=-8pt]  at (B) {$\bm{v_2}$};
\node[left,xshift=-8pt]  at (C) {$\bm{v_1}$};
\node[left,xshift=-8pt]  at (D) {$\bm{v_5}$};
\node[right,xshift=8pt]  at (E) {$\bm{v_4}$};

% Draw red-yellow path
\node[circle,draw,fill=yellow]  (X1) at (126:5) {};
\node[circle,draw,fill=red] (X2) at (45:8)  {};

\draw[thick,double distance=2pt] 
  (C) -- (X1) -- (X2) -- (A) -- (O);
\draw[thick,double distance=6pt] (O) -- (C);

% Draw blue-green nodes within red-yellow path
\node[circle,draw,fill=green] (Y1)  at (50:5) {};

% Draw red-green path
\node[circle,draw,fill=green] (Z1)  at (-160:5) {};
\node[circle,draw,fill=red]   (Z2)  at (-80:6)  {};

%\draw[thick,dashed,double distance=6pt] (O) -- (C);
\draw[thick,dashed,double distance=2pt] 
  (O) -- (C) -- (Z1) -- (Z2) -- (E) -- (O);

% Draw blue-yellow nodes within red-green path
\node[circle,draw,fill=yellow]   (U1)  at (-90:4)  {};

% Connect adjacent nodes not in paths
\draw (X1) -- (B) -- (Y1) -- (A) -- (B) -- 
      (C) -- (D) -- (E) -- (A);
\draw (Z2) -- (U1) -- (D) -- (O) -- (B);

%\end{tikzpicture}
%\end{center}

\begin{scope}[xshift=12cm]
%\begin{center}
%\begin{tikzpicture}[scale=.6,minimum size=6mm,inner sep=0pt]

% Draw center node and adjacent nodes
\foreach \name/\color/\theta in
    {A/yellow/18,B/green/90,C/red/162,D/yellow/234,E/green/306}
  \node[circle,draw,fill=\color] (\name) at (\theta:3) {};
\node[circle,draw,fill=blue!80] (O) at (0,0) {};
\node[above right]     at (O) {$\bm{v}$};

\node[right,xshift=10pt] at (A) {$\bm{v_3}$};
\node[left,xshift=-8pt]  at (B) {$\bm{v_2}$};
\node[left,xshift=-8pt]  at (C) {$\bm{v_1}$};
\node[left,xshift=-8pt]  at (D) {$\bm{v_5}$};
\node[right,xshift=8pt]  at (E) {$\bm{v_4}$};

% Draw red-yellow path
\node[circle,draw,fill=yellow]  (X1) at (126:5) {};
\node[circle,draw,fill=red] (X2) at (45:8)  {};

\draw[thick,double distance=2pt] 
  (C) -- (X1) -- (X2) -- (A) -- (O);
\draw[thick,double distance=6pt] (O) -- (C);

% Draw blue-green nodes within red-yellow path
\node[circle,draw,fill=blue!80] (Y1)  at (50:5) {};

% Draw red-green path
\node[circle,draw,fill=green] (Z1)  at (-160:5) {};
\node[circle,draw,fill=red]   (Z2)  at (-80:6)  {};

\draw[thick,dashed,double distance=2pt] 
  (O) -- (C) -- (Z1) -- (Z2) -- (E) -- (O);

% Draw blue-yellow nodes within red-green path
\node[circle,draw,fill=blue!80]   (U1)  at (-90:4)  {};

% Connect adjacent nodes not in paths
\draw (X1) -- (B) -- (Y1) -- (A) -- (B) -- 
      (C) -- (D) -- (E) -- (A);
\draw (Z2) -- (U1) -- (D) -- (O) -- (B);
\end{scope}
\end{tikzpicture}
\end{center}

\L{Heawood}
שם לב שלמסלולים הסגורים, המוגדרים על ידי השרשראות האדום-צהוב והאדום-ירוק, ייתכן שיש צמתים אדומים משותפים )%
$v_1$
והצומת האדום מתחת ל-%
$v_4$
באיור השמאלי(.
כאשר מחליפים צבעים בשרשראות הכחול-ירוק והכחול-צהוב, יש אפשרות שיהיו צמתים צבועים בכחול הקשורים בקשת )איור ימני(, כך שהצביעה כבר לא חוקית.

\begin{center}
\selectlanguage{english}
\begin{tikzpicture}[scale=.6,minimum size=6mm,inner sep=0pt]

% Draw center node and adjacent nodes
\foreach \name/\color/\theta in
    {A/yellow/18,B/blue!80/90,C/red/162,D/blue!80/234,E/green/306}
  \node[circle,draw,fill=\color] (\name) at (\theta:3) {};
\node[circle,draw] (O) at (0,0) {};
\node[above right]     at (O) {$\bm{v}$};

\node[right,xshift=10pt] at (A) {$\bm{v_3}$};
\node[right,xshift=8pt]  at (B) {$\bm{v_2}$};
\node[above right,xshift=8pt]  at (C) {$\bm{v_1}$};
\node[below right,yshift=-8pt] at (D) {$\bm{v_5}$};
\node[right,xshift=8pt]  at (E) {$\bm{v_4}$};

% Draw red-yellow path
\node[circle,draw,fill=yellow] (X1) at (-170:5) {};
\node[circle,draw,fill=red]    (X2) at (-80:7)  {};

\draw[thick,double distance=2pt] (A) -- (O);
\draw[thick,double distance=6pt] (O) -- (C);
\draw[thick,double distance=2pt] (C) --(X1) -- (X2);
\draw[thick,double distance=2pt,bend right=40] (X2) to (A);

% Draw red-green path
\node[circle,draw,fill=green] (Y1) at (100:6)  {};

\draw[dashed,thick,double distance=2pt] (O) -- (C) -- (Y1);
\draw[dashed,thick,double distance=2pt] 
  (Y1) .. controls (40:10) and (-50:9) .. (X2);
\draw[dashed,thick,double distance=2pt] (X2) -- (E) -- (O);

% Draw adjacent nodes
\draw (X1) -- (D) -- (O) -- (B) -- (Y1) -- (X1);

%\end{tikzpicture}
%\end{center}

\begin{scope}[xshift=12cm]
%\begin{center}
%\begin{tikzpicture}[scale=.6,minimum size=6mm,inner sep=0pt]

% Draw center node and adjacent nodes
\foreach \name/\color/\theta in
    {A/yellow/18,B/green/90,C/red/162,D/yellow/234,E/green/306}
  \node[circle,draw,fill=\color] (\name) at (\theta:3) {};
\node[circle,draw,fill=blue!80] (O) at (0,0) {};
\node[above right]     at (O) {$\bm{v}$};

\node[right,xshift=10pt] at (A) {$\bm{v_3}$};
\node[right,xshift=8pt]  at (B) {$\bm{v_2}$};
\node[above right,xshift=8pt]  at (C) {$\bm{v_1}$};
\node[below right,yshift=-8pt] at (D) {$\bm{v_5}$};
\node[right,xshift=8pt]  at (E) {$\bm{v_4}$};


% Draw red-yellow path
\node[circle,draw,fill=blue!80] (X1) at (-170:5) {};
\node[circle,draw,fill=red]  (X2) at (-80:7)  {};

\draw[thick,double distance=2pt] (A) -- (O);
\draw[thick,double distance=6pt] (O) -- (C);
\draw[thick,double distance=2pt] (C) --(X1) -- (X2);
\draw[thick,double distance=2pt,bend right=40] (X2) to (A);

% Draw red-green path
\node[circle,draw,fill=blue!80] (Y1) at (100:6)  {};

\draw[dashed,thick,double distance=2pt] (O) -- (C) -- (Y1);
\draw[dashed,thick,double distance=2pt] 
  (Y1) .. controls (40:10) and (-50:9) .. (X2);
\draw[dashed,thick,double distance=2pt] (X2) -- (E) -- (O);

% Draw adjacent nodes
\draw (X1) -- (D) -- (O) -- (B) -- (Y1) -- (X1);
\end{scope}
\end{tikzpicture}
\end{center}

\newpage

\nocite{*}
\selectlanguage{english}
\bibliographystyle{plain}
\bibliography{five-he}
\end{document}
