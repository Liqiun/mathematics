% !TeX root = construct.tex

\chapter{How to (Almost) Square a Circle}

\section{Approximations to $\pi$}\label{s.square-intro}

In the nineteenth century it was proved that three constructions are impossible with straightedge and compass: trisecting an angle, duplicating a cube and squaring a circle. Given a line segment defined to have length $1$, the constructible numbers (lengths) are those obtainable from that line segment using the operators $+,-,\times,/,\surd$.

In the constructions in this chapter we need to divide a line segment into three parts; here we show how the division of two lengths can be constructed. Given a line segment of length $1$ and line segments of lengths $a,b$, by similar triangles $1/b=\overline{OD}/a$ so $\overline{OD}=a/b$.

\begin{center}
\begin{tikzpicture}
%\clip (-.6,-1) rectangle (4,3);
\draw[name path=horz] (0,0) coordinate (o) -- (7,0);
\fill (o) circle(1.5pt) node[left] {$O$};
\fill (6,0) circle(1.5pt) coordinate (a) node[below] {$A$};
\draw (0,0) -- (30:5.5);
\fill (30:3) circle(1.5pt) coordinate (c) node[above] {$C$};
\fill (30:5) circle(1.5pt) coordinate (b) node[above] {$B$};
\draw (a) -- (b);
\path[name path=par] (c) -- +($(a)-(b)$);
\path[name intersections={of=par and horz,by=d}];
\fill (d) circle(1.5pt) node[below] {$D$};
\draw (c) -- (d);
\draw[<->] (-.4,.5) -- node[fill=white] {$b$} +(30:5.2);
\path (o) -- node[above] {$1$} (c);
\draw[<->] (0,-.8) -- node[fill=white] {$a$} +(6,0);
\path (o) -- node[below] {$a/b$} (d);
\end{tikzpicture}
\end{center}

To square a circle the length $\sqrt{\pi}$ must be constructed, however, $\pi$ is \emph{transcendental}, meaning that it is not the solution of any algebraic equation.

This chapter brings three constructions of approximations to $\pi$. The following table shows the formulas of the lengths that are constructed, their approximate values, the difference between these values and the value of $\pi$, and the error in meters that results if the approximation is used to compute the circumference of the earth given that its radius is $6378$ km.
\[
\renewcommand{\arraystretch}{2.2}
\begin{array}{|l|c|c|c|c|}
\hline
\textrm{Construction} & \textrm{Formula} &\textrm{Value} & \textrm{Difference} & \textrm{Error (m)}\\\hline
\pi & & 3.14159265359 & - & -\\\hline
\textrm{Kochansky} & \sqrt{\disfrac{40}{3}-2\sqrt{3}}&
  3.14153338705 & 5.932 \times 10^{-5} & 756\\\hline
\textrm{Ramanujan}\; 1 & \disfrac{355}{113} &
  3.14159292035 &2.667  \times 10^{-7}&3.4\\\hline
\textrm{Ramanujan}\; 2 &\left(9^2+\disfrac{19^2}{22}\right)^{1/4}&
  3.14159265258 & 1.007 \times 10^{-9}& 0.013\\\hline
\end{array}
\]
Kochansky's construction from $1685$ can be found in \cite{bold}.

Ramanujan's constructions from $1913$ can be found in \cite{ramanujan1,ramanujan2}.

\newpage


%%%%%%%%%%%%%%%%%%%%%%%%%%%%%%%%%%%%%%%%%%%%%%%%%%%%%%%%%%%%%%%%%%%

\section{Kochansky's construction}

Construct three circles:
\begin{enumerate}
\item Construct a unit circle centered at $O$, let $\overline{AB}$ be a diameter and construct a tangent to the circle at $A$.
\item Construct a unit circle centered at $A$. Its intersection with the first circle is $C$.\footnote{For the second and third circles, the diagram only shows the arc that intersects the previous circle.}
\item Construct a unit circle centered at $C$. Its intersection with the second circle is $D$. 
\end{enumerate}
Construct $\overline{OD}$ and denote its intersection with the tangent by $E$.

From $E$ construct $F,G,H$, each at distance $1$ from the previous point; then $\overline{AH}=3-\overline{EA}$.

Construct $\overline{BH}$.

Claim: $\overline{BH}=\sqrt{\disfrac{40}{3}-2\sqrt{3}}\approx \pi$.
\begin{center}
\begin{tikzpicture}[scale=1]
% Scale at 4

% Coordinates of circle
\coordinate (O) at (0,0);
\coordinate (A) at (0,-4);
\coordinate (B) at (0,4);
\fill (O) circle(2pt) node[above right] {$O$};
\fill (A) circle(2pt) node[below right] {$A$};
\fill (B) circle(2pt) node[above right] {$B$};
\draw (A) rectangle +(12pt,12pt);

% Draw circle and diameter
\node [thick,draw,circle through=(A),name path=circle] at (O) {};
\draw [thick] (A) -- (B);

% Draw tangent at A
\draw[thick,name path=tangent] ($(A)+(-4.5,0)$) -- ($(A)+(10.5,0)$);

% Draw arc centered at A which intersects circle at C
\draw[thick,name path=Aarc] (O)
   arc [start angle=90,end angle=220,radius=4];
\path[name intersections={of=circle and Aarc,by=C}];
\fill (C) circle(2pt) node[left,xshift=-4pt] {$C$};

% Draw arc centered at C which intersects the first arc at D
\draw[thick,name path=Carc] ($(C)+(260:4)$)
   arc [start angle=260,end angle=280,radius=4];
\path[name intersections={of=Carc and Aarc,by=D}];
\fill (D) circle(2pt) node[below left] {$D$};

% Draw O--D which intersects the tangent at E
\draw[thick,name path=OD] (O) -- (D);
\path[name intersections={of=tangent and OD,by=E}];
\fill (E) circle(2pt) node[above left] {$E$};

% Find point H at length 3 from E
\coordinate (F) at ($(E)+(4,0)$);
\fill (F) circle(2pt) node[above right,xshift=4pt] {$F$};
\coordinate (G) at ($(F)+(4,0)$);
\fill (G) circle(2pt) node[above] {$G$};
\coordinate (H) at ($(G)+(4,0)$);
\fill (H) circle(2pt) node[above] {$H$};

% Draw BH of length approximately pi
\draw[thick] (B) -- (H);
\end{tikzpicture}
\end{center}

\newpage

To prove the claim extract the following diagram from the first one. Dashed line segments have been added. Since all the circles are unit circles, it is easy to see that the length of each dashed line segment is $1$. It follows that  $\overline{AOCD}$ is a rhombus so its diagonals are perpendicular to and bisect each other at $K$, so $\overline{AK}=\frac{1}{2}$.
\begin{center}
\begin{tikzpicture}[scale=1.25]
% Scale at 4

\clip (-4.5,-6.5) rectangle +(5,7);
% Coordinates of circle
\coordinate (O) at (0,0);
\coordinate (A) at (0,-4);
\coordinate (B) at (0,4);

% Draw circle
\node [circle through=(A),name path=circle] at (O) {};
\draw ($(O)+(200:4)$) arc [start angle=200,end angle=280,radius=4];

% Draw tangent at A
\draw[thick,name path=tangent] ($(A)+(-4.5,0)$) -- ($(A)+(10.5,0)$);
\draw (A) rectangle +(6pt,6pt);

% Draw arc centered at O which intersects circle at C
\draw[thin,name path=Aarc] (O)
   arc [start angle=90,end angle=220,radius=4];
\path[name intersections={of=circle and Aarc,by=C}];

% Draw arc centered at C which intersects the first arc at D
\draw[thin,name path=Carc] ($(C)+(260:4)$)
   arc [start angle=260,end angle=280,radius=4];
\path[name intersections={of=Carc and Aarc,by=D}];

% Draw O--D which intersects the tangent at E
\draw[thick,name path=OD] (O) -- (D);
\path[name intersections={of=tangent and OD,by=E}];

% Find point H at length 3 from E
\coordinate (F) at ($(E)+(4,0)$);
\coordinate (G) at ($(F)+(4,0)$);
\coordinate (H) at ($(G)+(4,0)$);

% Draw BH of length approximately pi
\draw[thick] (B) -- (H);

\draw[ultra thick,dashed] (A) -- (O) -- (C) -- (D) -- cycle;
\draw[ultra thick,dashed,name path=AC] (A) -- (C);

\node[above left,yshift=10pt,xshift=2pt] at (A) {$60^\circ$};
\node[left,yshift=10pt,xshift=-20pt] at (A) {$30^\circ$};

\path[name intersections={of=AC and OD,by=K}];
\draw[thick,rotate=-30] (K) rectangle +(6pt,6pt);

\path (A) -- node[above,xshift=2pt,yshift=2pt] {$1/2$} (K);
\path (A) -- node[below,xshift=-4pt] {$1/\sqrt{3}$} (E);

\fill (O) circle(1.25pt) node[above right] {$O$};
\fill (A) circle(1.25pt) node[below right] {$A$};
\fill (B) circle(1.25pt) node[above right] {$B$};
\fill (C) circle(1.25pt) node[left,xshift=-4pt] {$C$};
\fill (D) circle(1.25pt) node[below left] {$D$};
\fill (E) circle(1.25pt) node[above left] {$E$};
\fill (F) circle(1.25pt) node[above right,xshift=4pt] {$F$};
\fill (G) circle(1.25pt) node[above] {$G$};
\fill (H) circle(1.25pt) node[above] {$H$};
\fill (K) circle(1.25pt) node[above,yshift=4pt] {$K$};
\end{tikzpicture}
\end{center}
The diagonal $\overline{AC}$ forms two equilateral triangles $\triangle OAC, \triangle DAC$ so $\angle OAC=60^\circ$. Since tangent forms a right angle with the radius $\overline{OA}$, $\angle KAE=30^\circ$. Now:
\begin{displaymath}
\renewcommand{\arraystretch}{1.5}
\begin{array}{lcl}
\disfrac{1/2}{\overline{EA}}&=&
\cos 30^\circ=\disfrac{\sqrt{3}}{2}\\
\overline{EA}&=&\disfrac{1}{\sqrt{3}}\\
\overline{AH}&=&3-\overline{EA}=\left(3-\disfrac{1}{\sqrt{3}}\right)
=\disfrac{3\sqrt{3}-1}{\sqrt{3}}\\
\end{array}
\end{displaymath}

Returning to the first diagram, we see that $\triangle ABH$ is a right triangle:
\begin{displaymath}
\renewcommand{\arraystretch}{2}
\begin{array}{lcl}
\overline{BH}^2&=&\overline{OB}^2+\overline{AH}^2\\
&=&4+\disfrac{9\cdot 3 -6\sqrt{3}+1}{3}=\disfrac{40}{3}-2\sqrt{3}\\
\overline{BH}&=&\sqrt{\disfrac{40}{3}-2\sqrt{3}}\approx 3.141533387=\approx \pi\,.
\end{array}
\end{displaymath}

\newpage


%%%%%%%%%%%%%%%%%%%%%%%%%%%%%%%%%%%%%%%%%%%%%%%%%%%%%%%%%%%%%%%%%%%



\section{Ramanujan's first construction}


\subsection{The construction}

Construct a unit circle centered at $O$ and let $\overline{PR}$ be a diameter. 

$H$ bisects $\overline{PO}$ and $T$ trisects $\overline{RO}$. 

Construct a perpendicular at $T$ that intersects the circle at $Q$.

Construct a chord $\overline{RS}=\overline{QT}$.

Construct $\overline{PS}$.

Construct a line parallel to $RS$ from $T$ that intersects $\overline{PS}$ at $N$.

Construct a line parallel to $\overline{RS}$ from $O$ that intersects $\overline{PS}$ at $M$.

Construct the chord $\overline{PK}=\overline{PM}$.

Construct the tangent at $P$ of length $\overline{PL}=\overline{MN}$.

Connect the points $K,L,R$.

Find point $C$ such that $\overline{RC}$ is equal to $\overline{RH}$.

Construct $\overline{CD}$ parallel to $\overline{KL}$ that intersects $\overline{LR}$ at $D$. 

Claim: $\overline{RD}^2=\disfrac{355}{113}\approx pi$.


\begin{center}
\begin{tikzpicture}[scale=1.1,align=left]
\clip (-6,-5.1) rectangle +(11.5,10.2);
% Draw circle and horizontal diameter
\draw[name path=circle] (0,0)  coordinate (o) node[below] {$O$} circle[radius=5cm];
\draw (-5,0) coordinate (p) node[left] {$P$} -- (5,0) coordinate (r) node[right] {$R$};
\fill (o) circle (1pt);
\fill (p) circle (1pt);
\fill (r) circle (1pt);
\fill (-2.5,0) coordinate (h) node[below] {$H$} circle (1pt);
\fill (10/3,0) coordinate (t) node[below left] {$T$} circle (1pt);
\path (p) -- node[above,xshift=10pt] {$1/2$} (h) -- node[above] {$1/2$} (o) -- node[above] {$2/3$} (t) -- node[above] {$1/3$} (r);

% Draw perpendicular TQ
\path[name path=tq] (t) -- +(0,5);
\path[name intersections={of=tq and circle,by=q}];
\draw (t) -- node[left] {$a$} (q) node[above] {$Q$};
\fill (q) circle (1pt);

% Draw chord RS and line PS
\path[name path=tq] (t) -- +(0,5);
\path[name intersections={of=tq and circle,by=q}];
\path[name path=rcirc] (r) let \p1 = ($ (t) - (q) $) in circle ({veclen(\x1,\y1)});
\path[name intersections={of=rcirc and circle,by=s}];
\draw (r) -- node[right] {$a$} (s);
\fill (s) node[above right] {$S$} circle (1pt);
\draw[name path=ps] (p) -- (s);

% Draw TN
\path[name path=tn] (t) -- +($(s)-(r)$);
\path[name intersections={of=ps and tn,by=n}];
\draw (t) -- (n);
\fill (n) node[above] {$N$} circle (1pt);

% Draw OM
\path[name path=om] (o) -- +($(s)-(r)$);
\path[name intersections={of=ps and om,by=m}];
\draw (o) -- (m);
\fill (m) node[above left] {$M$} circle (1pt);
\path (p) -- (m);
\path (m) -- (n);

% Draw chord PK
\draw (p) -- +(-62.3:4.64) coordinate (k) node[below left] {$K$};

% Draw tangent PL
\draw let \p1 = ($ (m) - (n) $), \n1 = {veclen(\x1,\y1)} in (p) -- (-5,-\n1) coordinate (l) node[left] {$L$};

% Connect L and K to R
\draw (r) -- (l) -- (k) -- cycle;

% Find point C on RK
\coordinate (c) at ($(r)!7.5cm!(k)$);
\path (r) -- (c);
\fill (c) node[below] {$C$} circle (1pt);

% Draw CD
\path[name path=cd] (c) -- +($(l)-(k)$);
\path[name path=lr] (l) -- (r);
\path[name intersections={of=cd and lr,by=d}];
\draw (c) -- (d);
\fill (d) node[above,xshift=2pt] {$D$} circle (1pt);
\path (r) -- (d);

\draw[rotate=90] (t) rectangle +(8pt,8pt);
\draw[rotate=-160] (s) rectangle +(8pt,8pt);
\end{tikzpicture}
\end{center}

\newpage

\subsection{The computation}

By Pythagoras' theorem on $\triangle QOT$:
\[
\overline{QT} = \sqrt{1^2-\left(\frac{2}{3}\right)^2}=\frac{\sqrt{5}}{3}\,.
\]
$\triangle PSR$ is a right triangle because it subtends a diameter. By Pythagoras theorem:
\[
\overline{PS} = \sqrt{2^2-\left(\frac{\sqrt{5}}{3}\right)^2}=\sqrt{4-\frac{5}{9}}=\frac{\sqrt{31}}{3}\,.
\]
$\triangle MPO\sim \triangle SPR$ so:
\begin{form}{2.2}
\disfrac{\overline{PM}}{\overline{PO}}&=&\disfrac{\overline{PS}}{\overline{PR}}\\
\disfrac{\overline{PM}}{1}&=&\disfrac{\sqrt{31}/3}{2}\\
\overline{PM}&=&\disfrac{\sqrt{31}}{6}\,.
\end{form}
$\triangle NPT\sim \triangle SPR$ so:

\begin{form}{2.2}
\disfrac{\overline{PN}}{\overline{PT}}&=&\disfrac{\overline{PS}}{\overline{PR}}\\
\disfrac{\overline{PN}}{5/3}&=&\disfrac{\sqrt{31}/3}{2}\\
\overline{PN}&=&\disfrac{5\sqrt{31}}{18}\\
\overline{MN}&=&\overline{PN}-\overline{PM}\\
&=&\sqrt{31}\left(\disfrac{5}{18}-\disfrac{1}{6}\right) = \disfrac{\sqrt{31}}{9}\,.
\end{form}

$\triangle PKR$ is a right triangle because it subtends a diameter. By Pythagoras's theorem:
\[
\overline{RK}=\sqrt{2^2-\left(\frac{\sqrt{31}}{6}\right)^2} = \frac{\sqrt{113}}{6}\,.
\]

$\triangle PLR$ is a right triangle because it $\overline{PL}$ is a tangent. By Pythagoras's theorem:
\[
\overline{RL}=\sqrt{2^2+\left(\frac{\sqrt{31}}{9}\right)^2} = \frac{\sqrt{355}}{9}\,.
\]

\newpage

$\overline{RC}=\overline{RH}=\displaystyle\frac{1}{3}+\frac{2}{3}+\frac{1}{2}=\frac{3}{2}$. Since $\overline{CD}$ is parallel to $\overline{LK}$, by similar triangles:
\begin{form}{2.2}
\disfrac{\overline{RD}}{\overline{RC}}&=&\disfrac{\overline{RL}}{\overline{RK}}\\
\disfrac{\overline{RD}}{3/2}&=&\disfrac{\sqrt{355}/9}{\sqrt{113}/6}\\
\overline{RD}&=&\sqrt{\disfrac{355}{113}}\,.
\end{form}

Here is the construction with line segments labeled with their lengths:

\begin{center}
\begin{tikzpicture}[scale=1.1,align=left]
\clip (-6,-5.1) rectangle +(11.5,10.2);
% Draw circle and horizontal diameter
\draw[name path=circle] (0,0)  coordinate (o) node[below] {$O$} circle[radius=5cm];
\draw (-5,0) coordinate (p) node[left] {$P$} -- (5,0) coordinate (r) node[right] {$R$};
\fill (o) circle (1pt);
\fill (p) circle (1pt);
\fill (r) circle (1pt);
\fill (-2.5,0) coordinate (h) node[below] {$H$} circle (1pt);
\fill (10/3,0) coordinate (t) node[below] {$T$} circle (1pt);
\path (p) -- node[above,xshift=10pt] {$1/2$} (h) -- node[above] {$1/2$} (o) -- node[above] {$2/3$} (t) -- node[above] {$1/3$} (r);
% Draw chord RS and line PS
\path[name path=tq] (t) -- +(0,5);
\path[name intersections={of=tq and circle,by=q}];
\path[name path=rcirc] (r) let \p1 = ($ (t) - (q) $) in circle ({veclen(\x1,\y1)});
\path[name intersections={of=rcirc and circle,by=s}];
\draw (r) -- node[left] {$\sqrt{5}/3$} (s);
\fill (s) node[above right] {$S$} circle (1pt);
\draw[name path=ps] (p) -- node[above right,yshift=16pt] {$\sqrt{31}/3$} (s);
% Draw TN
\path[name path=tn] (t) -- +($(s)-(r)$);
\path[name intersections={of=ps and tn,by=n}];
\draw (t) -- (n);
\fill (n) node[above] {$N$} circle (1pt);
% Draw OM
\path[name path=om] (o) -- +($(s)-(r)$);
\path[name intersections={of=ps and om,by=m}];
\draw (o) -- (m);
\fill (m) node[above left] {$M$} circle (1pt);
\path (p) -- node[below,xshift=32pt,yshift=12pt] {$\sqrt{31}/6$} (m);
\path (m) -- node[below,xshift=8pt,yshift=-6pt] {$\sqrt{31}/9$} (n);
% Draw chord PK
\draw (p) -- node[right,xshift=-2pt,yshift=10pt] {$\sqrt{31}/6$} +(-62.3:4.64) coordinate (k) node[below left] {$K$};
% Draw tangent PL
\draw let \p1 = ($ (m) - (n) $), \n1 = {veclen(\x1,\y1)} in (p) -- node[left] {$\disfrac{\sqrt{31}}{9}$} (-5,-\n1) coordinate (l) node[left] {$L$};
% Connect L and K to R
\draw (r) -- (l) -- (k) -- cycle;
% Find point C on RK
\coordinate (c) at ($(r)!7.5cm!(k)$);
%\path (r) -- node[below,yshift=-16pt] {$RC=3/2$\\$RK=\sqrt{113}/6$} (c);
\path (r) -- node[below,xshift=16pt,yshift=-2pt] {$\overline{RC}=3/2$} (c);
\path (r) -- node[below,xshift=-4pt,yshift=-20pt] {$\overline{RK}=\sqrt{113}/6$} (k);
\fill (c) node[below] {$C$} circle (1pt);
% Draw CD
\path[name path=cd] (c) -- +($(l)-(k)$);
\path[name path=lr] (l) -- (r);
\path[name intersections={of=cd and lr,by=d}];
\draw (c) -- (d);
\fill (d) node[above,xshift=2pt] {$D$} circle (1pt);
%\path (r) -- node[above,xshift=-40pt,yshift=-8pt] {$RD=\sqrt{355/113}$\\$RL=\sqrt{355}/9$} (d);
\path (r) -- node[above,xshift=-40pt,yshift=-4pt] {$\overline{RD}=\sqrt{355/113}$} (d);
\path (r) -- node[below,xshift=-28pt,yshift=-15pt] {$\overline{RL}=\sqrt{355}/9$} (l);
\end{tikzpicture}
\end{center}

\bigskip

The value $\disfrac{355}{113}$ could be constructed by constructing two line segments of length $355$ and $113$ and then using the division construction shown in Section~\ref{s.square-intro}, but that is rather tedious!

\newpage

%%%%%%%%%%%%%%%%%%%%%%%%%%%%%%%%%%%%%%%%%%%%%%%%%%%%%%%%%%%%

\section{Ramanujan's second approximation}

\subsection{The construction}

Construct a unit circle centered at $O$ with diameter $\overline{AB}$, and let $C$ be the intersection of the perpendicular at $O$ with the circle.

Trisect $\overline{AO}$ so that $\overline{AT}=1/3$ and $\overline{TO}=2/3$.

Construct $\overline{BC}$ and find points $M,N$ such that $\overline{CM}=\overline{MN}=\overline{AT}=1/3$.

Construct $\overline{AM}$ and $\overline{AN}$ and let $P$ be the point on $\overline{AN}$ such that $\overline{AP}=\overline{AM}$.

From $P$ construct a line parallel to $\overline{MN}$ that intersects $\overline{AM}$ at $Q$.

Construct $\overline{OQ}$ and then from $T$ construct a line parallel to $\overline{OQ}$ that intersects $\overline{AM}$ at $R$.

Construct a line segment $\overline{AS}$ tangent to $A$ of length $\overline{AR}$.

Construct $\overline{SO}$.

Claim: $3\sqrt{\overline{SO}}=\left(9^2+\disfrac{19^2}{22}\right)^{\frac{1}{4}}\approx \pi$.

\begin{center}
\begin{tikzpicture}[scale=1.4]
\clip (-4.4,-4.2) rectangle +(8.8,8.6);
% Scale at 4

% Coordinates of circle
\coordinate (O) at (0,0);
\coordinate (A) at (-4,0);
\coordinate (B) at (4,0);
\coordinate (C) at (0,4);

% Draw circle and diameter
\node [thick,draw,circle through=(A),name path=circle] at (O) {};
\draw [thick] (A) -- (B);
\draw [thick,dashed] (C) -- (O);
\draw (O) rectangle +(8pt,8pt);
\draw[rotate=-90] (A) rectangle +(8pt,8pt);

\coordinate (T) at (-2.667,0);
\path (A) -- node[below] {$1/3$} (T);
\path (T) -- node[below] {$2/3$} (O);
\path (O) -- node[below] {$1$} (B);

\draw (C) -- node[right] {$1/3$} +(-45:1.333) coordinate (M);
\draw (M) -- node[right] {$1/3$} +(-45:1.333) coordinate (N);
\draw (N) -- node[near start,right] {$\sqrt{2}\!-\!2/3$}(B);

\draw[name path=AM] (A) -- (M);
\draw[name path=AN] (A) -- (N);

\node [circle through=(M),name path=AMcircle] at (A) {};

\path[name intersections={of=AMcircle and AN,by=P}];

\path[name path=PQ] (P) -- +(135:2);
\path[name intersections={of=PQ and AM,by=Q}];
\draw (P) -- (Q) -- (O);

\path[name path=QT] (T) -- ($(Q)+(-2.667,0)$) -- (Q);
\path[name intersections={of=QT and AM,by=R}];
\draw (T) -- (R);

\node [circle through=(R),name path=ARcircle] at (A) {};
\path[name path=AS] (A) -- ($(A)+(0,-2.5)$);
\path[name intersections={of=ARcircle and AS,by=S}];
\draw (A) -- (S);

\draw (S) -- (O);

\fill (O) circle(1.2pt) node[below right] {$O$};
\fill (A) circle(1.2pt) node[left] {$A$};
\fill (B) circle(1.2pt) node[right] {$B$} node[above left,xshift=-8pt] {$45^\circ$};
\fill (C) circle(1.2pt) node[above] {$C$};
\fill (T) circle(1.2pt) node[below] {$T$};
\fill (M) circle(1.2pt) node[right] {$M$};
\fill (N) circle(1.2pt) node[right] {$N$};
\fill (P) circle(1.2pt) node[below] {$P$};
\fill (Q) circle(1.2pt) node[above left] {$Q$};
\fill (R) circle(1.2pt) node[above left] {$R$};
\fill (S) circle(1.2pt) node[above left] {$S$};

\end{tikzpicture}
\end{center}

\newpage

\subsection{Proof}

$\triangle COB$ is a right triangle, $\overline{OB}=\overline{OC}=1$, 	so by Pythagoras' theorem $\overline{CB}=\sqrt{2}$ and $\overline{NB}=\sqrt{2}-2/3$. The triangle is isoceles so $\angle NBA =\angle MBA=45^\circ$.

We use the law of cosines on $\triangle NBA$ to compute $\overline{AN}$:
\begin{form}{2}
\overline{AN}^2&=&\overline{BA}^2 + \overline{BN}^2-2\cdot\overline{BA}\cdot\overline{BN}\cdot\cos \angle NBA\\
&=&2^2+\left(\sqrt{2}-\disfrac{2}{3}\right)^2-2\cdot 2 \cdot \left(\sqrt{2}-\disfrac{2}{3}\right)\cdot \disfrac{\sqrt{2}}{2}\\
&=&\left(4+2+\disfrac{4}{9}-4\right) + \sqrt{2}\cdot \left(-\disfrac{4}{3}+\disfrac{4}{3}\right)=\disfrac{22}{9}\\
\overline{AN}&=&\sqrt{\disfrac{22}{9}}\,.
\end{form}

Similarly, we use the law of cosines on $\triangle MBA$ to compute $\overline{AM}$:
\begin{form}{2}
\overline{AM}^2&=&\overline{BA}^2 + \overline{BM}^2-2\cdot\overline{BA}\cdot\overline{BM}\cdot\cos \angle MBA\\
&=&2^2+\left(\sqrt{2}-\disfrac{1}{3}\right)^2-2\cdot 2 \cdot \left(\sqrt{2}-\disfrac{1}{3}\right)\cdot \disfrac{\sqrt{2}}{2}\\
&=&\left(4+2+\disfrac{1}{9}-4\right) + \sqrt{2}\cdot \left(-\disfrac{2}{3}+\disfrac{2}{3}\right)=\disfrac{19}{9}\\
\overline{AM}&=&\sqrt{\disfrac{19}{9}}\,.
\end{form}

By construction $\overline{QP}\parallel \overline{MN}$ so
$\triangle MAN\sim \triangle QAP$, and by construction $\overline{AP}=\overline{AM}$ giving:
\begin{form}{2.4}
\disfrac{\overline{AQ}}{\overline{AM}}&=&\disfrac{\overline{AP}}{\overline{AN}}=\disfrac{\overline{AM}}{\overline{AN}}\\
\overline{AQ}&=&\disfrac{\overline{AM}^2}{\overline{AN}}=\disfrac{19/9}{\sqrt{22/9}}=\disfrac{19}{3\sqrt{22}}\,.
\end{form}

By construction $\overline{TR}\parallel \overline{OQ}$ so
$\triangle RAT\sim \triangle QAO$ giving:
\begin{form}{2.4}
\disfrac{\overline{AR}}{\overline{AQ}}&=&\disfrac{\overline{AT}}{\overline{AO}}\\
\overline{AR}&=&\overline{AQ}\cdot\disfrac{\overline{AT}}{\overline{AO}}=\disfrac{19}{3\sqrt{22}}\cdot\disfrac{1/3}{1}=\disfrac{19}{9\sqrt{22}}\,.
\end{form}

\newpage

By construction $\overline{AS}=\overline{AR}$ and $\triangle OAS$ is a right triangle. By Pythagoras' theorem:
\begin{form}{3}
\overline{SO}&=&\sqrt{1^2+\left(\disfrac{19}{9\sqrt{22}}\right)^2}\\
3\sqrt{\overline{SO}}&=&3\left(1+\disfrac{19^2}{9^2\cdot 22}\right)^\frac{1}{4}\\
&=&\left(3^4+\disfrac{3^4\cdot 19^2}{9^2\cdot 22}\right)^\frac{1}{4}\\
&=&\left(9^2+\disfrac{19^2}{22}\right)^\frac{1}{4}\approx 3.14159265262\approx \pi\,.
\end{form}

Here is the construction with line segments labeled with their lengths:


\begin{center}
\begin{tikzpicture}[scale=1.4]
\clip (-5,-4.2) rectangle +(10,8.5);
% Scale at 4

% Coordinates of circle
\coordinate (O) at (0,0);
\coordinate (A) at (-4,0);
\coordinate (B) at (4,0);
\coordinate (C) at (0,4);

% Draw circle and diameter
\node [thick,draw,circle through=(A),name path=circle] at (O) {};
\draw [thick] (A) -- (B);
\draw [thick,dashed] (C) -- (O);
\draw (O) rectangle +(8pt,8pt);
\draw[rotate=-90] (A) rectangle +(8pt,8pt);

\coordinate (T) at (-2.667,0);
\path (A) -- node[below] {$1/3$} (T);
\path (T) -- node[below] {$2/3$} (O);
\path (O) -- node[below] {$1$} (B);

\draw (C) -- node[right] {$1/3$} +(-45:1.333) coordinate (M);
\draw (M) -- node[right] {$1/3$} +(-45:1.333) coordinate (N);
\draw (N) -- node[near start,right] {$\sqrt{2}\!-\!2/3$}(B);

\draw[name path=AM] (A) -- node[above,xshift=15pt,yshift=24pt] {$\overline{AM}=\sqrt{19/9}$} (M);
\draw[name path=AN] (A) -- node[below,yshift=-10pt] {$\overline{AN}=\sqrt{22/9}$} (N);

\node [circle through=(M),name path=AMcircle] at (A) {};

\path[name intersections={of=AMcircle and AN,by=P}];

\path[name path=PQ] (P) -- +(135:2);
\path[name intersections={of=PQ and AM,by=Q}];
\draw (P) -- (Q) -- (O);
\path (A) -- node[above,xshift=-14pt,yshift=10pt] {$\overline{AQ}=19/3\sqrt{22}$} (Q);

\path[name path=QT] (T) -- ($(Q)+(-2.667,0)$) -- (Q);
\path[name intersections={of=QT and AM,by=R}];
\draw (T) -- (R);
\path (A) -- node[above,xshift=-32pt,yshift=6pt] {$\overline{AR}=19/9\sqrt{22}$} (R);

\node [circle through=(R),name path=ARcircle] at (A) {};
\path[name path=AS] (A) -- ($(A)+(0,-2.5)$);
\path[name intersections={of=ARcircle and AS,by=S}];
\draw (A) -- node[xshift=10pt,yshift=6pt] {$\overline{AS}=\,19/9\sqrt{22}$} (S);

\draw (S) -- node[right,xshift=-2pt,yshift=-10pt] {$\overline{SO}=\sqrt{1+
  \left(\disfrac{19^2}{9^2\cdot 22}\right)}$} (O);

\fill (O) circle(1.2pt) node[below right] {$O$};
\fill (A) circle(1.2pt) node[left] {$A$};
\fill (B) circle(1.2pt) node[right] {$B$} node[above left,xshift=-8pt] {$45^\circ$};
\fill (C) circle(1.2pt) node[above] {$C$};
\fill (T) circle(1.2pt) node[below] {$T$};
\fill (M) circle(1.2pt) node[right] {$M$};
\fill (N) circle(1.2pt) node[right] {$N$};
\fill (P) circle(1.2pt) node[below] {$P$};
\fill (Q) circle(1.2pt) node[above left] {$Q$};
\fill (R) circle(1.2pt) node[above left] {$R$};
\fill (S) circle(1.2pt) node[above left] {$S$};

\end{tikzpicture}
\end{center}
