\documentclass[11pt,a4paper]{article}
\usepackage{mathpazo}
\usepackage[utf8x]{inputenc}

\usepackage{url}

\textwidth=15cm
\textheight=23cm
\topmargin=0pt
\headheight=0pt
\oddsidemargin=2em
\headsep=0pt
\parindent=0pt

\begin{document}
\thispagestyle{empty}
\begin{center}
\textbf{\LARGE $\mathbf{\sqrt{x+5}=5-x^2}$}

\bigskip

\textbf{\Large Moti Ben-Ari\\\bigskip\url{http://www.weizmann.ac.il/sci-tea/benari/}}

\bigskip
\end{center}


\begin{footnotesize}
\copyright{}\  2017 by Moti Ben-Ari. This work is licensed under the Creative Commons Attribution-ShareAlike 3.0 Unported License.
\end{footnotesize}

\bigskip

Solve for $x$. Square both sides of the equation and collect terms:
\[
f(x) = x^4 - 10 x^2 - x + 20 = 0\,.
\]
Does the polynomial of degree four factor as two polynomials of degree two with integer coefficients? If so, \textit{the coefficients of the $x$ terms must be equal and of opposite signs}, since there is no $x^3$ term! Let $n$ be a positive integer and $k_1,k_2$ be any integers:
\[
f(x) = (x^2 - nx + k_1)\, (x^2 + nx + k_2)\,.
\]
Carry out the multiplication:
\[
\renewcommand{\arraystretch}{1.2}
\begin{array}{rrrrrr}
f(x) = &x^4 & + nx^3 & + k_2 x^2\\
&& -nx^3 &- n^2x^2 &-nk_2x\\
&&&+k_1x^2 &+ nk_1x &+ k_1k_2\,.
\end{array}
\]
Equating the coefficients results in three equations in three unknowns:
\begin{eqnarray*}
(k_1+k_2)-n^2 &=& -10\\
n(k_1-k_2) &=& -1\\
k_1k_2 &=& 20\,.
\end{eqnarray*}
From the last two equations and the choice of $n$ as a positive integer, it is clear that:
\[
k_1=4,\,k_2=5  \;\;\textrm{or} \;\; k_1=-5,\, k_2=-4\,.
\]
Only $k_1=-5,\, k_2=-4$ satisfy the first equation for the coefficient of the $x^2$ term:
\[
f(x) = (x^2 - x - 5)\, (x^2 + x - 4)\,.
\]
$f(x)$ can be zero if either factor is zero. Solving the quadratic equations gives four possible solutions:
\[
\frac{1\pm\sqrt{21}}{2}  \;\;,\;\; \frac{-1\pm\sqrt{17}}{2} \,.
\]
Because of the square root in $\sqrt{x+5}$, we have $5-x^2\geq 0$ and:
\[
-2.24\approx-\sqrt{5}\leq x \leq \sqrt{5}\approx 2.24
\]
By numerically computing the roots, we see that there are only two solutions:
\[
\frac{1-\sqrt{21}}{2} \approx -1.79 \;\;,\;\; \frac{-1+\sqrt{17}}{2}\approx 1.56 \,.
\]

\end{document}