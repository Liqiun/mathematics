\documentclass[12pt,a4paper]{article}
\usepackage[utf8x]{inputenc}
\usepackage[english,hebrew]{babel}
\usepackage{verbatim}
\usepackage{url}

\textwidth=15cm
\textheight=23cm
\topmargin=0pt
\headheight=0pt
\oddsidemargin=2em
\headsep=0pt
\parindent=0pt

\begin{document}
\thispagestyle{empty}

\begin{center}
\textbf{\LARGE $\mathbf{\sqrt{x+5}=5-x^2}$}

\bigskip

\selectlanguage{hebrew}

\textbf{\Large מוטי בן-ארי}

\bigskip

\url{http://www.weizmann.ac.il/sci-tea/benari/}

\bigskip

יצירה זו מופצת תחת רישיון ייחוס-שיתוף זהה
$3.0$
לא מותאם של
\L{CreativeCommons}.
\end{center}

\bigskip
\bigskip

פתור עבור
$x$.
חשב את הריבוע של שני צדי המשוואה ואסוף איברים:
\[
f(x) = x^4 - 10 x^2 - x + 20 = 0\,.
\]
האם אפשר לפרק את הפולינום ממעלה ארבע לשני פולינומים ממעלה שניים עם מקדמים שלמים? אם כן,
\textbf{
המקדמים של האיברים
$x$
חייבים להיות בעלי ערך שווה וסימנים הפוכים כי אין איבר
$x^3$!
}
יהי
$n$
מספר שלם חיובי, ו-
$k_1,k_2$
מספרים שלמים כלשהם:
\[
f(x) = (x^2 - nx + k_1)\, (x^2 + nx + k_2)\,.
\]
נכפיל את הפולינומים:
\[
\renewcommand{\arraystretch}{1.2}
\begin{array}{rrrrrr}
f(x) = &x^4 & + nx^3 & + k_2 x^2\\
&& -nx^3 &- n^2x^2 &-nk_2x\\
&&&+k_1x^2 &+ nk_1x &+ k_1k_2\,.
\end{array}
\]
ונשווה מקדמים. נקבל שלוש משוואות בשלושה נעלמים:
\begin{eqnarray*}
(k_1+k_2)-n^2 &=& -10\\
n(k_1-k_2) &=& -1\\
k_1k_2 &=& 20\,.
\end{eqnarray*}
משתי המשוואות האחרונות ומהבחירה של
$n$
כמספר שלם חיובי, ברור ש:
\[
k_1=4,\,k_2=5  \;\;\textrm{or} \;\; k_1=-5,\, k_2=-4\,.
\]
רק 
$k_1=-5,\, k_2=-4$
מספקים את המשוואה הראשונה עבור המקדם של
$x^2$:
\[
f(x) = (x^2 - x - 5)\, (x^2 + x - 4)\,.
\]
הפוקציה
$f(x)$
שווה לאפס אם אחד הגורמים שווה לאפס. נפתור את שני המשוואות הריבועיות ונקבל ארבעה פתרונות אפשריים:
\[
\frac{1\pm\sqrt{21}}{2}  \;\;,\;\; \frac{-1\pm\sqrt{17}}{2} \,.
\]
בגלל השורש ב-
$\sqrt{x+5}$,
מתקיים
$5-x^2\geq 0$,
ולכן:
\[
-2.24\approx-\sqrt{5}\leq x \leq \sqrt{5}\approx 2.24\,.
\]
מחישוב נומרי מתקבלים שני פתרונות:
\[
\frac{1-\sqrt{21}}{2} \approx -1.79 \;\;,\;\; \frac{-1+\sqrt{17}}{2}\approx 1.56 \,.
\]

\end{document}