\documentclass[12pt,a4paper]{article}
\usepackage[utf8x]{inputenc}
\usepackage[english,hebrew]{babel}
\selectlanguage{hebrew}
\usepackage{graphicx}
\usepackage{verbatim}
\usepackage{url}
\newcommand*{\p}[1]{\textsf{\small #1}}

\graphicspath{{images/}}

\usepackage{tikz}
\usetikzlibrary{external,intersections}
\tikzexternalize[prefix=tikz/]

% Use stealth arrows
\tikzset {
  >=stealth
}

\textwidth=15cm
\textheight=23cm
\topmargin=0pt
\headheight=0pt
\oddsidemargin=2em
\headsep=0pt
\parindent=0pt
\setlength{\parskip}{2.5mm plus 4mm minus 3mm}

\begin{document}
\thispagestyle{empty}

\begin{center}
\textbf{\LARGE גרפים כאמצעי עזר לפתרון בעיות תנועה והספק}

\bigskip
\bigskip

\textbf{\Large מוטי בן-ארי}

\bigskip

\textbf{\large המחלקה להוראת המדעים}


\textbf{\large מכון ויצמן למדע}

\end{center}

\section*{מבוא}

אחת הדרכים המקובלות לפתרון של בעיות תנועה היא לצייר תרשים חד-ממדי: קו אופקי עבור מסלול התנועה של כל דמות )מכונית, סירה, וכדומה( המשתתפת בתסריט הבעיה. מימד המרחק מוצג באופן מפורש, אבל מימד הזמן מוצג באופן מובלע כהתקדמות לאורך הקו. מאמר זה מציע להסתייע בגרפים כדי שמימד הזמן יוצג באופן מפורש.


הציר האופקי בגרף הוא ציר הזמן והציר האנכי הוא ציר המרחק. לכל דמות נצייר מסלול המורכב ממקטעים לפי ניסוח הבעיה. המהירות קבועה בכל מקטע ולכן כל מקטע יהיה קטע קו ששיפועו הוא המהירות. ככל שהמהירות גבוהה יותר, הקו תלול יותר. יש לשים לב שבניגוד לתרשים חד-ממדי בו אורך קטע קו הוא מרחק בין שתי נקודות, כאן המרחק בין שתי נקודות הוא ההפרש בציר האנכי בין הנקודות.


הפתרון של בעיית תנועה מחייב שימוש בנוסחה מרחק שווה זמן כפול מהירות, ומקובל לארגן את הנתונים בטבלה. הגרף מיועד לשפר את השלב הראשון בהתמודדות עם הבעיה: הבנת הקשרים בין התכונות של מסלולי הדמויות בתסריט. במאמר זה, לאחר הצגת הגרף אקצר בשלבים הבאים בפתרון.


הנוסחה של בעיות הספק דומה לזו של בעיות תנועה, ולכן ניתן להשתמש בגרפים גם עבור בעיות הספק.


כדי לבדוק את השיטה, פתרתי בעזרת גרפים את כל בעיות התנועה וההספק משאלוני
$806$
של בחינות הבגרות בשנים תשע"ד ועד תשע"ז. מאמר זה מציג תחילה את הפתרון לשאלה פשוטה יחסית הלקוחה מבחינת הבגרות של קיץ תשע"ד מועד א', ואחר כך את הפתרון לשאלה הלקוחה מהבחינה של קיץ תשע"ה מועד ב'. לדעתי, בדוגמה השנייה, הגרף תורם תרומה משמעותית להבנת התסריט ולפתרון הבעיה.


שאר הפתרונות נמצאים במסמך שניתן להוריד מהאתר שלי:


\selectlanguage{english}
\url{http://www.weizmann.ac.il/sci-tea/benari/mathematics/}.
\selectlanguage{hebrew}


כהשלמה למאמר זה אני ממליץ לקרוא אלבוים-כהן וקופר
$(2015)$
המציג פתרונות גיאומטריים לבעיות הספק.

%%%%%%%%%%%%%%%%%%%%%%%%%%%%%%%%%%%%%%%%%%%%%%%%%%%%%%%%%%%%%%%%

\newpage

\section*{מפגש בין מכונית למשאית}

\textbf{השאלה}

משאית יצאה מעיר
$A$,
וכעבור 
$6$
שעות מרגע יציאתה הגיעה לעיר
$B$.
זמן מה אחרי יציאת המשאית יצאה מכונית מעיר
$A$,
והגיעה לעיר
$B$
$2$
שעות לפני המשאית. המשאית והמכונית נפגשו כעבור שעה מרגע יציאתה של המכונית. המהירויות של המשאית ושל המכונית היו הקבועות. מצא כמה שעות אחרי היציאה של המשאית יצאה המכונית )מצא את שני הפתרונות(.

\textbf{ייצוג בגרף}

הציר האופקי בגרף מסמן את מעבר הזמן מיציאת המשאית מעיר
$A$
ועד להגעתה לעיר
$B$
כעבור
$6$
שעות. הציר האנכי הוא המרחק בין שתי הערים. מתחת לציר האופקי בגרף מסומנים הזמנים לפי התסריט. המהירויות קבועות, ולכן לכל דמות מסלול המורכב ממקטע אחד בלבד. המכונית נוסעת מהר יותר מהמשאית כך שהקו שלה תלול יותר מזה של המשאית. אין חשיבות לדיוק בקנה המידה בגרף, כי מטרת הגרף היא רק להבין את הקשרים בין המסלולים ולא לחשב את הפתרונות.

\begin{center}
\selectlanguage{english}
\begin{tikzpicture}
\draw (0,0) node[left] {$A$} -- (10,0);
\draw (0,0) -- (0,6) node[left] {$B$};
\draw[dashed] (0,6) -- (10,6);
\draw[thick,name path=truck] (0,0) -- node[left,near start,xshift=-6pt] {
\R{משאית}
} (10,6);
\draw[thick,name path=car] (3,0) -- node[left,near end] {
\R{מכונית}
} (7,6);
\path [name intersections={of=truck and car,by=meeting}];
\draw[dashed] (meeting) |- coordinate (time) (0,0);
\draw[dashed] (meeting) -| (0,0);
\draw[dashed] (10,0) -- (10,6);
\draw[dashed] (7,0) -- (7,6);
\fill (meeting) circle [radius=2pt];
\fill (time) circle [radius=2pt];
\fill (0,0) circle [radius=2pt];
\fill (3,0) circle [radius=2pt];
\fill (7,0) circle [radius=2pt];
\fill (10,0) circle [radius=2pt];
\draw[<->] (0,-.5) -- node[fill=white] {$t$} (3,-.5);
\draw[<->] (3,-.5) -- node[fill=white] {
\R{שעה}
 $1$} (time |- 0,-.5);
\draw[<->] (7,-.5) -- node[fill=white] {
\R{שעות}
 $2$} (10,-.5);
\draw[<->] (3.1,-1) -- node[fill=white] {$4-t$} (6.9,-1);
\draw[<->] (0,-1.5) -- node[fill=white] {
\R{שעות}
 $6$} (10,-1.5);
\end{tikzpicture}
\end{center}

נסמן:
$=t$
זמן יציאת המכונית,
$=v_c$
מהירות המכונית,
$=v_m$
מהירות המשאית.


המרחקים הם ההפרשים בין הנקודות בציר האנכי. נכתוב משוואות למרחקים שווים, מ-
$A$
עד למפגש ומ-
$A$
עד ל-
$B$:
\begin{eqnarray*}
v_m(t+1) &=& v_c\cdot 1\\
v_m \cdot 6 &=& v_c (4-t)\,.
\end{eqnarray*}
משתי המשוואות מתקבלת משוואה ריבועית ב-
$t$:
\[
t^2 - 3t + 2 = 0
\]
שיש לה שני פתרונות
$1=t$
שעה ו-
$2=t$
שעות.




%%%%%%%%%%%%%%%%%%%%%%%%%%%%%%%%%%%%%%%%%%%%%%%%%%%%%%%%%%%%%%%%

\newpage

\section*{יוסי נפגש עם אמא}

\textbf{השאלה}

בזמן הנסיעה באוטובוס הבחין יוסי ברגע מסוים באימא שלו, ההולכת ליד האוטובוס 
בכיוון הפוך לכיוון הנסיעה של האוטובוס. כעבור
$10$
שניות מהרגע שיוסי הבחין באימו, עצר האוטובוס בתחנה, ויוסי רץ מיד כדי להשיג את אימו. מהירות הריצה של יוסי גדולה פי
$2$
ממהירות ההליכה של אימו, והיא
$\frac{1}{7}$
ממהירות הנסיעה של האוטובוס. כל המהירויות הן קבועות.

א. כמה זמן רץ יוסי כדי להשיג את אימו?

ברגע שיוסי השיג את אימו, הם הלכו יחד
$3$
\textbf{דקות}
במהירות ההליכה של אימו )בכיוון ההליכה שלה(. מיד בתום
$3$
הדקות רץ יוסי בחזרה לתחנת האוטובוס שירד בה. )מהירות הריצה של יוסי היא כמו בסעיף א.(

ב. כמה זמן רץ יוסי בחזרה לתחנת האוטובוס?

%%%%%%%%%%%%%%%%%%%%%%%%%%%%%%%%%%%%%%%%%%%%%%%%%%%%%%%%%%%%%%%%

\bigskip

\textbf{ייצוג בתרשים חד-ממדי}

להלן תרשים חד-ממדי שהשתמשתי בו בניסיונות הראשונים שלי לפתור את הבעיה:

\begin{center}
\selectlanguage{english}
\begin{tikzpicture}
\draw[dashed,->] (4,1) node[yshift=1pt,above] {
\R{תחנה}
} -- (4,-2);
\draw[dashed,->] (0,1) node[above] {
$1$
\R{מפגש}
} -- (0,.1);
\draw[dashed,->] (-3,1) node[yshift=1pt,above] {
$2$
\R{מפגש}
} -- (-3,0);
\draw[dashed,->] (-6,1) node[yshift=1pt,above] {
\R{פרידה}
} -- (-6,-2);
\draw[->] (0,0) -- node[above] {
\R{יוסי אוטובוס}
} (4,0);
\draw[->] (0,0) -- node[above] {
\R{אמא}
} (-3,0);
\draw[fill] (0,0) circle [radius=1pt];
\draw[->] (4,-1) -- node[above] {
\R{יוסי ריצה}
} (-3,-1);
\path (0,-1) -- (0,-1.1);
\draw[->] (-3,0) -- node[above] {
\R{הליכה ביחד}
} (-6,0);
\draw[<-,dashed] (-3,0) -- (-3,-1);
\draw[->] (-6,-2) -- node[above] {
\R{יוסי ריצה בחזרה}
} (4,-2);
\path (0,-3.1) rectangle (5,.1);
\end{tikzpicture}
\end{center}
\vspace*{-6ex}

ברור מהתרשים מה היחסים בין המרחקים אבל קשה להבין את יחסים בין הזמנים.

%%%%%%%%%%%%%%%%%%%%%%%%%%%%%%%%%%%%%%%%%%%%%%%%%%%%%%%%%%%%%%%%

\bigskip

\textbf{ייצוג בגרף}

להלן גרף המתאר את התסריט בבעיה:

\begin{center}
\selectlanguage{english}
\begin{tikzpicture}[scale=.9]
\draw (0,0) -- (14,0);
\draw (0,-4) node[left] {
\R{פרידה}
} -- (0,-2) node[left] {
\R{מפגש 2}
} -- (0,0) node[left] {
\R{מפגש 1}
} -- (0,4) node[left] {
\R{תחנה}
};
\fill (0,0) circle [radius=2pt];
\fill (1,0) circle [radius=2pt] node[below right] {$M$};
\fill (4,0) circle [radius=2pt] node[above] {$N$};
\fill (8,0) circle [radius=2pt] node[above] {$P$};
\fill (11,0) circle [radius=2pt] node[below right] {$Q$};
\fill (14,0) circle [radius=2pt] node[below] {$R$};
\draw[thick] (0,0) -- node[left] {$a$} (1,4) -- node[right,near start] {$b$} (4,-2);
\draw[thick] (0,0) -- node[below,near start,xshift=-2mm] {$c$} node[right,near end,yshift=2mm] {$d$} (8,-4)  node[below] {$P'$} -- (14,4)  node[right] {$R'$};
\draw[dashed] (0,4) -- (14,4);
\draw[dashed] (0,-2) -- (14,-2);
\draw[dashed] (0,-4) -- (14,-4);
\draw[dashed] (1,4)  node[above] {$M'$} -- (1,0);
\draw[dashed] (4,0) -- (4,-2) node[below,yshift=-1mm] {$N'$};
\draw[dashed] (8,-4) -- (8,0);
\draw[dashed] (11,0) -- (11,4);
\draw[dashed] (14,0) -- (14,4);
\draw[<->] (0,.7) -- node[fill=white] {$10$} (1,.7);
\draw[<->] (1,.7) -- node[near start,fill=white] {$t$} (4,.7);
\draw[<->] (4,.7) -- node[fill=white] {$180$} (8,.7);
\draw[<->] (8,.7) -- node[fill=white] {$t_1$} (11,.7);
\draw[<->] (11,.7) -- node[fill=white] {$t_2$} (14,.7);
\path (8,-2) --  node[below right,xshift=5mm,yshift=-2mm] {$e_1$} (11,0);
\path (11,0) --  node[left,yshift=3mm] {$e_2$} (14,4);
\end{tikzpicture}
\end{center}


%%%%%%%%%%%%%%%%%%%%%%%%%%%%%%%%%%%%%%%%%%%%%%%%%%%%%%%%%%%%%%%%

המסלול של יוסי מורכב מארבעה מקטעים נפרדים )אחד משותף עם אמו( בשלוש מהירויות שונות )נסיעה באוטובוס, ריצה, הליכה עם אמא(. בחרתי לחלק את המקטע האחרון של יוסי לשני מקטעים: הריצה לנקודת המפגש הראשונה ומשם לתחנה.

כנקודת ייחוס בחרתי את הנקודה ההתחלתית של התסריט. בגלל שהתנועה היא בשני כיוונים מנקודה זו, לציר האנכי ערכים חיוביים ושליליים.

כדי להקל על הדיון סימנתי כל מקטע באות אנגלית קטנה וכל נקודת זמן באות אנגלית גדולה. משמשעות הקטעים היא:
$=a$
יוסי נוסע באוטובוס,
$=b$
יוסי רץ לפגישה עם אמא,
$=c$
אמא הולכת עד למפגש עם יוסי,
$=d$
יוסי ואמא הולכים ביחד,
$=e_1+e_2$
יוסי רץ חזרה לתחנה.

היחסים בין המהירויות של הדמויות נתונים בשאלה וציירתי את הקווים כך שהשיפוע של קו האוטובוס גדול מזה של הקו של יוסי, והשיפוע של הקו שלו גדול מהשיפוע של הקו של אמו.


נסמן:
$=t$
הזמן שיוסי רץ מהתחנה כדי להשיג את אמא.

נסמן מהירויות:
$=v_y$
יוסי, 
$=v_a$
אמא, 
$=v_b$
אוטובוס.


נתון:
$v_y=2v_a$, $v_y=v_b/7$.

%%%%%%%%%%%%%%%%%%%%%%%%%%%%%%%%%%%%%%%%%%%%%%%%%%%%%%%%%%%%%%%%

\paragraph{סעיף א}

$NN'$
הוא המרחק ממפגש
$1$
למפגש
$2$.
אמא הלכה לפי הקו
$c$
ולכן המרחק שהלכה הוא
$v_a(t+10)$.
יוסי הלך לכיוון אחד לפי הקו
$a$
וחזר לפי הקו
$b$.
ההפרש בין מרחקים יהיה שווה גם הוא ל-
$NN'$.
נכתוב משוואה לשוויון המרחקים:

\[
v_a(t+10) = v_yt - v_b \cdot 10.
\]
לאחר הצבת יחסי המהירויות הנתונים:
\[
\frac{v_y}{2}(t+10) = v_yt - 7\cdot v_y\cdot 10
\]
נקבל 
$150=t$
שניות.
%%%%%%%%%%%%%%%%%%%%%%%%%%%%%%%%%%%%%%%%%%%%%%%%%%%%%%%%%%%%%%%%


\paragraph{סעיף ב}

הקו
$e_1+e_2$
מתאר את הריצה של יוסי בחזרה לתחנה.


$PP'$
הוא המרחק שאמא עברה בין מפגש
$1$
לבין נקודת הפרידה וגם המרחק של
$e_1$,
המקטע הראשון של הריצה של יוסי. פרק הזמן שאמא הלכה מרחק זה הוא
$340=180+150+10$
שניות. יוסי רץ פי שניים מהר מאמא, ולכן את המקטע
$e_1$
הוא עבר ב
$170=t_1$
שניות.


$MM'$
הוא המרחק שהאוטובוס עבר ממפגש
$1$
ועד התחנה.
$RR'=MM'$,
המרחק של
$e_2$,
המקטע השני של הריצה של יוסי. האוטובוס עבר מרחק זה ב
$10$
שניות. יוסי רץ פי שבע לאט, ולכן את המקטע
$e_2$
הוא עבר ב
$70=t_2$
שניות.


נסכם ונקבל שיוסי רץ מנקודת הפרידה לתחנה ב
$240=140+70=t_1+t_2$
שניות.


\textbf{תרומת הגרף}
בשאלה זו מצאתי שהגרף עזר פעמיים: לראות שהריצה של יוסי בחזרה לתחנה מורכבת משני מקטעים )עד הציר האופקי
$e_1$
ומעל לציר האופקי
$e_2$(,
ובחישוב הזמנים
$340=180+150+10$
שממש "קופץ לעין" מהגרף.

%%%%%%%%%%%%%%%%%%%%%%%%%%%%%%%%%%%%%%%%%%%%%%%%%%%%%%%%%%%%%%%%

\section*{מסקנות}

כדי לפתור בעיות תנועה יש למצוא משוואות בין מרחקים שווים או זמנים שווים. השימוש בגרף מאפשר למצוא את המשוואות בקלות. גם המהירויות מיוצגות בצורה ברורה כשיפועים של קווי התנועה. התרומה העיקרית של הגרף היא בחלוקת התסריט למקטעים ברורים, ולכן אין צורך לדייק בקנה המידה וניתן להשתמש בשיטה זו בציור ידני במבחנים.

\section*{הבעת תודה}

ברצוני להודות לאביטל אלבוים-כהן שהעלתה בפניי את האפשרות להשתמש בגרפים. קבלתי הערות מועילות מגילה רון והשופטים.


\section*{מקורות}

אלבוים-כהן, א., קופר, ג.
$(2015)$.
\textit{פתרונות שונים לבעיות הספק באמצעים גרפיים}.
על"ה
$51$, $14$-$19$.
\end{document}
