\documentclass[11pt,a4paper]{report}

\usepackage{mathpazo}
\usepackage{microtype}
\usepackage{verbatim}

\textwidth=15cm
\textheight=23cm
\topmargin=12pt
\headheight=0pt
\oddsidemargin=2em
\headsep=0pt
\renewcommand{\baselinestretch}{1.1}
\setlength{\parskip}{0.2\baselineskip plus 1pt minus 1pt}
\parindent=0pt

\newcommand*{\p}[1]{\texttt{\textup{\footnotesize #1}}}
\newcommand*{\ih}{\stackrel{\bullet}{=}}
\newcommand*{\ihge}{\stackrel{\bullet}{\geq}}
\newcommand*{\ihlt}{\stackrel{\bullet}{<}}
\newcommand*{\ihle}{\stackrel{\bullet}{\leq}}
\newcommand*{\qeq}{\stackrel{?}{=}}
\newcommand*{\qge}{\stackrel{?}{\geq}}

\newcommand*{\qed}{\hfill\rule{1ex}{1.5ex}}
\newcommand*{\qedd}[1]{\vspace*{-#1ex}\qed}

\begin{document}
\thispagestyle{empty}

By Christian Goldbach

Fermat numbers are defined:
\[
F_n=2^{2^n}+1,\quad\quad n\geq 0\,.
\]
The first five Fermat numbers are:
\[
F_0=3,\;F_1=5,\;F_2=17,\;F_3=257,\;F_4=65537\,.
\]
Prove:
\[
F_n = \prod_{k=0}^{n-1} F_k + 2\,.
\]
Examples:
\[
5=3+2,\quad 17=3\cdot 5+2,\quad 257=3\cdot 5\cdot 17+2,\quad 65537=3\cdot 5\cdot 17\cdot 257+2\,.
\]

Proof: By induction.

%\begin{comment}
For $n=1$:
\[
5=F_1=\prod_{k=0}^{0} F_k + 2=F_0+2=3+2\,.
\]
Suppose the formula holds for $n$. Then:
\begin{eqnarray*}
\prod_{k=0}^{n}F_k
&=&(\prod_{k=0}^{n-1}F_k) F_n \\
&=& (F_n-2)F_n\\
&=& (2^{2^n}+1-2)(2^{2^n}+1)\\
&=&(2^{2^n}-1)(2^{2^n}+1)\\
&=& (2^{2^n})^2-1^2\\
&=& 2^{2^{n+1}}-1\\
&=& (2^{2^{n+1}}+1)-2\\
&=&F_{n+1}-2\,.
\end{eqnarray*}

%\end{comment}

Claim: Any two Fermat numbers are relatively prime. For if not, if $m>1$ divides $F_{n_1}$ and $F_{n_2}$, $F_{n_2} < F_{n_1}$, it must divide
\[
F_{n_1} - F_{n_2}\left(\prod_{k=n_2+1}^{n_1-1}F_k\right) = 2\,.
\]
Therefore, $m=2$ but that is impossible because all Fermat numbers are odd.

It follows that there is an infinite number of prime numbers.


\end{document}
