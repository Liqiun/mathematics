% !TeX root = induction-he.tex

\chapter{%
עיקרון הסדר הטוב%
}\label{s.well}

עיקרון הסדר הטוב שקול לאינדוקציה מתמטית, אבל קל יותר להשתכנע בנכונותו. בפועל, קל יותר להשתמש באינדוקציה.

\section{%
סדר מלא ועיקרון הסדר הטוב%
}

\begin{definition}
תהי 
$S$
קבוצה עם יחס בינרי
$\leq$.
\begin{enumerate}
\item
קבוצה
$S$
\textbf{מסודרת בסדר מלא}
אם עבור כל זוג איברים
$x,y\in S$,
מתקיים
$x\leq y$
או
$y \leq x$
או
$x=y$.
\item
לקבוצה מסודרת בסדר מלא
$S$
\textbf{חסם תחתון}
אם קיים
$b$
כך ש
$b\leq n$
לכל
$n\in S$.
\item
לקבוצה מסודרת בסדר מלא
$S$
\textbf{איבר קטן ביותר}
אם קיים
$b\in S$
כך ש-%
$b\leq n$
לכל
$n\in S$.
\end{enumerate}
\end{definition}
איבר קטן ביותר הוא גם חסם תחתון, אבל cנוסף הuא איבר בתוך הקבוצה.

כל תת-קבוצה של המספרים השלמים היא מסודרת בסדר מלא, למשל,
$S_1=\{8,3,19,5,6,23\}$
וקבוצת המספרים הזוגיים
$E=\{\ldots, -4, -2, 0, 2, 4, \ldots\}$.
חלק מהחסמים התחתונים של
$S_1$
הם
$3, 0, -10$.
למעשה, כל
$b\leq 3$
הוא חסם תחתון עבור
$S_1$.
האיבר הקטן ביותר הוא של
$S_1$
הוא
$3$.
ברור שלקבוצה
$E$
אין חסם תחתון ובוודאי שאין לה איבר קטן ביותר.

לקבוצת המספרים הרציונליים 
\textbf{החיוביים}
מספר אינסופי של חסמים תחתונים )אפס וכל המספרים הרציונליים השליליים(, אבל אין לה איבר קטן ביותר כי לכל מספר רציונלי חיובי
$x$, $x/2$
הוא מספר רציונלי חיובי קטן יותר.

\begin{definition}
תהי 
$S$
קבוצה המסודרת במסדר מלא.
$S$
היא
\textbf{מסודרת היטב}
אם
\textbf{בכל}
תת-קבוצה לא ריקה
של 
$S$
קיים איבר קטן ביותר.
\end{definition}

הקבוצה
$S_1$
היא מסודרת היטב כי לכל תת-קבוצה לא ריקה קיים איבר קטן ביותר. האיבר הקטן ביותר של 
$S_1$
הוא
$3$,
האיבר הקטן ביותר של
$\{8, 19, 5\}$
הוא
$5$,
ואתם יכולים לבדוק שלכל אחת מ-%
$63$
התת-קבוצות הלא-ריקות של
$S_1$
קיים איבר קטן ביותר. הקבוצה
$E$
\textbf{אינה}
מסודרת היטב כי 
$E$
היא תת-קבוצה של עצמה ואין מספר זוגי קטן ביותר. אולם, הקבוצה
$E_6=\{6,12,18,\ldots\}$,
קבוצת המספרים הזוגיים החיוביים המתחלקים ב-%
$6$
היא כן מסודרת היטב כי לכל תת-קבוצה קיים איבר קטן ביותר.

\begin{axiom}[%
עיקרון הסדר הטוב%
]
כל תת-קבוצה לא ריקה של המספרים השלמים שיש לה חסם תחתון היא מסודרת היטב.
\end{axiom}
נשתמש במקרה פרטי של האקסיומה:
\begin{axiom}
בכל תת-קבוצה לא ריקה של המספרים החיוביים קיים איבר קטן ביותר.
\end{axiom}


\section{%
השקילות של עיקרון הסדר הטוב ואינדוקציה מתמטית%
}

\begin{theorem}\label{th.wop}
עיקרון הסדר הטוב גורר את העיקרון של אינדוקציה מתמטית.
\end{theorem}

\textbf{הוכחה}
אם העיקרון של אינדוקציה מתמטית לא נכון, חייבת להיות תכונה כלשהי
$P(n)$,
כך ש-%
$P(1)$
נכונה והצעד האינדוקטיבי נכון )לכל
$m$, $P(m)$
גורר
$P(m+1)$(,
אבל, עבור
$n>1$
כלשהו
$P(n)$
אינה נכונה. תהי
$S$
קבוצת המספרים השלמים החיוביים
$k$
כך ש-%
$P(k)$
אינה נכונה. הקבוצה
$S$
אינה ריקה כי
$n\in S$.
לפי עיקרון הסדר הטוב, לקבוצה איבר קטן ביותר
$b\in S$.
לפי ההגדרה של
$S$, $P(b)$
איננה נכונה. אבל
$b-1$
קטן מ-%
$b$
כך ש-%
$b-1\not\in S$
ולכן
$P(b-1)$
נכונה. לפי צעד האינדוקציה,
$P(b)$
נכונה וקיימת סתירה.
\qed

נדגים את ההוכחה באמצעות דוגמה. תהי 
$P(n)$
תכונה כך ש-%
$P(1)$
נכונה )מסומן ב-%
$+$(, 
צעד האינדוקציה נכון, אבל
$P(244)$
לא נכונה )מסומן על ידי
$-$(.
תהי
$S=\{244,57,102,\ldots\}$
קבוצת המספרים החיוביים
$k$
עבורם
$P(k)$
לא נכונה. הקבוצה
$S$
אינה ריקה כי
$244\in S$.
לפי עיקרון הסדר טוב ל-%
$S$
איבר קטן ביותר
$57$.
אבל,
$56\not\in S$
כך ש-%
$P(56)$
נכונה. לפי צעד האינדוקציה
$P(57)$
נכונה וזו סתירה )מסומן על ידי
$\mp$(.
\[
\begin{array}{ccccccccc}
P(1) & P(2) & \cdots & P(56) & P(57) & \cdots & P(102) & \cdots & P(244)\\
+ & + & + & + & \mp & + & -&+&-
\end{array}
\]

\vspace{-3ex}

\begin{theorem}
העיקרון של אינדוקציה מתמטית גורר את עיקרון הסדר הטוב.
\end{theorem}

\textbf{הוכחה} 
תהי
$S$
תת-קבוצה לא-ריקה של המספרים החיוביים. נניח שאין איבר קטן ביותר בקבוצה 
$S$.
נגדיר את התכונה
$P(n)$:
\begin{quote}
$P(n)$
נכונה אם לכל 
$k\leq n$, $k$
הוא  חסם תחתון של
$S$
אבל
$k\neq S$.
\end{quote}
נוכיח ש-%
$P(n)$
נכונה לכל המספרים החיוביים, ולכן כל המספרים החיוביים אינם איברים של
$S$,
סתירה להנחה ש-%
$S$
לא ריקה.

טענת בסיס: 
$1$
הוא חסם תחתון של
$S$
כי הוא המספר החיובי הקטן ביותר. לפי ההגדרה של
$S$,
אין ב-%
$S$
איבר קטן ביותר, כך ש-%
$1\not\in S$
ו-%
$P(1)$
נכונה.

צעד אינדוקטיבי: נניח ש-%
$P(n)$
נכונה כך ש-%
$n$
הוא חסם תחתון עבור 
$S$, 
אבל
$n\not\in S$,
כלומר, לכל
$k\leq n$, $k<s$
לכל
$s\in S$.
אזי
$n+1\leq s$
לכל
$s \in S$,
ולכן
$n+1$
הוא חסם תחתון ל-%
$S$.
אם
$n+1\in S$,
$n+1$
הוא איבר קטן ביותר ב-%
$S$ 
)כי
$k\not\in S$
לכל
$k\leq n$(,
סתירה להגדרה של 
$S$.
מכאן ש-%
$n+1\not\in S$,
כלומר,
$P(n+1)$
נכונה.
\qed


%%%%%%%%%%%%%%%%%%%%%%%%%%%%%%%%%%%%%%%%%%%%%%%%%%%%%%%%%%%%%%%%%%%

\section{%
אינדוקציה מוזרה ביותר%
}

אנו קושרים אינדוקציה עם הוכחת תכונות של מספרים שלמים ומסמך זה הראה את החשיבות של אינדוקציה מבנית. כאן אנו מביאים הוכחת באינדוקציה על קבוצה מוזרה של מספרים שלמים. האינדוקציה תקפה כי הדרישה היחידה היא שהקבוצה מקיימת את עיקרון הסדר בטוב עם אופרטור יחס כלשהו.

להלן פונקציה רקורסיבית מעל למספרים השלמים:
\[
f(x) = \textrm{if}\;\; x > 100 \;\;\textrm{then}\;\; x - 10 \;\;\textrm{else}\;\; f(f(x+11))\,.
\]
עבור מספרים גדולים מ-
$100$,
חישוב הפונקציה פשוטה ביותר:
\[
f(101) = 91, \;\; f(102) = 92,\;\; f(103) = 93,\;\; f(104) = 94\,.
\]
מה עם מספרים גדולים או שווים ל-%
$100$?
\begin{eqnarray*}
f(100) &=& f(f(100+11)) = f(f(111)) = f(101) = 91\\
f(99) &=& f(f(99+11)) = f(f(110)) = f(100) = 91\\
f(98) &=& f(f(98+11)) = f(f(109)) = f(99) = 91\\
&\cdots&\\
f(91) &=& f(f(91+11)) = f(f(102)) = f(92) = f(f(103)) = f(93) = \cdots\\
&&f(99) = f(f(110)) = f(100) = f(f(111)) = f(101) = 91\\
f(90) &=& f(f(90+11)) = f(f(101)) = f(91) = 91\\
f(89) &=& f(f(89+11)) = f(f(100)) = f(f(111)) = f(101) = 91\,.
\end{eqnarray*}
כפי שאמרה עליסה: "יותר מיותר מוזר!" נשער שעבור כל המספרים השלמים, הפונקציה
$f$
שווה לפונקציה
$g$:
\[
g(x) = \textrm{if}\;\; x > 100 \;\;\textrm{then}\;\; x - 10 \;\;\textrm{else}\;\; 91\,.
\]
הפונצקיה
$f$
הוגדרה לראשונה על ידי
\L{John McCarthy},
אחד מחלוצי מדעי המחשב, ונקרא פונקצית -%
$91$
של
\L{McCarthy}.

\newpage

\begin{theorem}
עבור כל מספר שלם
$x$, $f(x) = g(x)$.
\end{theorem}
ההוכחה מבוססת על הוכחה ב-%
\L{Z. Manna. \textit{Mathematical Theory of Computing}, 1974, 411--12},
ומיוחס ל-%
\L{R.M. Burstall}.

ההוכחה באינדוקציה מעל לקבוצת המספרים:
\[
S=\{x\,|\,x\leq 101\}
\]
אם אופרטור היחס
$\prec$
המוגדר כך:
\[
x \prec y \;\; \textrm{iff}\;\; y < x\,,
\]
כאשר בצד הימני 
$<$
הוא אופרטור היחס הרגיל מעל למספרים שלמים. הנה סדר המספרים הנובע מהיחס
$\prec$:
\[
101 \prec 100 \prec 99 \prec 98 \prec 97 \prec \cdots\,.
\]
הקבוצה
$S$
עם האופרטור
$\prec$
מסודר בסדר טוב כי כל תת-קבוצה של
$S$
מכילה איבר קטן ביותר.

\noindent\textbf{הוכחה}
נוכיח את המשפט בשלושה חלקים.

\noindent\textbf{מקרה 1}  $x > 100$.
ההוכחה מיידית מההגדרות של 
$f$
ו-
$g$.

\noindent\textbf{מקרה 2} 
$90\leq x \leq 100$.

\noindent{}%
טענת הבסיס היא:
\[
f(100) = f(f(100+11)) = f(f(111)) = f(101) = 91 = g(100)\,,
\]
לפי ההגדרה של
$g$
לכל המספרים השלמים פחות או שווה ל-%
$100$.

\noindent{}%
הנחת האינדוקציה היא
$f(y) = g(y)$
עבור
$y\prec x$.

\noindent{}%
הצעד האינדוקטיבי הוא:
\vspace*{-1ex}
\begin{eqnarray}
f(x) &=& f(f(x+11))\label{m91-1}\\
&=& f(x+11-10)= f(x+1)\label{m91-3}\\
&=& g(x+1)\label{m91-4}\\
&=& 91\label{m91-5}\\
&=& g(x)\label{m91-6}\,.
\end{eqnarray}
משוואה%
~\ref{m91-1}
נכונה מההגדרה של
$f$
כי
$x\leq 100$.
השוויון בין משוואה%
~\ref{m91-1}
לבין משוואה%
~\ref{m91-3}
נכון מההגדרה של
$f$
כי
$x \geq 90$
ולכן
$x+11 > 100$.
השוויון בין משוואה%
~\ref{m91-3}
ומשוואה%
~\ref{m91-4}
נובע מהנחת האינדוקציה:
\[
x\leq 100 \Rightarrow x+1 \leq 101 \Rightarrow x+1\in S \Rightarrow x+1\prec x\,.
\]
השוויון בין המשוואות%
~\ref{m91-4}, \ref{m91-5}, \ref{m91-6}
נכון מההגדרה של 
$g$
ו-%
$x+1 \leq 101$.

\newpage

\noindent\textbf{מקרה 3} $x< 90$.

\noindent{}%
טענת הבסיס היא:
\[
f(89) = f(f(100)) = f(f(f(111))) = f(f(101)) = f(91) = 91 = g(89)\,,
\]
לפי ההגדרה של
$g$
כי
$89<100$.

\noindent{}%
הנחת האינדוקציה היא
$f(y) = g(y)$
עבור
$y\prec x$.

\noindent{}%
הצעד האינדוקטיבי הוא:
\vspace*{-1ex}
\begin{eqnarray}
f(x) &=& f(f(x+11))\label{m91a}\\
&=& f(g(x+11))\label{m91b}\\
&=& f(91)\label{m91c}\\
&=& 91\label{m91d}\\
&=& g(x)\,.
\end{eqnarray}
משוואה%
~\ref{m91a}
נכונה לפי ההגדרה של
$f$
ו-%
$x<90\leq 100$.
השוויון בין המשוואות
~\ref{m91a}
ו-%
~\ref{m91b}
נובע מהנחת האינדוקציה:
\[
x < 90 \Rightarrow x+11< 101 \Rightarrow x+11\in S \Rightarrow x+11 \prec x\,.
\]
השוויון בין המשוואות%
~\ref{m91b}
ו-%
~\ref{m91c}
נכון לפי ההגדרה של
$g$
ו-%
$x+11 < 101$.
לבסוף, כבר הוכחנו ש-%
$f(91)=91$,
ולפי ההגדרה,
$g(x)=91$
עבור
$x<90$.\qed

%%%%%%%%%%%%%%%%%%%%%%%%%%%%%%%%%%%%%%%%%%%%%%%%%%%%%%%%%%%%%%%%%%%


\chapter{%
מסקנות%
}\label{s.conclusion}

אינדוקציה היא
\textbf{אקסיומה}
שמשתמשים בה לעתים קרובות במתמטיקה. ראינו שאינדוקציה מופיעה בתחפושות רבות היכולות לבלבל, אבל בכל מקרה המושגים הבסיסיים הם אחידים:
\begin{itemize}
\item
הוכח תכונה עבור מבנים קטנים שהם כל כך פשוטים שההוכחה ברורה מאליו. ייתכן שיהיו מספר טענות בסיס ויש להתייחס אל כולן.
\item
בדומה להפלת שורה של לבני דומינו, הראה שההנחה שהתכונה נכונה עבור מבנים קטנים יכולה לשמש להוכחת התכונה עבור מבנים גדולים יותר. ייתכן שיהיו מספר צעדי אינדוקציה. 
\item
לפי עיקרון האינדוקציה ניתן עכשיו להסיק שהתכונה נכונה לכל מבנה.
\end{itemize}

האינדוקציה יכולה להיות מעל למספרים השלמים, מעל למספר הקווים בתרשים גיאומטרי או מעל לנוסחאות לוגיות או אוטומטים.

לעתים השימוש באינדוקציה הוא לא-מפורש ומסתתר מתחת לביטויים כמו "בלי הגבלת הכלליות" או "החלף את כל המופעים של". יש לזהות את השימוש באינדוקציה במקרים אלה גם אם לא רושמים את כל הפרטים.

%%%%%%%%%%%%%%%%%%%%%%%%%%%%%%%%%%%%%%%%%%%%%%%%%%%%%%%%%%%%%%%%%%%
