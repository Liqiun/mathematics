% !TeX root = induction-he.tex

\chapter{לוגיקה מתמטית}\label{s.logic}

אינדוקציה היא כלי חיוני בלוגיקה מתמטית כי המשפטים טוענים לתכונות של 
\textbf{כל}
הנוסחאות. בגלל שהנוסחאות בנויות מתת-נוסחאות, המורכבות באמצעות אופרטורים לוגיים, הוכחת משפט היא על ידי אינדוקציה על מבנה הנוסחה. בפרק זה נראה איך משתמשים באינדוקציה, לעתים בצורה לא-מפורשת, בהוכחות בתחשיב הפסוקים.

\begin{definition}
נוסחה 
$A$
היא 
\textbf{ספיקה}
אם ורק אם קיימת השמה של
$T$
או
$F$
לכל
\textbf{אטום}
כך שערכה של הנוסחה
$A$
היא
$T$.
\end{definition}

\begin{definition}
נוסחה
$A$
\textbf{שקולה לוגית}
לנוסחה
$A'$
אם ורק אם ערכיהן שווים לכל השמה לאטומים. סימון:
$A\equiv A'$.
\end{definition}

הנה דוגמה המבהירה את ההבדל בין המושגים 
\textbf{שקולה לוגית}
ו-
\textbf{ספיקה אם ורק אם}.

נתונה שתי נוסחאות:
\[
\begin{array}{l}
A=(p \vee q \vee \neg r) \;\wedge\; (p \vee \neg q) \;\wedge\; (\neg p \vee q)\,,\\
B=(p \vee \neg q) \;\wedge\; (\neg p \vee q)\,.
\end{array}
\]
$A$
ספיקה בגלל שערכה הוא
$T$
עבור ההשמה
$\{p=F,q=F,r=F\}$
ו-
$B$
ספיקה עבור אותה השמה. אולם, עבור ההשמה
$\{p=F,q=F,r=T\}$,
ערכה של
$A$
הוא
$F$
וערכה של
$B$
הוא
$T$,
כך שהנוסחאות אינן שקולות לוגית.

\section{%
המרת נוסחה ל-%
\L{\small CNF}
}

\begin{definition}
נוסחה בתחשיב הפסוקים היא בצורת
\L{\textbf{conjunctive normal form (\small CNF)}}
אם היא חיתןך )"ו"( של איחוד )"או"( של ליטרלים )אטומים או שלילה של אטומים(.
\end{definition}

\vspace*{-10pt}
\textbf{דוגמה}
הנוסחה שלהלן איננה ב-%
\L{\small CNF}:
\begin{equation}\label{eq.logic}
(p \wedge \neg q) \;\rightarrow\; (\neg q \wedge p)\,,
\end{equation}
אבל היא שקולה לוגית לנוסחה שלהלן שהיא ב-%
\L{\small CNF}:
\begin{equation}\label{eq.cnf3}
(\neg p \vee q \vee \neg q) \;\wedge\; (\neg p \vee q \vee p)\,.
\end{equation}
לצורת
\L{\small CNF}
חשיבות רבה במדעי המחשב, גם בתיאוריה וגם בשימוש מעשי. בתיאוריה: 
\L{Stephen Cook}
הוכיח שהקביעה האם נוסחה ב-
\L{\small CNF}
היא ספיקה היא בעיית
\L{-NP}%
שלמה. צורת
\L{\small CNF}
נמצאת בשימוש נרחב בהוכחה אוטומטית של משפטים, בתכנות לוגי ובפתרון בעיות על ידי בדיקת ספיקות.
\begin{theorem}\label{t.cnf}
תהי 
$A$
נוסחה. אזי ניתן לבנות נוסחה
$A'$
בצורת
\L{\small CNF}
שהיא שקולה ל-
$A$: $A\equiv A'$.
\end{theorem}

לא נביא כאן את פרטי ההוכחה. במקום זה, נדגים את המשפט על הנוסחה~%
\ref{eq.logic}:

\begin{center}
\begin{tabular}{l@{\hspace{2em}}l}
$(p \wedge \neg q) \;\rightarrow\; (\neg q \wedge p)$&\R{הנוסחה המקורית}\\
$\neg (p \wedge \neg q) \;\vee\; (\neg q \wedge p)$&$\rightarrow$ \R{סלק את האופרטור}\\
$(\neg p \vee \neg \neg q) \;\vee\; (\neg q \wedge p)$&\R{דחוף שלילה פנימה}\\
$(\neg p \vee q) \;\vee\; (\neg q \wedge p)$&$\neg\neg$ \R{סלק את זוג האופרטורים}\\
$(\neg p \vee q \vee \neg q) \;\wedge\; (\neg p \vee q \vee p)$&\R{חוק הפילוג}\\
\end{tabular}
\end{center}
הנימוקים לצעדים מבוססים על שקילויות לוגיות שניתן להוכיח:
\begin{eqnarray*}
A\rightarrow B &\equiv& \neg A \vee B\\
\neg (A \vee B) &\equiv& \neg A \wedge \neg B\\
\neg (A \wedge B) &\equiv& \neg A \vee \neg B\\
\neg\neg A &\equiv& A\\
A\vee (B\wedge C) &\equiv& (A \vee B) \wedge (A \vee C)\,.
\end{eqnarray*}
ההוכחה של משפט~
\ref{t.cnf}
היא סדרת למות כגון:
\begin{lemma}
תהי 
$A$
נוסחה. אזי קיימת נוסחה
$A'$
ללא האופרטור 
$\rightarrow$
כך ש-$A\equiv A'$.
\end{lemma}
הלמה תמימה למראה מהרגע שהוכחנו 
$A\rightarrow B \equiv \neg A \vee B$,
אבל ההוכחה מסתירה אינדוקציה לא-מפורשת:

\textbf{הוכחה}
טענת הבסיס היא כאשר הנוסחה
$A$
היא אטום
$p$,
אבל ברור שהאטום
$p$
לא מכיל את האופרטור
$\rightarrow$.
הנחת האינדוקציה היא שעבור כל נוסחה שגודלה פחות או שווה ל-%
$n$,%
\footnote{%
ניתן להגדיר את גודלה של נוסחה כגובה עץ הבניה שלה, אבל ההגדרה חורגת מתחום מסמך זה.%
}
קיימת נוסחה שקולה
$A'$
ללא 
$\rightarrow$.
תהי 
$A_1 \vee A_2$
נוסחה כך שגודלן של
$A_1, A_2$
פחות או שווה ל-%
$n$.
לפי הנחת האינדוקציה, קיימות נוסחאות 
$A_1', A_2'$
ללא 
$\rightarrow$,
כך ש-%
$A_1\equiv A_1'$
ו-%
$A_2\equiv A_2'$.
ברור ש-%
$A_1' \vee A_2'$
שקולה ל-%
$A$
ולא מכילה
$\rightarrow$.

צעדי אינדוקציה דומים מוכיחים את המשפט עבור
$\neg A_1$
ו-%
$A_1 \wedge A_2$.

עבור
$A_1 \rightarrow A_2$,
לפי הנחת האינדוקציה, קיימות נוסחאות
$A_1', A_2'$
ללא
$\rightarrow$,
כך ש-%
$A_1\equiv A_1'$
ו-%
$A_2\equiv A_2'$.
הנוסחה
$\neg A_1 \vee A_2 \equiv \neg A_1' \vee A_2'$,
שקולה ל-%
$A$
ולא מכילה
$\rightarrow$.\qed

כדי לסיים את ההוכחה של משפט~%
\ref{t.cnf}
אנחנו צריכים שלוש למות נוספות עם הוכחות דומות. כל אינדוקציה היא על 
\textbf{מבנה}
של הנוסחה, ולכן מספר הצעדים האינדוקטיביים הוא כמספר האופרטורים בלוגיקה.%
\footnote{%
מקובל להשתמש בלפחות ארבעה אופרטורים
$\neg,\vee,\wedge,\rightarrow$
ולפעמים ארבעה נוספים
$\leftrightarrow$ שקילות, $\oplus$ אי-שקילות, $\uparrow$ \L{nand}, $\downarrow$ \L{nor}.
\L{Raymond Smullyan}
הגדיר סימון פשוט יותר כך שיש רק שני צעדי אינדוקציה.}
בשום ספר לימוד בלוגיקה מתמטית לא תמצאו הוכחה כל כך מפורטת, כי ההוכחות ברורות משקילויות כגון
$A\rightarrow B\equiv\neg A \vee B$,
אבל חשוב להבין שמשתמשים באינדוקציה בצורה לא-מפורשת.

\section{%
מ-
\L{\small CNF}
ל-
\L{\small 3CNF}
}

\begin{definition}
נוסחה בצורת 
\L{\small CNF}
היא בצורת
\L{\small 3CNF}
אם ורק אם בכל איחוד יש בדיוק שלושה ליטרלים.
\end{definition}

\vspace*{-10pt}

\textbf{דוגמה}
הנוסחה~%
\ref{eq.cnf3}
היא ב-%
\L{\small 3CNF}.
\begin{theorem}\label{t.cnf3}
תהי
$A$
נוסחה בתחשיב הפסוקים. ניתן לבנות נוסחה
$A'$
ב-%
\L{\small 3CNF},
כך ש-%
$A$
ספיקה אם ורק אם
$A'$
ספיקה.
\end{theorem}

\vspace*{-8pt}

ההוכחה הרגילה של משפט~%
\ref{t.cnf3}
משתמשת באינדוקציה בצורה לא-מפורשת. כאן אנו במביאים הוכחה המציגה את האינדוקציה בצורה מפורשת.

\textbf{הוכחה}
מספיק להוכיח שעבור כל איחוד בודד
$A=x_1 \vee x_2 \vee \cdots \vee x_n$
ניתן לבנות חיתוך של איחודים עם שלושה ליטרלים כל אחד.

יש שלוש טענות בסיס:
\begin{itemize}
\item 
אם ל-%
$A$
שלושה ליטרלים, אין מה להוכיח.
\item
אם ל-%
$A$
שני ליטרלים
$p_1\vee p_2$,
הנוסחה
$A'$
תהיה:
\[
(p_1 \vee p_2 \vee q) \;\wedge\; (p_1 \vee p_2 \vee \neg q)\,,
\]
כאשר
$q$
הוא אטום חדש. אם
$A$
ספיקה, אזי ההשמה נותנת את הערך
$T$
או ל-%
$p_1$
או ל-%
$p_2$,
ולכן גם הערך של
$A'$
הוא
$T$.
בכיוון ההפוך, אם 
$A'$
ספיקה, אז אחת מהשמות
$\{p_1=F,p_2=F,q=T\}$
ו-%
$\{p_1=F,p_2=F,q=F\}$
מספקת את 
$A'$,
כך שאו
$p_1$
או
$p_2$
)או שניהם( צריכים לקבל את הערך
$T$.
מכאן שגם
$A$
ספיקה.

\item
\begin{exercise}
אם
$A$
הוא ליטרל בודד
$p$
או
$\neg p$,
קיימת נוסחה 
$A'$
עם בדיוק שלושה ליטרלים שהיא ספיקה אם ורק אם 
$A$
ספיקה.
\textbf{רמז}
כמה אטומים חדשים נחוצים?
\end{exercise}
\end{itemize}

הנחת האינדוקציה היא: אם
$A=p_1 \vee p_2 \vee \cdots \vee p_n$
היא איחוד עם 
$n\geq 3$
ליטרלים, אזי קיימת נוסחה
$A'$
בצורת
\L{\small 3CNF}
שהיא ספיקה אם ורק אם
$A$
ספיקה. הצעד האינדוקטיבי הוא: יהי
$A$
איחוד
$p_1 \vee p_2 \vee \cdots \vee p_n \vee p_{n+1}$
עם
$n+1$
ליטרלים. בנה
$A'$
כך:
\begin{displaymath}
(p_1 \vee \cdots \vee p_{n-1} \vee q) \;\wedge\; (\neg q \vee p_n \vee p_{n+1})\,,
\end{displaymath}
כאשר
$q$
הוא אטום חדש. ניתן להראות ש-%
$A'$
ספיקה אם ורק אם
$A$
ספיקה בהוכחה דומה להוכחה של טענת הבסיס עבור שני ליטרלים.

לפי הנחת האינדוקציה לאיחוד הראשון עם
$n$
ליטרלים קיימת נוסחה
$A''$
בצורת
\L{\small 3CNF}
שהיא ספיקה אם ורק אם נוסחת האיחוד הראשון ספיקה. מכאן, ש:
\begin{displaymath}
A'' \;\wedge\; (\neg q \vee p_n \vee p_{n+1})
\end{displaymath}
נוסחה בצורת
\L{\small 3CNF}
שהיא ספיקה אם ורק אם
$A$
ספיקה.
\qed

%%%%%%%%%%%%%%%%%%%%%%%%%%%%%%%%%%%%%%%%%%%%%%%%%%%%%%%%%%%%%%%%%%%
