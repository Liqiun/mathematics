%  Slides for showing that Minesweeper is NP-Complete
%  as proved by Richard Kaye,
%  see: http://www.mat.bham.ac.uk/R.W.Kaye/minesw/
%
% Copyright \copyright{} 2009 by Mordechai (Moti) Ben-Ari.
% This work is licensed under the Creative Commons
% Attribution-Noncommercial-ShareAlike 3.0 License. To view a copy of
% this license, visit
% \url{http://creativecommons.org/licenses/by-nc-sa/3.0/}; or, (b) send a letter 
% to Creative Commons, 543 Howard Street, 5th Floor, San Francisco, California, 94105, USA.

% \puztwo draws the configuration for Kaye's square of 2's
% \puzthree draws the same configuration with 3's

\newsavebox{\grid}
\sbox{\grid}{
\thicklines
  \put(0,0){\framebox(120,120){}}           % Outer box
  \multiput(20,0)(20,0){5}{\line(0,1){120}} % Vertical grid lines
  \multiput(0,20)(0,20){5}{\line(1,0){120}} % Horizontal grid lines
}

% A puzzle is a grid filled with zero and the parameter n
\newcommand{\puz}[1]{
  \thicklines
  \usebox{\grid}
  \multiput(20,20)(0,60){2}{
    \multiput(0,0)(20,0){4}{\makebox(20,20){\lrg #1}}
  }
  \multiput(20,40)(0,20){2}{
    \multiput(0,0)(60,0){2}{\makebox(20,20){\lrg #1}}
    \multiput(20,0)(20,0){2}{\makebox(20,20){\lrg 0}}
  }
}

% Save boxes for puzzles with 2's and 3's
\newsavebox{\puztwo}
\sbox{\puztwo}{\puz{2}}

\newsavebox{\puzthree}
\sbox{\puzthree}{\puz{3}}

% Commands for symbols
\newcommand{\mine}[0]{\makebox(20,20){\rule{9\lng}{9\lng}}}
\newcommand{\ques}[0]{\makebox(20,20){\lrg\bf ?}}
\newcommand{\open}[0]{\put(10,10){\circle{10}}}
\newcommand{\incon}[0]{\put(5,5){\line(1,1){10}}\put(5,15){\line(1,-1){10}}}
\newcommand{\fulfil}[0]{\put(4,4){\framebox(12,12){}}}
\newcommand{\smallmine}[0]{\makebox(10,10){\rule{4\lng}{4\lng}}}
\newcommand{\smallopen}[0]{\put(5,5){\circle{5}}}
\newcommand{\smallincon}[0]{\put(1,1){\line(1,1){8}}\put(1,9){\line(1,-1){8}}}

% Commands for each configuration
\newcommand{\configa}[0]{
  \usebox{\puztwo}
  \multiput(80,100)(20,0){2}{\mine}
  \put(60,100){\open}
  \multiput(100,80)(0,-20){2}{\open}
  \put(80,80){\fulfil}
  \put(80,60){\incon}
}

\newcommand{\configb}[0]{
  \usebox{\puztwo}
  \multiput(60,100)(40,0){2}{\mine}
  \put(80,100){\open}
  \multiput(100,80)(0,-20){2}{\open}
  \put(80,80){\fulfil}
  \put(80,60){\incon}
}

\newcommand{\configc}[0]{
  \usebox{\puztwo}
  \multiput(40,100)(60,0){2}{\mine}
  \multiput(60,100)(20,0){2}{\ques}
  \put(60,80){\incon}
}

\newcommand{\configd}[0]{
  \usebox{\puztwo}
  \multiput(20,100)(80,0){2}{\mine}
  \multiput(40,100)(20,0){3}{\ques}
  \multiput(40,80)(20,0){2}{\incon}
}

\newcommand{\confige}[0]{
  \usebox{\puztwo}
  \multiput(0,100)(100,0){2}{\mine}
  \multiput(20,100)(20,0){4}{\ques}
  \multiput(40,80)(20,0){2}{\incon}
}

\newcommand{\configf}[0]{
  \usebox{\puztwo}
  \multiput(60,100)(20,0){2}{\mine}
  \put(40,100){\open}
  \multiput(100,100)(0,-20){3}{\open}
  \multiput(60,80)(20,0){2}{\fulfil}
  \put(80,60){\incon}
}

\newcommand{\configg}[0]{
  \usebox{\puztwo}
  \multiput(40,100)(40,0){2}{\mine}
  \put(60,100){\open}
  \put(20,100){\ques}
  \put(60,80){\fulfil}
  \put(40,80){\incon}
}

\newcommand{\configh}[0]{
  \usebox{\puztwo}
  \multiput(20,100)(60,0){2}{\mine}
  \multiput(40,100)(20,0){2}{\ques}
  \multiput(40,80)(20,0){2}{\incon}
}

\newcommand{\configi}[0]{
  \usebox{\puztwo}
  \multiput(0,0)(0,100){2}{
    \multiput(40,0)(20,0){2}{\mine}
  }
  \multiput(0,20)(0,60){2}{
    \multiput(20,0)(20,0){4}{\fulfil}
  }
  \multiput(0,0)(100,0){2}{
    \multiput(0,40)(0,20){2}{\mine}
  }
  \multiput(20,0)(60,0){2}{
    \multiput(0,40)(0,20){2}{\fulfil}
  }
}

\newcommand{\configj}[0]{
  \usebox{\puzthree}
  \multiput(40,100)(20,0){3}{\mine}
  \put(60,80){\fulfil}
}

\newcommand{\configk}[0]{
  \usebox{\puzthree}
  \multiput(0,0)(0,100){2}{
    \multiput(20,0)(20,0){4}{\mine}
  }
  \multiput(0,20)(0,20){4}{
    \multiput(0,0)(100,0){2}{\mine}
  }
  \multiput(0,20)(0,60){2}{
    \multiput(40,0)(20,0){2}{\fulfil}
    \multiput(20,0)(60,0){2}{\incon}
  }
  \multiput(0,40)(0,20){2}{
    \multiput(20,0)(60,0){2}{\fulfil}
  }
}
