\documentclass[12pt,a4paper]{article}

\usepackage[utf8x]{inputenc}
\usepackage[english,hebrew]{babel}

\usepackage{verbatim}

\usepackage{tikz}
\usetikzlibrary{intersections,calc,through,arrows.meta}
\tikzset {>=Stealth}

\usepackage{url}

\newtheorem{theorem}{\R{משפט}}
\newtheorem{definition}{\R{הגדרה}}

\textwidth=15cm
\textheight=23cm
\topmargin=0pt
\headheight=0pt
\oddsidemargin=2em
\headsep=0pt
\parindent=0pt
\renewcommand{\baselinestretch}{1.15}
\setlength{\parskip}{0.3\baselineskip plus 1pt minus 1pt}

\newenvironment{form}[1]{%
\begin{displaymath}%
\renewcommand{\arraystretch}{#1}%
\begin{array}{lcl}}%
{\end{array}%
\end{displaymath}%
}

\begin{document}

\thispagestyle{empty}
\selectlanguage{hebrew}
\begin{center}
\textbf{\LARGE איך לשמור על מוזיאון}

\bigskip

\textbf{\Large מוטי בן-ארי\\\bigskip\bigskip\normalsize\url{http://www.weizmann.ac.il/sci-tea/benari/}}


\medskip
\end{center}

\selectlanguage{english}
\begin{footnotesize}
\begin{center}
\copyright{}\  2021 \R{מוטי בן-ארי}
\end{center}

\medskip

This work is licensed under the Creative Commons Attribution-ShareAlike 3.0 Unported License. To view a copy of this license, visit \url{http://creativecommons.org/licenses/by-sa/3.0/} or send a letter to Creative Commons, 444 Castro Street, Suite 900, Mountain View, California, 94041, USA.

\end{footnotesize}

\bigskip
\bigskip

\selectlanguage{hebrew}

מסמך זה מבוסס על פרק
$39$
של
\L{Martin Aigner and Günter M. Ziegler. \textit{Proofs from THE BOOK (Fifth Edition)}. Springer-Verlag, Berlin Heidelberg, 2014.}
הוספתי הרבה איורים והוכחות.

\section{מבוא}
ב-%
$1973$
\L{Victor Klee}
שאל כמה שומרים נחוצים כדי לראות את כל הקירות של מוזיאון כדי לוודא שלא גונבים את הציורים. אם הקירות של המוזיאון מהווים מצולע משוכלל או אפילו מצולע קמור, אפשר להסתפק בשומר אחד:
\begin{center}
\selectlanguage{english}
\begin{tikzpicture}[scale=.8]
\coordinate (O) at (0,0);
\fill (O) circle (2pt);
\foreach \x/\name/\n/\po in {0/a/A/right,.6/b/B/above,1.6/c/C/left,2.4/d/D/below left,3.9/e/E/below right} {
  \coordinate (\name) at ($(O)+(\x*72+18:3cm)$);
%  \fill (\name) circle (1.5pt);
%  \node[\po] at (\name) {$\n$};
\draw[dashed] (O) -- (\name);
}
\draw (a) -- (b) -- (c) -- (d) --(e) -- cycle;
\end{tikzpicture}
\end{center}

מה עם מוזיאון עם קירות בצורה של מסור:

\begin{center}
\selectlanguage{english}
\begin{tikzpicture}[scale=1]
\coordinate (O) at (0,0);
\draw [thick] (O) -- (++110:1cm) coordinate (P);
\draw[thick] (O) --
  ++(-70:1cm) coordinate(A) node[below] {$1$} -- 
  ++(+70:1cm) -- ++(0:1.5cm) --
  ++(-70:1cm) coordinate(B) node[below] {$2$} -- 
  ++(+70:1cm) -- ++(0:1.5cm) --
  ++(-70:1cm) coordinate(C) node[below] {$3$}-- 
  ++(+70:1cm) -- ++(0:1.5cm) --
  ++(-70:1cm) coordinate(D) node[below] {$4$} -- 
  ++(+70:1cm) -- ++(0:1.5cm) --
  ++(-70:1cm) coordinate(E) node[below] {$5$} --
  ++(+70:2cm) -- (P);

\end{tikzpicture}
\end{center}

וודא על ידי ספירה שיש 
$15$
קירות.

\newpage

נכסה בהצללה את המשולשים המוגדרים על ידי כל שן אל המסור:
\begin{center}
\selectlanguage{english}
\begin{tikzpicture}[scale=1]
\coordinate (O) at (0,0);
\draw [thick] (O) -- (++110:1cm) coordinate (P);
\draw[thick] (O) --
  ++(-70:1cm) coordinate(A) node[below] {$1$} -- 
  ++(+70:1cm) -- ++(0:1.5cm) --
  ++(-70:1cm) coordinate(B) node[below] {$2$} -- 
  ++(+70:1cm) -- ++(0:1.5cm) --
  ++(-70:1cm) coordinate(C) node[below] {$3$}-- 
  ++(+70:1cm) -- ++(0:1.5cm) --
  ++(-70:1cm) coordinate(D) node[below] {$4$} -- 
  ++(+70:1cm) -- ++(0:1.5cm) --
  ++(-70:1cm) coordinate(E) node[below] {$5$} --
  ++(+70:2cm) -- (P);

\draw[fill,black!20!white] (A) -- ++(110:2cm) -- ++(0:1.35cm)-- cycle;
\draw[fill,black!20!white] (B) -- ++(110:2cm) -- ++(0:1.35cm)-- cycle;
\draw[fill,black!20!white] (C) -- ++(110:2cm) -- ++(0:1.35cm)-- cycle;
\draw[fill,black!20!white] (D) -- ++(110:2cm) -- ++(0:1.35cm)-- cycle;
\draw[fill,black!20!white] (E) -- ++(110:2cm) -- ++(0:1.35cm)-- cycle;

\draw[->,red,very thick] (2.7,1) -- (B);
\draw[->,very thick,green] (9,.9) -- (C);
\draw[->,very thick,blue] ($(O)+(.5,.5)$) -- ++(7.5,.42);
\draw[->,very thick,blue] ($(O)+(.5,.5)$) -- ++(3,-.5);
\end{tikzpicture}
\end{center}

שומרת הניצבת בכל מקום בתוך אחד המשולשים יכולה לצפות על כל הקירות של המשולש )חץ אדום(. בנוסף, אם השומרת ניצבת בקירבת הקיר העליון לאורך כל המוזיאון, היא יכולה לצפות על כל הקירות האופקיים )חצים כחולים(. ברור ש-
$\frac{15}{3}=5$
שומרות מספיקות כדי לשמור על כל המוזיאון. אם המשולשים המוצללים לא חופפים, שומרת במשולש אחד לא יכולה לראות את כל הקירות של משולש אחר, לכן חייבים להעסיק 
$5$
שומרות.

ניתן להכליל את הדוגמה ל-%
$\frac{n}{3}$
שיניים עם
$n$
קירות, ולכן אפשר להסיק 
\textbf{שלפחות}
$\frac{n}{3}$
שומרות נחוצות בכל מוזיאון. המשפט שלהלן טוען ש-%
$\frac{n}{3}$
שומרות מספיקות כדי לשמור על כל מוזיאון.



\begin{theorem}\label{thm.guarded} 
ניתן לשמור על כל מוזיאון עם
$n$
קירות עם 
$\frac{n}{3}$
שומרות.%
\footnote{אם 
$n$
לא מתחלק ב-%
$3$
מספר השומרות הנחוצות הוא 
$\lfloor \frac{n}{3}\rfloor$.
למשל,
$4$
שומרות מספיקות כדי לשמור על מוזיאונים עם
$12, 13, 14$
קירות כי
$\lfloor \frac{14}{3}\rfloor =\lfloor \frac{13}{3}\rfloor=\lfloor \frac{12}{3}\rfloor=4$.
לשם הפשטות נתעלם מסיבוך זה.}
\end{theorem}

ניקח מצולע שרירותי, ייתכן עם קודקודים קעורים. קודקוד הוא 
\textbf{קמור}
אם הזווית הפנימית פחות מ-%
$180^\circ$.
קודקוד הוא
\textbf{קעור}
אם הזווית הפנימית גדולה מ-%
$180^\circ$.
במצולע באיור להלן, קודקוד 
$1$
קמור וקודקוד
$2$
קעור.

\begin{center}
\selectlanguage{english}
\begin{tikzpicture}[scale=.8]
\draw[thick]
  (0,0) coordinate (A) node[below left] {$1$} -- 
  ++(3,0) coordinate (B) --
  ++(2,2) coordinate (C) --
  ++(-1.5,-.5) coordinate (D) --
  ++(3,3) coordinate (E) -- 
  ++(-4,-1) coordinate (F) --
  ++(-2,1) coordinate (G) --
  ++(-1,-1) coordinate (H) --
  ++(.5,-1.5) coordinate (I) --
  ++(-2,-.5) coordinate (J) --
  ++(3,-.2) coordinate (K) node[right] {$2$} -- 
  ++(-4,-.3) coordinate (L) --
  cycle;
  
\foreach \point in {A,B,C,D,E,F,G,H,I,J,K,L}
  \fill (\point) circle(1.25pt);
\end{tikzpicture}
\end{center}

\begin{definition}
ניתן
\textbf{לתלת}
\L{(triangulate)}
מצולע אם ניתן לצייר 
\textbf{אלכסונים}---%
קטעי קו שאינם נחתכים המחברים קודקודים והנמצאים בתוך המצולע לכל אורכם%
---%
כך שהשטח הפנימי של המצולע מכוסה על ידי משולשים זרים אחד מהשני.
\end{definition}


\begin{theorem}\label{thm.tri}
ניתן לתלת כל מצולע.
\end{theorem}

אנו דוחים את ההוכחה של משפט~%
\L{\ref{thm.tri}}
לשלב מאוחר יותר.
\begin{definition}
ניתן
\textbf{לצבוע מצולע בשלושה צבעים}
אם קיים מיפוי
$\}$%
\R{אדום,כחול,ירוק}%
$c: V \mapsto \{$,
כך ששני הקודקודים של צלע מקבלים צבעים שונים.
\end{definition}

\begin{theorem}
ניתן לצבוע מצולע מתולת בשלושה צבעים.
\label{thm.colored}
\end{theorem}

\textbf{הוכחה}
באינדוקציה על מספר הקודקודים. ברור, שניתן לצבוע משולש בשלושה צבעים. ניתן מצולע עם 
$n>3$
קודקודים. חייבים לצייר לפחות אלכסון אחד כדי לתלת את המצולע. בחר אלכסון שרירותי
$\overline{AB}$:

\begin{center}
\selectlanguage{english}
\begin{tikzpicture}[scale=.8]
\draw[thick]
  (0,0) coordinate (A) -- 
  ++(3,0) coordinate (B) --
  ++(2,2) coordinate (C) --
  ++(-1.5,-.5) coordinate (D) --
  ++(3,3) coordinate (E) -- 
  ++(-4,-1) coordinate (F) --
  ++(-2,1) coordinate (G) --
  ++(-1,-1) coordinate (H) --
  ++(.5,-1.5) coordinate (I) --
  ++(-2,-.5) coordinate (J) --
  ++(3,-.2) coordinate (K) -- 
  ++(-4,-.3) coordinate (L) --
  cycle;
  
\foreach \point in {A,B,C,D,E,F,G,H,I,J,K,L}
  \fill (\point) circle(1.25pt);

\node[above right,xshift=4pt] at (K) {$A$};
\node[above left,xshift=-4pt,yshift=-2pt] at (D) {$B$};

\draw[thick,dashed]
  (B) -- (D) -- (K) -- (F) -- (I) -- (K) -- (A) -- (D) -- (F) -- (H);
\end{tikzpicture}
\end{center}
וחלק את המצולע לאורך אלכסון זה לשני מצולעים קטנים יותר:
\begin{center}
\selectlanguage{english}
\begin{tikzpicture}[scale=.8]
\path
  (0,0) coordinate (A1) -- 
  ++(3,0) coordinate (B1) --
  ++(2,2) coordinate (C1) --
  ++(-1.5,-.5) coordinate (D1);
\draw[thick]
  (D1) --
  ++(3,3) coordinate (E1) -- 
  ++(-4,-1) coordinate (F1) --
  ++(-2,1) coordinate (G1) --
  ++(-1,-1) coordinate (H1) --
  ++(.5,-1.5) coordinate (I1) --
  ++(-2,-.5) coordinate (J1) --
  ++(3,-.2) coordinate (K1);
\path
  (K1) -- 
  ++(-4,-.3) coordinate (L1) --
  (A1);
  
\foreach \point in {D1,E1,F1,G1,H1,I1,J1,K1}
  \fill (\point) circle(1.25pt);

\node[below,yshift=-4pt] at (K1) {$A$};
\node[below,yshift=-4pt] at (D1) {$B$};

\draw[thick,dashed]
  (D1) -- (F1) -- (I1) -- (K1) -- (F1) -- (H1);
\draw[thick] (D1) -- (K1);

\begin{scope}[yshift=-1.8cm]

\draw[thick]
  (0,0) coordinate (A2) -- 
  ++(3,0) coordinate (B2) --
  ++(2,2) coordinate (C2) --
  ++(-1.5,-.5) coordinate (D2);
\path
  (D2) --
  ++(3,3) coordinate (E2) --
  ++(-4,-1) coordinate (F2) --
  ++(-2,1) coordinate (G2) --
  ++(-1,-1) coordinate (H2) --
  ++(.5,-1.5) coordinate (I2) --
  ++(-2,-.5) coordinate (J2) --
  ++(3,-.2) coordinate (K2);
\draw[thick]
  (K2) --
  ++(-4,-.3) coordinate (L2) --
  (A2);
  
\foreach \point in {A2,B2,C2,D2,K2,L2}
  \fill (\point) circle(1.25pt);
  
\node[above,yshift=4pt] at (K2) {$A$};
\node[above,yshift=4pt] at (D2) {$B$};

\draw[thick,dashed]
  (K2) -- (A2) -- (D2) -- (B2) -- (D2);
\draw[thick] (D2) -- (K2);

\end{scope}
\end{tikzpicture}
\end{center}
לפי הנחת האינדוקציה, ניתן לצבוע כל אחד מהמצולעים הללו בשלושה צבעים:
\begin{center}
\selectlanguage{english}
\begin{tikzpicture}[scale=.8]
\path
  (0,0) coordinate (A1) -- 
  ++(3,0) coordinate (B1) --
  ++(2,2) coordinate (C1) --
  ++(-1.5,-.5) coordinate (D1);
\draw[thick]
  (D1) --
  ++(3,3) coordinate (E1) -- 
  ++(-4,-1) coordinate (F1) --
  ++(-2,1) coordinate (G1) --
  ++(-1,-1) coordinate (H1) --
  ++(.5,-1.5) coordinate (I1) --
  ++(-2,-.5) coordinate (J1) --
  ++(3,-.2) coordinate (K1);
\path
  (K1) -- 
  ++(-4,-.3) coordinate (L1) --
  (A1);
  
\foreach \point/\color in {D1/red,E1/blue,F1/green,G1/red,H1/blue,I1/red,J1/green,K1/blue}
  \fill[color=\color] (\point) circle(3pt);

\draw[thick,dashed]
  (D1) -- (F1) -- (I1) -- (K1) -- (F1) -- (H1);
\draw[thick] (D1) -- (K1);

\node[below,yshift=-4pt] at (K1) {$A$};
\node[below,yshift=-4pt] at (D1) {$B$};

\begin{scope}[yshift=-1.8cm]

\draw[thick]
  (0,0) coordinate (A2) -- 
  ++(3,0) coordinate (B2) --
  ++(2,2) coordinate (C2) --
  ++(-1.5,-.5) coordinate (D2);
\path
  (D2) --
  ++(3,3) coordinate (E2) --
  ++(-4,-1) coordinate (F2) --
  ++(-2,1) coordinate (G2) --
  ++(-1,-1) coordinate (H2) --
  ++(.5,-1.5) coordinate (I2) --
  ++(-2,-.5) coordinate (J2) --
  ++(3,-.2) coordinate (K2);
\draw[thick]
  (K2) --
  ++(-4,-.3) coordinate (L2) --
  (A2);
  
\foreach \point/\color in {A2/red,B2/blue,C2/red,D2/green,K2/blue,L2/green}
  \fill[color=\color] (\point) circle(3pt);

\draw[thick,dashed]
  (K2) -- (A2) -- (D2) -- (B2) -- (D2);
\draw[thick] (D2) -- (K2);
\node[above,yshift=4pt] at (K2) {$A$};
\node[above,yshift=4pt] at (D2) {$B$};

\end{scope}
\end{tikzpicture}
\end{center}

השיוך של צבעים לקודקודים הוא שרירותי, כך שאם הקודקודים 
$A,B$
מקבלים צבעים שונים בשני המצולעים, ניתן לשנות את הצבעים באחד מהם כך שהצבעים של 
$A,B$
שווים בשני המצולעים. כאן אנו מחליפים 
\textbf{אדום}
ו-%
\textbf{ירוק}
במצולע התחתון:
\begin{center}
\selectlanguage{english}
\begin{tikzpicture}[scale=.8]
\path
  (0,0) coordinate (A1) -- 
  ++(3,0) coordinate (B1) --
  ++(2,2) coordinate (C1) --
  ++(-1.5,-.5) coordinate (D1);
\draw[thick]
  (D1) --
  ++(3,3) coordinate (E1) -- 
  ++(-4,-1) coordinate (F1) --
  ++(-2,1) coordinate (G1) --
  ++(-1,-1) coordinate (H1) --
  ++(.5,-1.5) coordinate (I1) --
  ++(-2,-.5) coordinate (J1) --
  ++(3,-.2) coordinate (K1);
\path
  (K1) -- 
  ++(-4,-.3) coordinate (L1) --
  (A1);
  
\foreach \point/\color in {D1/red,E1/blue,F1/green,G1/red,H1/blue,I1/red,J1/green,K1/blue}
  \fill[color=\color] (\point) circle(3pt);
\node[below,yshift=-4pt] at (K1) {$A$};
\node[below,yshift=-4pt] at (D1) {$B$};

\draw[thick,dashed]
  (D1) -- (F1) -- (I1) -- (K1) -- (F1) -- (H1);
\draw[thick] (D1) -- (K1);

\begin{scope}[yshift=-1.8cm]

\draw[thick]
  (0,0) coordinate (A2) -- 
  ++(3,0) coordinate (B2) --
  ++(2,2) coordinate (C2) --
  ++(-1.5,-.5) coordinate (D2);
\path
  (D2) --
  ++(3,3) coordinate (E2) --
  ++(-4,-1) coordinate (F2) --
  ++(-2,1) coordinate (G2) --
  ++(-1,-1) coordinate (H2) --
  ++(.5,-1.5) coordinate (I2) --
  ++(-2,-.5) coordinate (J2) --
  ++(3,-.2) coordinate (K2);
\draw[thick]
  (K2) --
  ++(-4,-.3) coordinate (L2) --
  (A2);
  
\foreach \point/\color in {A2/green,B2/blue,C2/green,D2/red,K2/blue,L2/red}
  \fill[color=\color] (\point) circle(3pt);

\draw[thick,dashed]
  (K2) -- (A2) -- (D2) -- (B2) -- (D2);
\draw[thick] (D2) -- (K2);
\node[above,yshift=4pt] at (K2) {$A$};
\node[above,yshift=4pt] at (D2) {$B$};

\end{scope}
\end{tikzpicture}
\end{center}
כעת ניתן להדביק את שני המצולעים ביחד כדי לשחזר את המצולע המקורי עם
$n$
קודקודים. המצולע יהיה צבוע בשלושה צבעים:
\begin{center}
\selectlanguage{english}
\begin{tikzpicture}[scale=.8]
\draw[thick]
  (0,0) coordinate (A) -- 
  ++(3,0) coordinate (B) --
  ++(2,2) coordinate (C) --
  ++(-1.5,-.5) coordinate (D) --
  ++(3,3) coordinate (E) -- 
  ++(-4,-1) coordinate (F) --
  ++(-2,1) coordinate (G) --
  ++(-1,-1) coordinate (H) --
  ++(.5,-1.5) coordinate (I) --
  ++(-2,-.5) coordinate (J) --
  ++(3,-.2) coordinate (K) -- 
  ++(-4,-.3) coordinate (L) --
  cycle;
  
\foreach \point/\color in {D/red,E/blue,F/green,G/red,H/blue,I/red,J/green,K/blue,A/green,B/blue,C/green,L/red}
  \fill[color=\color] (\point) circle(3pt);

\node[above right,xshift=4pt] at (K) {$A$};
\node[above left,xshift=-4pt,yshift=-2pt] at (D) {$B$};

\draw[thick,dashed]
  (B) -- (D) -- (K) -- (F) -- (I) -- (K) -- (A) -- (D) -- (F) -- (H);
\end{tikzpicture}
\end{center}


\textbf{הוכחה של משפט
\L{\ref{thm.guarded}}}
לפי משפט
\L{\ref{thm.tri}}
ניתן לתלת את המצולע ולפי משפט
\L{\ref{thm.colored}}
ניתן לצבוע את המצולע בשלושה צבעים. כל שלושת הקודקודים של כל משולש חייבים להיות צבועים בצבעים שונים כדי לקיים את התנאי שהמצולע צבוע בשלושה צבעים. בגלל שהמצולע צבוע בשלושה צבעים, צבע אחד לפחות, נניח אדום, מופיע לכל היותר
$\frac{n}{3}$
פעמים, ובכל משולש חייב להיות קודקוד צבוע אדום. אם נציב שומרת בכל קודקוד אדום, היא יכולה לראות על הקירות של כל המשולשים שקודקוד זה הוא אחד מהקודקודים שלו. כל המשולשים של תילות המצולע כוללים את כל הצלעות של המצולע, ולכן
$\frac{n}{3}$
שומרות מספיקות כדי לראות את כל הקירות של המוזיאון.

\newpage

כעת נוכיח את משפט
\L{\ref{thm.tri}}
שניתן לתלת כל מצולע.

\begin{theorem}
סכום הזוויות הפנימיות של מצולע עם
$n$
צלעות הוא
$180^\circ(n-2)$.
\end{theorem}

\textbf{הוכחה}
תחילה נוכיח את המשפט עבור מצולעים קמורים. נסמן את 
\textbf{הזוויות החיצוניות}
ב-%
$\theta_i$:
\begin{center}
\selectlanguage{english}
\begin{tikzpicture}[scale=.6]
\coordinate (O) at (0,0);
%\fill (O) circle (2pt);
\foreach \x/\name/\n/\po in {0/a/A/right,.6/b/B/above,1.6/c/C/left,2.4/d/D/below left,3.9/e/E/below right} {
  \coordinate (\name) at ($(O)+(\x*72+18:3cm)$);
%  \fill (\name) circle (1.5pt);
%  \node[\po] at (\name) {$\n$};
%\draw[dashed] (O) -- (\name);
}
\draw[thick] (a) -- (b) -- (c) -- (d) --(e) -- cycle;

\draw[thick,dashed] (a) 
  node[above,xshift=-2pt,yshift=8pt] {$\theta_1$} -- 
  ($(a)!2!(b)$);
\draw[thick,dashed] (b)
  node[above left,xshift=-8pt,yshift=0pt] {$\theta_2$} -- 
  ($(b)!1.7!(c)$);
\draw[thick,dashed] (c) 
  node[below left,xshift=-4pt,yshift=-2pt] {$\theta_3$} -- 
  ($(c)!1.7!(d)$);
\draw[thick,dashed] (d)
  node[below right,xshift=0pt,yshift=-4pt] {$\theta_4$} -- 
  ($(d)!1.5!(e)$);
\draw[thick,dashed] (e)
  node[right,xshift=4pt,yshift=4pt] {$\theta_5$} -- 
  ($(e)!1.7!(a)$);

\end{tikzpicture}
\end{center}
אם נזיז קו מקווקוו אחד לשני הזוויות החיצוניות ישלימו מעגל כך ש-%
$\displaystyle\sum_1^n \theta_i = 360^\circ$.
עבור כל זוויות חיצונית
$\theta_i$,
נסמן את הזווית הפנימיות של אותו קודקוד ב-%
$\phi_i$.
נחשב:
\begin{form}{2}
\displaystyle\sum_1^n \theta_i =\displaystyle\sum_1^n (180^\circ-\phi_i)= 360^\circ\\
%n\cdot 180^\circ-\displaystyle\sum_1^n \phi_i &=& 360^\circ\\
\displaystyle\sum_1^n \phi_i = n\cdot 180^\circ-360^\circ =180^\circ(n-2)\,.
\end{form}
נבדוק מה קורה אם נוסיף קודקוד קעור:
\begin{center}
\selectlanguage{english}
\begin{tikzpicture}
\draw[thick] (0,0) -- (3,0) coordinate (A) node[above left,yshift=8pt] {$\alpha$} -- ++(60:2) coordinate (B) node[above,yshift=8pt] {$\beta$} -- ++(-60:2) coordinate (C) node[above right,yshift=8pt] {$\gamma$}  -- ++(3,0);

\draw ($(A)+(-.4,0)$) arc(180:60:.4);
\draw ($(B)+(-60:.3)$) arc(-60:240:.3);
\draw ($(C)+(.4,0)$) arc(0:120:.4);

\draw[thick,dashed] (A) -- (C);
\end{tikzpicture}
\end{center}
קיים משולש המורכב משני הצלעות שנוגעים בקודקוד הקעור והצלע המסומן בקו מקווקוו המחבר את הקודקודים שאחרים של הצלעות הללו. נסכם את הזוויות של המשולש:
\begin{form}{1.2}
(180^\circ - \alpha) + (360^\circ - \beta) + (180^\circ - \gamma) = 180^\circ\\
\alpha + \beta + \gamma = 3\cdot 180^\circ\,.
\end{form}
סכום הזוויות הפנימיות גדל ב-%
$\alpha+\beta+\gamma$
ומספר הקודקודים גדל בשלוש, כך שהמוואה במפשט נשמרת:
\begin{form}{1.4}
\displaystyle\sum_1^n \phi_i + (\alpha + \beta + \gamma) &=& 180^\circ(n-2)+3\cdot 180^\circ\\
&=& 180^\circ((n+3)-2)\,.
\end{form}
\begin{theorem}\label{thm.convex}
חייב להיות לפחות שלושה קודקודים קמורים במצולע.
\end{theorem}

\textbf{הוכחה}
נסמן ב-%
$k$
את מספר הקודקודים הקעורים, כאשר הזווית הפנימית של כל אחד הוא
$180^\circ+\epsilon_i$, $\epsilon_i>0$.
סכום הזוויות הפנימיות של הקודקודים
\textbf{הקעורים}
הוא בוודאי פחות או שווה לסכום
\textbf{כל}
הזוויות הפנימיות:
\begin{form}{1.4}
k\cdot 180^\circ +\displaystyle\sum_{i=1}^{k}\epsilon_i &\leq& 180^\circ(n-2)\\
(k+2)\cdot 180^\circ +\displaystyle\sum_{i=1}^{k}\epsilon_i &\leq& n\cdot 180^\circ\\
(k+2)\cdot 180^\circ &<& n\cdot 180^\circ\\
k&<&n-2\,.
\end{form}
מכאן שיש לא רק קודקוד אחד, אבל לפחות שלושה קודקודים שאינם קעורים אלא קמורים.

\textbf{הוכחה של משפט
\L{\ref{thm.tri}}}
באינדוקציה על מספר הקודקודים. עבור
$n=3$
אין מה להוכיח. נניח ש-%
$n>3$.
לפי משפט 
\L{\ref{thm.convex}},
חייב להיות קודקוד קמור
$C$.
סמנו את הקודקודים השכנים שלו
$B,D$.
אם
$\overline{BD}$
נמצא כולו בתוך המצולע אז הוא אלכסון וניתן לחלק את המצולע למשולש 
$\triangle BCD$
ולמצולע קטן יותר כאשר
$\overline{BD}$
הוא צלע. לפי הנחת האינדוקציה, ניתן לתלת את המצולע ואז להדביק אותו למשולש
$\triangle BCD$
ולקבל תילות של המצולע המקורי. אחרת, חייב להיות קודקוד קעור הקרוב ביותר ל-%
$C$.
$\overline{CF}$
הוא אלכסון המחלק את המצולע לשני מצולעים קטנים יותר. לפי הנחת האינדוקציה ניתן לתלת אותם ולהדביק אותם אחד לשני.
\begin{center}
\selectlanguage{english}
\begin{tikzpicture}[scale=2]
\clip (-2,-.2) rectangle (3.8,2.2);
\draw[thick]
  (0,0) coordinate (A) -- 
  ++(1.5,0) coordinate (B) --
  ++(2,2) coordinate (C) --
  ++(-1.8,-.5) coordinate (D) --
  ++(-1,.5) coordinate (E) --
  ++(1.3,-1) coordinate (F) --
  (A);
\draw[thick,dashed] (B) -- (D);
\draw[very thick,dotted] (C) -- (F);
\node [draw,circle through=(F)] at (C) {};
\foreach \point/\pos in {A/below,B/below,C/right,D/above,E/left,F/below}
  \fill (\point) circle(.7pt) node[\pos] {$\point$};
\end{tikzpicture}
\end{center}


\end{document}
